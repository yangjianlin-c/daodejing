\documentclass[a4paper,zihao=-4,twoside,landscape,UTF8]{ctexart}

%页面旋转
\usepackage{everypage}
\AddEverypageHook{\special{pdf: put @thispage <</Rotate 90>>}}

%% font文字旋转
\defaultCJKfontfeatures{RawFeature={vertical:+vert}}

%% 默认字体为源雲明體
\setCJKmainfont[BoldFont=源雲明體 TTF SemiBold,ItalicFont=源雲明體 TTF Light]{源雲明體 TTF Medium}

% 页码编号为汉字
%\renewcommand{\thepage}{\Chinese{page}}
\pagestyle{empty}

\usepackage{geometry}
\newgeometry{
top=50pt, bottom=65pt, left=150pt, right=80pt,
headsep=0pt,
}

\usepackage{titlesec}        % this can not be used when vertical mode
\titleformat{\section}[display]
          {\large\bfseries}
          {}{0pt}{}

\titlespacing{\section}{0pt}{0pt}{5pt}


\ctexset{
  today = big,
  punct = quanjiao, %% banjiao,
  autoindent = 0pt,
}



\title{黃帝內經}

\author{\normalsize 由 \textit{喝普洱茶的麥兜} 进行編輯整理及异体字注音}
\date{\normalsize\today 版}

\begin{document}
\maketitle
%\tableofcontents

\section{黃帝內經·素問}
昔在黃

\section{黃帝內經·靈樞}

昔在黃帝,\textbf{生而神靈},弱而能言,幼而徇齊,長而敦敏,成而登天。乃問於天師曰:余聞上古之人,春秋皆度百歲,而動作不衰;今時之人,年半百而動作皆衰者,時世異耶,人將失之耶。岐伯對曰:上古之人,其知道者,法於陰陽,和於術數,食飲有節,起居有常,不妄作勞,故能形與神俱,而盡終其天年,度百歲乃去。今時之人不然也,以酒為漿,以妄為常,醉以入房,以欲竭其精,以耗散其真,不知持滿,不時御神,務快其心,逆於生樂,起居無節,故半百而衰也。


夫上古聖人之教下也,皆謂之虛邪賊風,避之有時,恬淡虛無,真氣從之,精神內守,病安從來。是以志閒而少欲,心安而不懼,形勞而不倦,氣從以順,各從其欲,皆得所願。故美其食,任其服,樂其俗,高下不相慕,其民故曰朴。是以嗜欲不能勞其目,淫邪不能惑其心,愚智賢不肖不懼於物,故合於道。所以能年皆度百歲,而動作不衰者,以其德全不危也。

帝曰:人年老而無子者,材力盡耶,將天數然也。岐伯曰:女子七歲。腎氣盛,齒更髮長;二七而天癸至,任脈通,太衝脈盛,月事以時下,故有子;三七,腎氣平均,故真牙生而長極;四七,筋骨堅,髮長極,身體盛壯;五七,陽明脈衰,面始焦,發始墮;六七,三陽脈衰於上,面皆焦,發始白;七七,任脈虛,太衝脈衰少,天癸竭,地道不通,故形壞而無子也。丈夫八歲,腎氣實,髮長齒更;二八,腎氣盛,天癸至,精氣溢寫,陰陽和,故能有子;三八,腎氣平均,筋骨勁強,故真牙生而長極;四八,筋骨隆盛,肌肉滿壯;五八,腎氣衰,發墮齒槁;六八,陽氣衰竭於上,面焦,髮鬢頒白;七八,肝氣衰,筋不能動,天癸竭,精少,腎藏衰,形體皆極;八八,則齒發去。腎者主水,受五藏六府之精而藏之,故五藏盛,乃能寫。今五藏皆衰,筋骨解墮,天癸盡矣。故髮鬢白,身體重,行步不正,而無子耳。

%% 帝曰:有其年已老而有子者何也。岐伯曰:此其天壽過度,氣脈常通,而腎氣有餘也。此雖有子,男不過盡八八,女不過盡七七,而天地之精氣皆竭矣。

%% 帝曰:夫道者年皆百數,能有子乎。岐伯曰:夫道者能卻老而全形,身年雖壽,能生子也。

%% 黃帝曰:余聞上古有真人者,提挈天地,把握陰陽,呼吸精氣,獨立守神,肌肉若一,故能壽敝天地,無有終時,此其道生。中古之時,有至人者,淳德全道,和於陰陽,調於四時,去世離俗,積精全神,遊行天地之間,視聽八達之外,此蓋益其壽命而強者也,亦歸於真人。其次有聖人者,處天地之和,從八風之理,適嗜欲於世俗之間。無恚嗔之心,行不欲離於世,被服章,舉不欲觀於俗,外不勞形於事,內無思想之患,以恬愉為務,以自得為功,形體不敝,精神不散,亦可以百數。其次有賢人者,法則天地,像似日月,辨列星辰,逆從陰陽,分別四時,將從上古合同於道,亦可使益壽而有極時。



\end{document}
