\section{九針十二原第一}

黃帝問於岐伯曰:餘子萬民,養百姓,而收租稅。余哀其不給,而屬有疾病。余欲勿使被毒藥,無用砭石,欲以微針通其經脈,調其血氣,營其逆順出入之會。令可傳於後世,必明為之法。令終而不滅,久而不絕,易用難忘,為之經紀。異其章,別其表裡,為之終始。令各有形,先立針經,願聞其情。
岐伯答曰:臣請推而次之,令有綱紀,始於一,終於九焉。請言其道。小針之要,易陳而難入,粗守形,上守神,神乎神,客在門,未睹其疾,惡知其原。刺之微,在速遲,粗守關,上守機,機之動,不離其空,空中之機,清靜而微,其來不可逢,其往不可追。知機之道者,不可掛以發,不知機道,叩之不發。知其往來,要與之期,粗之暗乎,妙哉,工獨有之。往者為逆,來者為順,明知逆順,正行無問。逆而奪之,惡得無虛,追而濟之,惡得無實,迎之隨之,以意和之,針道畢矣。
凡用針者,虛則實之,滿則洩之,宛陳則除之,邪勝則虛之。大要曰:徐而疾則實,疾而徐則虛。言實與虛,若有若無,察後與先,若存若亡,為虛與實,若得若失。虛實之要,九針最妙,補瀉之時,以針為之。瀉曰:必持內之,放而出之,排陽得針,邪氣得洩。按而引針,是謂內溫,血不得散,氣不得出也。補曰隨之,隨之意若妄之,若行若按,如蚊虻止,如留如還,去如弦絕,令左屬右,其氣故止,外門已閉,中氣乃實,必無留血,急取誅之。持針之道,堅者為寶,正指直刺,無針左右,神在秋毫,屬意病者,審視血脈者,刺之無殆。方刺之時,必在懸陽,及與兩衛,神屬勿去,知病存亡。血脈者,在腧橫居,視之獨澄,切之獨堅。
九針之名,各不同形:一曰鑱針,長一寸六分;二曰員針,長一寸六分;三曰鍉(di2)針,長三寸半;四曰鋒針,長一寸六分;五曰鈹針,長四寸,廣二分半;六曰員利針,長一寸六分;七曰毫針,長三寸六分;八曰長針,長七寸;九曰大針,長四寸。鑱針者,頭大末銳,去瀉陽氣。員針者,針如卵形,揩摩分間,不得傷肌肉,以瀉分氣。鍉針者,鋒如黍粟之銳,主按脈勿陷,以致其氣。鋒針者,刃三隅,以發錮疾。鈹針者,末如劍鋒,以取大膿。員利針者,大如氂,且員且銳,中身微大,以取暴氣。毫針者,尖如蚊虻喙,靜以徐往,微以久留之而養,以取痛痹。長針者,鋒利身薄,可以取遠痹。大針者,尖如梃,其鋒微員,以瀉機關之水也。九針畢矣。
夫氣之在脈也,邪氣在上,濁氣在中,清氣在下。故針陷脈則邪氣出,針中脈則濁氣出,針太深則邪氣反沉,病益。故曰:皮肉筋脈各有所處,病各有所宜,各不同形,各以任其所宜。無實無虛,損不足而益有餘,是謂甚病,病益甚。取五脈者死,取三脈者恇;奪陰者死,奪陽者狂,針害畢矣。刺之而氣不至,無問其數;刺之而氣至,乃去之,勿復針。針各有所宜,各不同形,各任其所為。刺之要,氣至而有效,效之信,若風之吹雲,明乎若見蒼天,刺之道畢矣。
黃帝曰:願聞五藏六府所出之處。岐伯曰:五藏五腧,五五二十五腧;六府六腧,六六三十六腧。經脈十二,絡脈十五,凡二十七氣,以上下,所出為井,所溜為滎,所注為腧,所行為經,所入為合,二十七氣所行,皆在五腧也。節之交,三百六十五會,知其要者,一言而終,不知其要,流散無窮。所言節者,神氣之所遊行出入也,非皮肉筋骨也。觀其色,察其目,知其散復;一其形,聽其動靜,知其邪正。右主推之,左持而御之,氣至而去之。凡將用針,必先診脈,視氣之劇易,乃可以治也。五藏之氣已絕於內,而用針者反實其外,是謂重竭,重竭必死,其死也靜,治之者,輒反其氣,取腋與膺;五藏之氣已絕於外,而用針者反實其內,是謂逆厥,逆厥則必死,其死也躁,治之者,反取四末。刺之害中而不去,則精洩;害中而去,則致氣。精洩則病益甚而恇,致氣則生為癰瘍。
五藏有六府,六府有十二原,十二原出於四關,四關主治五藏。五藏有疾,當取之十二原。十二原者,五藏之所以稟三百六十五節氣味也。五藏有疾也,應出十二原,而原各有所出,明知其原,睹其應,而知五藏之害矣。陽中之少陰,肺也,其原出於太淵,太淵二。陽中之太陽,心也,其原出於大陵,大陵二。陰中之少陽,肝也,其原出於太沖,太沖二。陰中之至陰,脾也,其原出於太白,太白二。陰中之太陰,腎也,其原出於太溪,太溪二。膏之原,出於鳩尾,鳩尾一。肓之原,出於脖胦,脖胦一。凡此十二原者,主治五藏六府之有疾者也。脹取三陽,飧洩取三陰。
今夫五藏之有疾也,譬猶刺也,猶污也,猶結也,猶閉也。刺雖久,猶可拔也;污雖久,猶可雪也;結雖久,猶可解也;閉雖久,猶可決也。或言久疾之不可取者,非其說也。夫善用針者,取其疾也,猶拔刺也,猶雪污也,猶解結也,猶決閉也。疾雖久,猶可畢也。言不可治者,未得其術也。刺諸熱者,如以手探湯;刺寒清者,如人不欲行。陰有陽疾者,取之下陵三里,正往無殆,氣下乃止,不下復始也。疾高而內者,取之陰之陵泉;疾高而外者,取之陽之陵泉也。



\section{本輸第二}

黃帝問於岐伯曰:凡刺之道,必通十二經絡之所終始,絡脈之所別處,五輸之所留,六府之所與合,四時之所出入,五藏之所溜處,闊數之度,淺深之狀,高下所至。願聞其解。

岐伯曰:請言其次也。肺出於少商,少商者,手大指端內側也,為井木;溜於魚際,魚際者,手魚也,為滎;注於太淵,太淵,魚後一寸陷者中也,為腧;行於經渠,經渠,寸口中也,動而不居,為經;入於尺澤,尺澤,肘中之動脈也,為合,手太陰經也。心出於中沖,中沖,手中指之端也,為井木;溜於勞宮,勞宮,掌中中指本節之內間也,為滎;注於大陵,大陵,掌後兩骨之間方下者也,為腧;行於間使,間使之道,兩筋之間,三寸之中也,有過則至,無過則止,為經;入於曲澤,曲澤,肘內廉下陷者之中也,屈而得之,為合,手少陰也。肝出於大敦,大敦者,足大指之端及三毛之中也,為井木;溜於行間,行間,足大指間也,為滎;注於太沖,太沖,行間上二寸陷者之中也,為腧;行於中封,中封,內踝之前一寸半,陷者之中,使逆則宛,使和則通,搖足而得之,為經;入於曲泉,輔骨之下,大筋之上也,屈膝而得之,為合,足厥陰也。脾出於隱白,隱白者,足大指之端內側也,為井木;溜於大都,大都,本節之後,下陷者之中也,為滎;注於太白,太白,腕骨之下也,為腧;行於商丘,商丘,內踝之下,陷者之中也,為經;入於陰之陵泉,陰之陵泉,輔骨之下,陷者之中也,伸而得之,為合,足太陰也。腎出於湧泉,湧泉者,足心也,為井木;溜於然谷,然谷,然骨之下者也,為滎;注於太溪,太溪內踝之後,跟骨之上陷中者也,為腧;行於復留,復留,上內踝二寸,動而不休,為經;入於陰谷,陰谷,輔骨之後,大筋之下,小筋之上也,按之應手,屈膝而得之,為合,足少陰經也。
膀胱出於至陰,至陰者,足小指之端也,為井金;溜於通谷,通谷,本節之前外側也,為滎;注於束骨,束骨,本節之後,陷者中也,為腧;過於京骨,京骨,足外側大骨之下,為原;行於崑崙,崑崙,在外踝之後,跟骨之上,為經;入於委中,委中,膕中央,為合,委而取之,足太陽也。膽出於竅陰,竅陰者,足小指次指之端也,為井金;溜於俠溪,俠溪,足小指次指之間也,為滎;注於臨泣,臨泣,上行一寸半陷者中也,為腧;過於丘墟,丘墟,外踝之前下,陷者中也,為原;行於陽輔,陽輔,外踝之上,輔骨之前,及絕骨之端也,為經;入於陽之陵泉,陽之陵泉,在膝外陷者中也,為合,伸而得之,足少陽也。胃出於厲兌,厲兌者,足大指內次指之端也,為井金;溜於內庭,內庭,次指外間也,為滎;注於陷谷,陷谷者,上中指內間上行二寸陷者中也,為腧;過於沖陽,沖陽,足跗上五寸陷者中也,為原,搖足而得之;行於解溪,解溪,上衝陽一寸半陷者中也,為經;入於下陵,下陵,膝下三寸,胻骨外三里也,為合;復下三里三寸為巨虛上廉,復下上廉三寸為巨虛下廉也,大腸屬上,小腸屬下,足陽明胃脈也,大腸小腸,皆屬於胃,是足陽明也。三焦者,上合手少陽,出於關沖,關沖者,手小指次指之端也,為井金;溜於液門,液門,小指次指之間也,為滎;注於中渚,中渚,本節之後陷者中也,為腧;過於陽池,陽池,在腕上陷者之中也,為原;行於支溝,支溝,上腕三寸,兩骨之間陷者中也,為經;入於天井,天井,在肘外大骨之上陷者中也,為合,屈肘乃得之;三焦下腧,在於足大指之前,少陽之後,出於膕中外廉,名曰委陽,是太陽絡也。手少陽經也。三焦者,足少陽太陰之所將,太陽之別也,上踝五寸,別入貫腨腸,出於委陽,並太陽之正,入絡膀胱,約下焦,實則閉癃,虛則遺溺,遺溺則補之,閉癃則瀉之。手太陽小腸者,上合手太陽,出於少澤,少澤,小指之端也,為井金;溜於前谷,前谷,在手外廉本節前陷者中也,為滎;注於後溪,後溪者,在手外側本節之後也,為腧;過於腕骨,腕骨,在手外側腕骨之前,為原;行於陽谷,陽谷,在銳骨之下陷者中也,為經;入於小海,小海,在肘內大骨之外,去端半寸陷者中也,伸臂而得之,為合,手太陽經也。大腸上合手陽明,出於商陽,商陽,大指次指之端也,為井金;溜於本節之前二間,為滎;注於本節之後三間,為腧;過於合谷,合谷,在大指岐骨之間,為原;行於陽溪,陽溪,在兩筋間陷者中也,為經;入於曲池,在肘外輔骨陷者中,屈臂而得之,為合,手陽明也。是謂五藏六府之腧,五五二十五腧,六六三十六腧也。六府皆出足之三陽,上合於手者也。

缺盆之中,任脈也,名曰天突,一。次任脈側之動脈,足陽明也,名曰人迎,二。次脈手陽明也,名曰扶突,三。次脈手太陽也,名曰天窗,四。次脈足少陽也,名曰天容,五。次脈手少陽也,名曰天牖,六。次脈足太陽也,名曰天柱,七。次脈頸中央之脈,督脈也,名曰風府。腋內動脈,手太陰也,名曰天府。腋下三寸,手心主也,名曰天池。刺上關者,呿不能欠;刺下關者,欠不能呿。刺犢鼻者,屈不能伸;刺兩關者,伸不能屈。

足陽明挾喉之動脈也,其腧在膺中。手陽明次在其腧外,不至曲頰一寸。手太陽當曲頰。足少陽在耳下曲頰之後。手少陽出耳後,上加完骨之上。足太陽挾項大筋之中髮際。陰尺動脈在五里,五腧之禁也。

肺合大腸,大腸者,傳道之府。心合小腸,小腸者,受盛之府。肝合膽,膽者,中精之府。脾合胃,胃者,五穀之府。腎合膀胱,膀胱者,津液之府也。少陽屬腎,腎上連肺,故將兩藏。三焦者,中瀆之府也,水道出焉,屬膀胱,是孤之府也,是六府之所與合者。

春取絡脈諸滎大經分肉之間,甚者深取之,間者淺取之。夏取諸腧孫絡肌肉皮膚之上。秋取諸合,余如春法。冬取諸井諸腧之分,欲深而留之。此四時之序,氣之所處,病之所舍,藏之所宜。轉筋者,立而取之,可令遂已。痿厥者,張而刺之,可令立快也。



\section{小針解第三}

所謂易陳者,易言也。難入者,難著於人也。粗守形者,守刺法也。上守神者,守人之血氣有餘不足,可補瀉也。神客者,正邪其會也。神者,正氣也。客者,邪氣也。在門者,邪循正氣之所出入也。未睹其疾者,先知邪正何經之疾也。惡知其原者,先知何經之病所取之處也。刺之微在數遲者,徐疾之意也。粗守關者,守四支而不知血氣正邪之往來也。上守機者,知守氣也。機之動不離其空中者,知氣之虛實,用針之徐疾也。空中之機清淨以微者,針以得氣,密意守氣勿失也。其來不可逢者,氣盛不可補也。其往不可追者,氣虛不可瀉也。不可掛以發者,言氣易失也。扣之不發者,言不知補瀉之意也,血氣已盡而氣不下也。知其往來者,知氣之逆順盛虛也。要與之期者,知氣之可取之時也。粗之暗者,冥冥不知氣之微密也。妙哉!工獨有之者,盡知針意也。往者為逆者,言氣之虛而小,小者逆也。來者為順者,言形氣之平,平者順也。明知逆順,正行無問者,言知所取之處也。迎而奪之者,瀉也。追而濟之者,補也。

所謂虛則實之者,氣口虛而當補之也。滿則洩之者,氣口盛而當瀉之也。宛陳則除之者,去血脈也。邪勝則虛之者,言諸經有盛者,皆瀉其邪也。徐而疾則實者,言徐內而疾出也。疾而徐則虛者,言疾內而徐出也。言實與虛若有若無者,言實者有氣,虛者無氣也。察後與先若亡若存者,言氣之虛實,補瀉之先後也,察其氣之已下與常存也。為虛與實若得若失者,言補者佖然若有得也,瀉則怳然若有失也。

夫氣之在脈也,邪氣在上者,言邪氣之中人也高,故邪氣在上也。濁氣在中者,言水谷皆入於胃,其精氣上注於肺,濁溜於腸胃,言寒溫不適,飲食不節,而病生於腸胃,故命曰濁氣在中也。清氣在下者,言清濕地氣之中人也,必從足始,故曰清氣在下也。針陷脈則邪氣出者,取之上。針中脈則濁氣出者,取之陽明合也。針太深則邪氣反沉者,言淺浮之病,不欲深刺也,深則邪氣從之入,故曰反沉也。皮肉筋脈各有所處者,言經絡各有所主也。取五脈者死,言病在中,氣不足,但用針盡大瀉其諸陰之脈也。取三陽之脈者,唯言盡瀉三陽之氣,令病人恇然不復也。奪陰者死,言取尺之五里五往者也。奪陽者狂,正言也。

睹其色、察其目、知其散復、一其形、聽其動靜者,言上工知相五色於目,有知調尺寸大小緩急滑澀,以言所病也。知其邪正者,知論虛邪與正邪之風也。右主推之、左持而御之者,言持針而出入也。氣至而去之者,言補瀉氣調而去之也。調氣在於終始一者,持心也。節之交三百六十五會者,絡脈之滲灌諸節者也。所謂五藏之氣已絕於內者,脈口氣內絕不至,反取其外之病處與陽經之合,有留針以致陽氣,陽氣至則內重竭,重竭則死矣,其死也無氣以動,故靜。所謂五藏之氣已絕於外者,脈口氣外絕不至,反取其四末之輸,有留針以致其陰氣,陰氣至則陽氣反入,入則逆,逆則死矣,其死也陰氣有餘,故躁。所以察其目者,五藏使五色循明,循明則聲章,聲章者,則言聲與平生異也。




\section{邪氣藏府病形第四}

黃帝問於岐伯曰:邪氣之中人也奈何?岐伯答曰:邪氣之中人高也。黃帝曰:高下有度乎?岐伯曰:身半已上者,邪中之也;身半已下者,濕中之也。故曰,邪之中人也,無有常,中於陰則溜於府,中於陽則溜於經。黃帝曰:陰之與陽也,異名同類,上下相會,經絡之相貫,如環無端。邪之中人,或中於陰,或中於陽,上下左右,無有恆常,其故何也?岐伯曰:諸陽之會,皆在於面。中人也方乘虛時,及新用力,若飲食汗出腠理開,而中於邪。中於面則下陽明,中於項則下太陽,中於頰則下少陽,其中於膺背兩脅亦中其經。黃帝曰:其中於陰奈何?岐伯答曰:中於陰者,當從臂胻始。夫臂與胻,其陰皮薄,其肉淖澤,故俱受於風,獨傷其陰。黃帝曰:此故傷其藏乎?岐伯答曰:身之中於風也,不必動藏。故邪入於陰經,則其藏氣實,邪氣入而不能客,故還之於府。故中陽則溜於經,中陰則溜於府。黃帝曰:邪之中人藏奈何?岐伯曰:愁憂恐懼則傷心。形寒寒飲則傷肺,以其兩寒相感,中外皆傷,故氣逆而上行。有所墮墜,惡血留內,若有所大怒,氣上而不下,積於脅下,則傷肝。有所擊僕,若醉入房,汗出當風,則傷脾。有所用力舉重,若入房過度,汗出浴水,則傷腎。黃帝曰:五藏之中風奈何?岐伯曰:陰陽俱感,邪乃得往。黃帝曰:善哉。

帝問於岐伯曰:首面與身形也,屬骨連筋,同血合於氣耳。天寒則裂地凌冰,其卒寒或手足懈惰,然而其面不衣何也?岐伯答曰:十二經脈,三百六十五絡,其血氣皆上於面而走空竅,其精陽氣上走於目而為睛,其彆氣走於耳而為聽,其宗氣上出於鼻而為臭,其濁氣出於胃,走唇舌而為味。其氣之津液皆上熏於面,而皮又厚,其肉堅,故天氣甚寒不能勝之也。

黃帝曰:邪之中人,其病形何如?岐伯曰:虛邪之中身也,灑淅動形。正邪之中人也微,先見於色,不如於身,若有若無,若亡若存,有形無形,莫知其情。黃帝曰:善哉。

黃帝問於岐伯曰:余聞之,見其色,知其病,命曰明;按其脈,知其病,命曰神;問其病,知其處,命曰工。余願聞見而知之,按而得之,問而極之,為之奈何?岐伯答曰:夫色脈與尺之相應也,如桴鼓影響之相應也,不得相失也,此亦本末根葉之出候也,故根死則葉枯矣。色脈形肉不得相失也,故知一則為工,知二則為神,知三則神且明矣。黃帝曰:願卒聞之。岐伯答曰:色青者,其脈弦也;赤者,其脈鉤也;黃者,其脈代也;白者,其脈毛;黑者,其脈石。見其色而不得其脈,反得其相勝之脈,則死矣;得其相生之脈,則病已矣。黃帝問於岐伯曰:五藏之所生,變化之病形,何如?岐伯答曰:先定其五色五脈之應,其病乃可別也。黃帝曰:色脈已定,別之奈何?岐伯曰:調其脈之緩、急、小、大、滑、澀,而病變定矣。黃帝曰:調之奈何?岐伯答曰:脈急者,尺之皮膚亦急;脈緩者,尺之皮膚亦緩;脈小者,尺之皮膚亦減而少氣;脈大者,尺之皮膚亦賁而起;脈滑者,尺之皮膚亦滑;脈澀者,尺之皮膚亦澀。凡此變者,有微有甚。故善調尺者,不待於寸,善調脈者,不待於色。能參合而行之者,可以為上工,上工十全九;行二者,為中工,中工十全七;行一者,為下工,下工十全六。

黃帝曰:請問脈之緩、急、小、大、滑、澀之病形何如?岐伯曰:臣請言五藏之病變也。心脈急甚者為瘛瘲;微急為心痛引背,食不下。緩甚為狂笑;微緩為伏梁,在心下,上下行,時唾血。大甚為喉吤,微大為心痹引背,善淚出。小甚為善噦,微小為消癉。滑甚為善渴;微滑為心疝引臍,小腹鳴。澀甚為瘖;微澀為血溢,維厥,耳鳴,顛疾。

肺脈急甚為癲疾;微急為肺寒熱,怠惰,咳唾血,引腰背胸,若鼻息肉不通。緩甚為多汗;微緩為痿瘻,偏風,頭以下汗出不可止。大甚為脛腫;微大為肺痹引胸背,起惡日光。小甚為洩,微小為消癉。滑甚為息賁上氣,微滑為上下出血。澀甚為嘔血;微澀為鼠瘻,在頸支腋之間,下不勝其上,其應善(疒峻-山)矣。

肝脈急甚者為惡言;微急為肥氣在脅下,若覆杯。緩甚為善嘔,微緩為水瘕痹也。大甚為內癰,善嘔衄,微大為肝痹陰縮,咳引小腹。小甚為多飲,微小為消癉。滑甚為憒疝,微滑為遺溺。澀甚為溢飲,微澀為瘈攣筋痹。

脾脈急甚為瘈瘲;微急為膈中,食飲入而還出,後沃沫。緩甚為痿厥;微緩為風痿,四肢不用,心慧然若無病。大甚為擊僕;微大為疝氣,腹裡大膿血,在腸胃之外。小甚為寒熱,微小為消癉。滑甚為(疒貴)癃,微滑為蟲毒蛕蠍蛸腹熱。澀甚為腸(疒貴);微澀為內(疒貴),多下膿血。

腎脈急甚為骨癲疾;微急為沉厥奔豚,足不收,不得前後。緩甚為折脊;微緩為洞,洞者,食不化,下嗌還出。大甚為陰痿;微大為石水,起臍以下至小腹腄腄然,上至胃脘,死不治。小甚為洞洩,微小為消癉。滑甚為癃(疒貴);微滑為骨痿,坐不能起,起則目無所見。澀甚為大癰,微澀為不月沉痔。

黃帝曰:病之六變者,刺之奈何?岐伯答曰:諸急者多寒;緩者多熱;大者多氣少血;小者血氣皆少;滑者陽氣盛,微有熱;澀者多血少氣,微有寒。是故刺急者,深內而久留之。刺緩者,淺內而疾髮針,以去其熱。刺大者,微瀉其氣,無出其血。刺滑者,疾髮針而淺內之,以瀉其陽氣而去其熱。刺澀者,必中其脈,隨其逆順而久留之,必先按而循之,已髮針,疾按其痏,無令其血出,以和其脈。諸小者,陰陽形氣俱不足,勿取以針,而調以甘藥也。

黃帝曰:余聞五藏六府之氣,滎輸所入為合,令何道從入,入安連過,願聞其故。岐伯答曰:此陽脈之別入於內,屬於府者也。黃帝曰:滎輸與合,各有名乎?岐伯答曰:滎輸治外經,合治內府。黃帝曰:治內府奈何?岐伯曰:取之於合。黃帝曰:合各有名乎?岐伯答曰:胃合於三里,大腸合入於巨虛上廉,小腸合入於巨虛下廉,三焦合入於委陽,膀胱合入於委中央,膽合入於陽陵泉。黃帝曰:取之奈何?岐伯答曰:取之三里者,低跗;取之巨虛者,舉足;取之委陽者,屈伸而索之;委中者,屈而取之;陽陵泉者,正豎膝予之齊下至委陽之陽取之。取諸外經者,揄申而從之。

黃帝曰:願聞六府之病。岐伯答曰:面熱者足陽明病,魚絡血者手陽明病,兩跗之上脈豎陷者足陽明病,此胃脈也。大腸病者,腸中切痛而鳴濯濯,冬日重感於寒即洩,當臍而痛,不能久立,與胃同候,取巨虛上廉。胃病者,腹(月真)脹,胃脘當心而痛,上支兩脅,膈咽不通,飲食不下,取之三里也。小腸病者,小腹痛,腰脊控睾而痛,時窘之後,當耳前熱。若寒甚,若獨肩上熱甚,及手小指次指之間熱,若脈陷者,此其候也,手太陽病也,取之巨虛下廉。三焦病者,腹氣滿,小腹尤堅,不得小便,窘急,溢則水,留即為脹,候在足太陽之外大絡,大絡在太陽少陽之間,亦見於脈,取委陽。膀胱病者,小腹偏腫而痛,以手按之,即欲小便而不得,肩上熱若脈陷,及足小指外廉及脛踝後皆熱若脈陷,取委中央。膽病者,善太息,口苦,嘔宿汁,心下澹澹,恐人將捕之,嗌中吤吤然,數唾,在足少陽之本末,亦視其脈之陷下者灸之,其寒熱者取陽陵泉。黃帝曰:刺之有道乎?岐伯答曰:刺此者,必中氣穴,無中肉節,中氣穴則針染於巷,中肉節即皮膚痛,補瀉反則病益篤。中筋則筋緩,邪氣不出,與其真相搏,亂而不去,反還內著,用針不審,以順為逆也。




\section{根結第五}

岐伯曰:天地相感,寒暖相移,陰陽之道,孰少孰多?陰道偶,陽道奇,發於春夏,陰氣少,陽氣多,陰陽不調,何補何瀉?發於秋冬,陽氣少,陰氣多,陰氣盛而陽氣衰,故莖葉枯槁,濕雨下歸,陰陽相移,何瀉何補?奇邪離經,不可勝數,不知根結,五藏六府,折關敗樞,開合而走,陰陽大失,不可復取。九針之玄,要在終始,故能知終始,一言而畢,不知終始,針道咸絕。

太陽根於至陰,結於命門,命門者目也。陽明根於厲兌,結於顙大,顙大者鉗耳也。少陽根於竅陰,結於窗籠,窗籠者耳中也。太陽為開,陽明為合,少陽為樞。故開折則肉節瀆而暴病起矣,故暴病者取之太陽,視有餘不足,瀆者皮肉宛膲而弱也。合折則氣無所止息而痿疾起矣,故痿疾者取之陽明,視有餘不足,無所止息者,真氣稽留,邪氣居之也。樞折即骨繇而不安於地,故骨繇者取之少陽,視有餘不足,骨繇者節緩而不收也,所謂骨繇者搖故也,當窮其本也。太陰根於隱白,結於太倉。少陰根於湧泉,結於廉泉。厥陰根於大敦,結於玉英,絡於羶中。太陰為開,厥陰為合,少陰為樞。故開折則倉廩無所輸膈洞,膈洞者取之太陰,視有餘不足,故開折者氣不足而生病也。合折即氣絕而喜悲,悲者取之厥陰,視有餘不足。樞折則脈有所結而不通,不通者取之少陰,視有餘不足,有結者皆取之不足。

足太陽根於至陰,溜於京骨,注於崑崙,入於天柱、飛揚也。足少陽根於竅陰,溜於丘墟,注於陽輔,入於天容、光明也。足陽明根於厲兌,溜於沖陽,注於下陵,入於人迎、豐隆也。手太陽根於少澤,溜於陽谷,注於少海,入於天窗、支正也。手少陽根於關沖,溜於陽池,注於支溝,入於天牖、外關也。手陽明根於商陽,溜於合谷,注於陽溪,入於扶突、偏歷也。此所謂十二經者,盛絡皆當取之。

一曰一夜五十營,以營五藏之精,不應數者,名曰狂生。所謂五十營者,五藏皆受氣。持其脈口,數其至也,五十動而不一代者,五藏皆受氣;四十動一代者,一藏無氣;三十動一代者,二藏無氣;二十動一代者,三藏無氣;十動一代者,四藏無氣;不滿十動一代者,五藏無氣。予之短期,要在終始。所謂五十動而不一代者,以為常也,以知五藏之期。予之短期者,乍數乍疏也。

黃帝曰:逆順五體者,言人骨節之小大,肉之堅脆,皮之厚薄,血之清濁,氣之滑澀,脈之長短,血之多少,經絡之數,余已知之矣,此皆布衣匹夫之士也。夫王公大人,血食之君,身體柔脆,肌肉軟弱,血氣慓悍滑利,其刺之徐疾淺深多少,可得同之乎?岐伯答曰:膏梁菽藿之味,何可同也。氣滑即出疾,其氣澀則出遲,氣悍則針小而入淺,氣澀則針大而入深,深則欲留,淺則欲疾。以此觀之,刺布衣者深以留之,刺大人者微以徐之,此皆因氣慓悍滑利也。

黃帝曰:形氣之逆順奈何?岐伯曰:形氣不足,病氣有餘,是邪勝也,急瀉之。形氣有餘,病氣不足,急補之。形氣不足,病氣不足,此陰陽氣俱不足也,不可刺之,刺之則重不足,重不足則陰陽俱竭,血氣皆盡,五藏空虛,筋骨髓枯,老者絕滅,壯者不復矣。形氣有餘,病氣有餘,此謂陰陽俱有餘也,急瀉其邪,調其虛實。故曰有餘者瀉之,不足者補之,此之謂也。故曰刺不知逆順,真邪相搏。滿而補之,則陰陽四溢,腸胃充郭,肝肺內(月真),陰陽相錯。虛而瀉之,則經脈空虛,血氣竭枯,腸胃(亻聶)辟,皮膚薄著,毛腠夭膲,予之死期。故曰用針之要,在於知調陰與陽,調陰與陽,精氣乃光,合形與氣,使神內藏。故曰上工平氣,中工亂脈,下工絕氣危生。故曰下工不可不慎也。必審五藏變化之病,五脈之應,經絡之實虛,皮之柔粗,而後取之也。

\section{壽夭剛柔第六}

黃帝問於少師曰:余聞人之生也,有剛有柔,有弱有強,有短有長,有陰有陽,願聞其方。少師答曰:陰中有陰,陽中有陽,審知陰陽,刺之有方,得病所始,刺之有理,謹度病端,與時相應,內合於五藏六府,外合於筋骨皮膚。是故內有陰陽,外亦有陰陽。在內者,五藏為陰,六府為陽;在外者,筋骨為陰,皮膚為陽。故曰病在陰之陰者,刺陰之滎輸;病在陽之陽者,刺陽之合;病在陽之陰者,刺陰之經;病在陰之陽者,刺絡脈。故曰病在陽者命曰風,病在陰者命曰痹,陰陽俱病命曰風痹。病有形而不痛者,陽之類也;無形而痛者,陰之類也。無形而痛者,其陽完而陰傷之也,急治其陰,無攻其陽;有形而不痛者,其陰完而陽傷之也。急治其陽,無攻其陰。陰陽俱動,乍有形,乍無形,加以煩心,命曰陰勝其陽,此謂不表不裡,其形不久。

黃帝問於伯高曰:余聞形氣病之先後,外內之應奈何?伯高答曰:風寒傷形,憂恐忿怒傷氣。氣傷藏,乃病藏;寒傷形,乃應形;風傷筋脈,筋脈乃應。此形氣外內之相應也。黃帝曰:刺之奈何?伯高答曰:病九日者,三刺而已。病一月者,十刺而已。多少遠近,以此衰之。久痹不去身者,視其血絡,盡出其血。黃帝曰:外內之病,難易之治奈何?伯高答曰:形先病而未入藏者,刺之半其日;藏先病而形乃應者,刺之倍其日。此外內難易之應也。

黃帝問於伯高曰:余聞形有緩急,氣有盛衰,骨有大小,肉有堅脆,皮有厚薄,其以立壽夭奈何?伯高曰:形與氣相任則壽,不相任則夭。皮與肉相果則壽,不相果則夭。血氣經絡勝形則壽,不勝形則夭。黃帝曰:何謂形之緩急?伯高答曰:形充而皮膚緩者則壽,形充而皮膚急者則夭。形充而脈堅大者順也,形充而脈小以弱者氣衰,衰則危矣。若形充而顴不起者骨小,骨小而夭矣。形充而大肉(月囷)堅而有分者肉堅,肉堅則壽矣;形充而大肉無分理不堅者肉脆,肉脆則夭矣。此天之生命,所以立形定氣而視壽夭者。必明乎此立形定氣,而後以臨病人,決死生。黃帝曰:余聞壽夭,無以度之。伯高答曰:牆基卑,高不及其地者,不滿三十而死;其有因加疾者,不及二十而死也。黃帝曰:形氣之相勝,以立壽夭奈何?伯高答曰:平人而氣勝形者壽;病而形肉脫,氣勝形者死,形勝氣者危矣。
黃帝曰:余聞刺有三變,何謂三變?伯高答曰:有刺營者,有刺衛者,有刺寒痹之留經者。黃帝曰:刺三變者奈何?伯高答曰:刺營者出血,刺衛者出氣,刺寒痹者內熱。黃帝曰:營衛寒痹之為病奈何?伯高答曰:營之生病也,寒熱少氣,血上下行。衛之生病也,氣痛時來時去,怫愾賁響,風寒客於腸胃之中。寒痹之為病也,留而不去,時痛而皮不仁。黃帝曰:刺寒痹內熱奈何?伯高答曰:刺布衣者,以火焠之。刺大人者,以藥熨之。黃帝曰:藥熨奈何?伯高答曰:用淳酒二十斤,蜀椒一升,乾薑一斤,桂心一斤,凡四種,皆(口父)咀,漬酒中。用綿絮一斤,細白布四丈,並內酒中。置酒馬矢熅中,蓋封涂,勿使洩。五日五夜,出布綿絮,曝干之,干復漬,以盡其汁。每漬必晬其日,乃出干。干,並用滓與綿絮,復布為復巾,長六七尺,為六七巾。則用之生桑炭炙巾,以熨寒痹所刺之處,令熱入至於病所,寒復炙巾以熨之,三十遍而止。汗出以巾拭身,亦三十遍而止。起步內中,無見風。每刺必熨,如此病已矣,此所謂內熱也。




\section{官針第七}

凡刺之要,官針最妙。九針之宜,各有所為,長短大小,各有所施也,不得其用,病弗能移。疾淺針深,內傷良肉,皮膚為癰;病深針淺,病氣不瀉,支為大膿。病小針大,氣瀉太甚,疾必為害;病大針小,氣不洩瀉,亦復為敗。失針之宜,大者瀉,小者不移,已言其過,請言其所施。

病在皮膚無常處者,取以鑱針於病所,膚白勿取。病在分肉間,取以員針於病所。病在經絡痼痹者,取以鋒針。病在脈,氣少當補之者,取以針於井滎分輸。病為大膿者,取以鈹針。病痹氣暴發者,取以員利針。病痹氣痛而不去者,取以毫針。病在中者,取以長針。病水腫不能通關節者,取以大針。病在五藏固居者,取以鋒針,瀉於井滎分輸,取以四時。

凡刺有九,以應九變。一曰輸刺;輸刺者,刺諸經滎輸藏腧也。二曰遠道刺;遠道刺者,病在上,取之下,刺府腧也。三曰經刺;經刺者,刺大經之結絡經分也。四日絡刺;絡刺者,刺小絡之血脈也。五日分刺;分刺者,刺分肉之間也。六曰大瀉刺;大瀉刺者,刺大膿以鈹針也。七曰毛刺;毛刺者,刺浮痹皮膚也。八曰巨刺;巨刺者,左取右,右取左。九曰焠刺;焠刺者,刺燔針則取痹也。

凡刺有十二節,以應十二經。一曰偶刺;偶刺者,以手直心若背,直痛所,一刺前,一刺後,以治心痹,刺此者傍針之也。二曰報刺;報刺者,刺痛無常處也,上下行者,直內無拔針,以左手隨病所按之,乃出針復刺之也。三曰恢刺;恢刺者,直刺傍之,舉之前後,恢筋急,以治筋痹也。四曰齊刺;齊刺者,直入一,傍入二,以治寒氣小深者。或曰三刺;三刺者,治痹氣小深者也。五曰揚刺;揚刺者,正內一,傍內四,而浮之,以治寒氣之博大者也。六曰直針刺;直針刺者,引皮乃刺之,以治寒氣之淺者也。七曰輸刺;輸刺者,直入直出,稀髮針而深之,以治氣盛而熱者也。八曰短刺;短刺者,刺骨痹,稍搖而深之,致針骨所,以上下摩骨也。九曰浮刺;浮刺者,傍入而浮之,以治肌急而寒者也。十日陰刺;陰刺者,左右率刺之,以治寒厥,中寒厥,足踝後少陰也。十一曰傍針刺;傍針刺者,直刺傍刺各一,以治留痹久居者也。十二曰贊刺;贊刺者,直入直出,數髮針而淺之出血,是謂治癰腫也。

脈之所居深不見者刺之,微內針而久留之,以致其空脈氣也。脈淺者勿刺,按絕其脈乃刺之,無令精出,獨出其邪氣耳。所謂三刺則谷氣出者,先淺刺絕皮,以出陽邪;再刺則陰邪出者,少益深,絕皮致肌肉,未入分肉間也;已入分肉之間,則谷氣出。故刺法曰始刺淺之,以遂邪氣而來血氣;後刺深之,以致陰氣之邪;最後刺極深之,以下谷氣。此之謂也。故用針者,不知年之所加,氣之盛衰,虛實之所起,不可以為工也。

凡刺有五,以應五藏。一曰半刺;半刺者,淺內而疾髮針,無針傷肉,如拔毛狀,以取皮氣,此肺之應也。二曰豹文刺;豹文刺者,左右前後針之,中脈為故,以取經絡之血者,此心之應也。三曰關刺;關刺者,直刺左右,盡筋上,以取筋痹,慎無出血,此肝之應也,或曰淵刺,一曰豈刺。四曰合谷刺;合谷刺者,左右雞足,針於分肉之間,以取肌痹,此脾之應也。五曰輸刺;輸刺者,直入直出,深內之至骨,以取骨痺,此腎之應也。



\section{本神第八}

黃帝問於岐伯曰:凡刺之法,先必本於神。血、脈、營、氣、精神,此五藏之所藏也,至於淫泆離藏則精失、魂魄飛揚、志意恍亂、智慮去身者,何因而然乎?天之罪與?人之過乎?何謂德、氣、生、精、神、魂、魄、心、意、志、思、智、慮?請問其故。

岐伯答曰:天之在我者德也,地之在我者氣也,德流氣薄而生者也。故生之來謂之精,兩精相搏謂之神,隨神往來者謂之魂,並精而出入者謂之魄,所以任物者謂之心,心有所憶謂之意,意之所存謂之志,因志而存變謂之思,因思而遠慕謂之慮,因慮而處物謂之智。

故智者之養生也,必順四時而適寒暑,和喜怒而安居處,節陰陽而調剛柔,如是則僻邪不至,長生久視。是故怵惕思慮者則傷神,神傷則恐懼流淫而不止。因哀悲動中者,竭絕而失生。喜樂者,神憚散而不藏。愁憂者,氣閉塞而不行。盛怒者,迷惑而不治。恐懼者,神蕩憚而不收。

心怵惕思慮則傷神,神傷則恐懼自失,破(月囷)脫肉,毛悴色夭,死於冬。脾憂愁而不解則傷意,意傷則悗亂,四支不舉,毛悴色夭,死於春。肝悲哀動中則傷魂,魂傷則狂忘不精,不精則不正當人,陰縮而攣筋,兩脅骨不舉,毛悴色夭,死於秋。肺喜樂無極則傷魄,魄傷則狂,狂者意不存人,皮革焦,毛悴色夭,死於夏。腎盛怒而不止則傷志,志傷則喜忘其前言,腰脊不可以俯仰屈伸,毛卒色夭,死於季夏;恐懼而不解則傷精,精傷則骨痠痿厥,精時自下。是故五藏,主藏精者也,不可傷,傷則失守而陰虛,陰虛則無氣,無氣則死矣。是故用針者,察觀病人之態,以知精神魂魄之存亡得失之意,五者以傷,針不可以治之也。

肝藏血,血舍魂,肝氣虛則恐,實則怒。脾藏營,營舍意,脾氣虛則四支不用,五藏不安,實則腹脹經溲不利。心藏脈,脈舍神,心氣虛則悲,實則笑不休。肺藏氣,氣舍魄,肺氣虛則鼻塞不利少氣,實則喘喝胸盈仰息。腎藏精,精舍志,腎氣虛則厥,實則脹,五藏不安。必審五藏之病形,以知其氣之虛實,謹而調之也。




\section{終始第九}

凡刺之道,畢於終始,明知終始,五藏為紀,陰陽定矣。陰者主藏,陽者主府,陽受氣於四末,陰受氣於五藏。故瀉者迎之,補者隨之,知迎知隨,氣可令和。和氣之方,必通陰陽,五藏為陰,六府為陽,傳之後世,以血為盟,敬之者昌,慢之者亡,無道行私,必得夭殃。謹奉天道,請言終始,終始者,經脈為紀,持其脈口人迎,以知陰陽有餘不足,平與不平,天道畢矣。所謂平人者不病,不病者,脈口人迎應四時也,上下相應而俱往來也,六經之脈不結動也,本末寒溫之相守司也,形肉血氣必相稱也,是謂平人。少氣者,脈口人迎俱少而不稱尺寸也。如是者,則陰陽俱不足,補陽則陰竭,瀉陰則陽脫。如是者,可將以甘藥,不可飲以至劑。如此者弗灸,不已者因而瀉之,則五藏氣壞矣。

人迎一盛,病在足少陽,一盛而躁,病在手少陽。人迎二盛,病在足太陽,二盛而躁,病在手太陽。人迎三盛,病在足陽明,三盛而躁,病在手陽明。人迎四盛,且大且數,名曰溢陽,溢陽為外格。脈口一盛,病在足厥陰,厥陰一盛而躁,在手心主。脈口二盛,病在足少陰,二盛而躁,在手少陰。脈口三盛,病在足太陰,三盛而躁,在手太陰。脈口四盛,且大且數者,名曰溢陰,溢陰為內關,內關不通死不治。人迎與太陰脈口俱盛四倍以上,命曰關格,關格者與之短期。

人迎一盛,瀉足少陽而補足厥陰,二瀉一補,日一取之,必切而驗之,疏取之上,氣和乃止。人迎二盛,瀉足太陽,補足少陰,二瀉一補,二日一取之,必切而驗之,疏取之上,氣和乃止。人迎三盛,瀉足陽明而補足太陰,二瀉一補,日二取之,必切而驗之,疏取之上,氣和乃止。脈口一盛,瀉足厥陰而補足少陽,二補一瀉,日一取之,必切而驗之,疏而取之上,氣和乃止。脈口二盛,瀉足少陰而補足太陽,二補一瀉,二日一取之,必切而驗之,疏取之上,氣和乃止。脈口三盛,瀉足太陰而補足陽明,二補一瀉,日二取之,必切而驗之,疏而取之上,氣和乃止。所以日二取之者,陽明主胃,大富於谷氣,故可日二取之也。人迎與脈口俱盛三倍以上,命曰陰陽俱溢,如是者不開,則血脈閉塞,氣無所行,流淫於中,五藏內傷。如此者,因而灸之,則變易而為他病矣。

凡刺之道,氣調而止,補陰瀉陽,音氣益彰,耳目聰明,反此者血氣不行。所謂氣至而有效者,瀉則益虛,虛者脈大如其故而不堅也,堅如其故者,適雖言故,病未去也。補則益實,實者脈大如其故而益堅也,夫如其故而不堅者,適雖言快,病未去也。故補則實,瀉則虛,痛雖不隨針,病必衰去。必先通十二經脈之所生病,而後可得傳於終始矣。故陰陽不相移,虛實不相頃,取之其經。

凡刺之屬,三刺至谷氣,邪僻妄合,陰陽易居,逆順相反,沉浮異處,四時不得,稽留淫泆,須針而去。故一刺則陽邪出,再刺則陰邪出,三刺則谷氣至,谷氣至而止。所謂谷氣至者,已補而實,已瀉而虛,故以知谷氣至也。邪氣獨去者,陰與陽未能調,而病知愈也。故曰補則實,瀉則虛,痛雖不隨針,病必衰去矣。

陰盛而陽虛,先補其陽,後瀉其陰而和之。陰虛而陽盛,先補其陰,後瀉其陽而和之。三脈動於足大指之間,必審其實虛。虛而瀉之,是謂重虛,重虛病益甚。凡刺此者,以指按之,脈動而實且疾者疾瀉之,虛而徐者則補之,反此者病益甚。其動也,陽明在上,厥陰在中,少陰在下。膺腧中膺,背腧中背。肩膊虛者,取之上。重舌,刺舌柱以鈹針也。手屈而不伸者,其病在筋,伸而不屈者,其病在骨,在骨守骨,在筋守筋。補須一方實,深取之,稀按其痏,以極出其邪氣;一方虛,淺刺之,以養其脈,疾按其痏,無使邪氣得入。邪氣來也緊而疾,谷氣來也徐而和。脈實者,深刺之,以洩其氣;脈虛者,淺刺之,使精氣無得出,以養其脈,獨出其邪氣。

刺諸痛者,其脈皆實。故曰:從腰以上者,手太陰陽明皆主之;從腰以下者,足太陰陽明皆主之。病在上者下取之,病在下者高取之,病在頭者取之足,病在足者取之膕。病生於頭者頭重,生於手者臂重,生於足者足重,治病者先刺其病所從生者也。

春氣在毛,夏氣在皮膚,秋氣在分肉,冬氣在筋骨,刺此病者,各以其時為齊。故刺肥人者,以秋冬之齊;刺瘦人者,以春夏之齊。

病痛者陰也,痛而以手按之不得者陰也,深刺之。病在上者陽也,病在下者陰也。癢者陽也,淺刺之。病先起陰者,先治其陰而後治其陽;病先起陽者,先治其陽而後治其陰。

刺熱厥者,留針反為寒;刺寒厥者,留針反為熱。刺熱厥者,二陰一陽;刺寒厥者,二陽一陰。所謂二陰者,二刺陰也;一陽者,一刺陽也。久病者邪氣入深,刺此病者,深內而久留之,間日而復刺之,必先調其左右,去其血脈,刺道畢矣。

凡刺之法,必察其形氣,形肉未脫,少氣而脈又躁,躁厥者,必為繆刺之,散氣可收,聚氣可布。深居靜處,佔神往來,閉戶塞牖,魂魄不散,專意一神,精氣之分,毋聞人聲,以收其精,必一其神,令志在針,淺而留之,微而浮之,以移其神,氣至乃休。男內女外,堅拒勿出,謹守勿內,是謂得氣。
凡刺之禁:新內勿刺,新刺勿內。已醉勿刺,已刺勿醉。新怒勿刺,已刺勿怒。新勞勿刺,已刺勿勞。已飽勿刺,已刺勿飽。已飢勿刺,已刺勿飢。已渴勿刺,已刺勿渴。大驚大恐,必定其氣,乃刺之。乘車來者,臥而休之,如食頃乃刺之。出行來者,坐而休之,如行十里頃乃刺之。凡此十二禁者,其脈亂氣散,逆其營衛,經氣不次,因而刺之,則陽病入於陰,陰病出為陽,則邪氣復生,粗工勿察,是謂伐身,形體淫泆,乃消腦髓,津液不化,脫其五味,是謂失氣也。

太陽之脈,其終也,戴眼反折瘛瘲,其色白,絕皮乃絕汗,絕汗則終矣。少陽終者,耳聾,百節盡縱,目系絕,目系絕一日半則死矣,其死也,色青白乃死。陽明終者,口目動作,喜驚妄言,色黃,其上下之經盛而不行則終矣。少陰終者,面黑齒長而垢,腹脹閉塞,上下不通而終矣。厥陰終者,中熱嗌干,喜溺心煩,甚則舌卷卵上縮而終矣。太陰終者,腹脹閉不得息,氣噫善嘔,嘔則逆,逆則面赤,不逆則上下不通,上下不通則面黑皮毛燋而終矣。

\section{經脈第十}

雷公問於黃帝曰:禁脈之言,凡刺之理,經脈為始,營其所行,制其度量,內次五藏,外別六府,願盡聞其道。黃帝曰:人始生,先成精,精成而腦髓生,骨為干,脈為營,筋為剛,肉為牆,皮膚堅而毛髮長,谷入於胃,脈道以通,血氣乃行。雷公曰:願卒聞經脈之始生。黃帝曰:經脈者,所以能決死生,處百病,調虛實,不可不通。

肺手太陰之脈,起於中焦,下絡大腸,還循胃口,上膈屬肺,從肺系橫出腋下,下循臑內,行少陰心主之前,下肘中,循臂內上骨下廉,入寸口,上魚,循魚際,出大指之端;其支者,從腕後直出次指內廉,出其端。是動則病肺脹滿膨膨而喘咳,缺盆中痛,甚則交兩手而瞀,此為臂厥。是主肺所生病者,咳,上氣喘渴,煩心胸滿,臑臂內前廉痛厥,掌中熱。氣盛有餘,則肩背痛風寒,汗出中風,小便數而欠。氣虛則肩背痛寒,少氣不足以息,溺色變。為此諸病,盛則瀉之,虛則補之,熱則疾之,寒則留之,陷下則灸之,不盛不虛,以經取之。盛者寸口大三倍於人迎,虛者則寸口反小於人迎也。
大腸手陽明之脈,起於大指次指之端,循指上廉,出合谷兩骨之間,上入兩筋之中,循臂上廉,入肘外廉,上臑外前廉,上肩,出髃骨之前廉,上出於柱骨之會上,下入缺盆絡肺,下膈屬大腸;其支者,從缺盆上頸貫頰,入下齒中,還出挾口,交人中,左之右,右之左,上挾鼻孔。是動則病齒痛頸腫。是主津液所生病者,目黃口乾,鼽衄,喉痹,肩前臑痛,大指次指痛不用。氣有餘則當脈所過者熱腫,虛則寒慄不復。為此諸病,盛則瀉之,虛則補之,熱則疾之,寒則留之,陷下則灸之,不盛不虛,以經取之。盛者人迎大三倍於寸口,虛者人迎反小於寸口也。

胃足陽明之脈,起於鼻之交頞中,旁納太陽之脈,下循鼻外,入上齒中,還出挾口環唇,下交承漿,卻循頤後下廉,出大迎,循頰車,上耳前,過客主人,循髮際,至額顱;其支者,從大迎前下人迎,循喉嚨,入缺盆,下膈屬胃絡脾;其直者,從缺盆下乳內廉,下挾臍,入氣街中;其支者,起於胃口,下循腹裡,下至氣街中而合,以下髀關,抵伏兔,下膝臏中,下循脛外廉,下足跗,入中指內間;其支者,下廉三寸而別,下入中指外間;其支者,別跗上,入大指間,出其端。是動則病灑灑振寒,善呻數欠顏黑,病至則惡人與火,聞木聲則惕然而驚,心欲動,獨閉戶塞牖而處,甚則欲上高而歌,棄衣而走,賁響腹脹,是為骭厥。是主血所生病者,狂瘧溫淫汗出,鼽衄,口喎唇胗,頸腫喉痹,大腹水腫,膝臏腫痛,循膺、乳、氣街、股、伏兔、骭外廉、足跗上皆痛,中指不用。氣盛則身以前皆熱,其有餘於胃,則消谷善飢,溺色黃。氣不足則身以前皆寒慄,胃中寒則脹滿。為此諸病,盛則瀉之,虛則補之,熱則疾之,寒則留之,陷下則灸之,不盛不虛,以經取之。盛者人迎大三倍於寸口,虛者人迎反小於寸口也。
脾足太陰之脈,起於大指之端,循指內側白肉際,過核骨後,上內踝前廉,上踹內,循脛骨後,交出厥陰之前,上膝股內前廉,入腹屬脾絡胃,上膈,挾咽,連舌本,散舌下;其支者,復從胃,別上膈,注心中。是動則病舌本強,食則嘔,胃脘痛,腹脹善噫,得後與氣則快然如衰,身體皆重。是主脾所生病者,舌本痛,體不能動搖,食不下,煩心,心下急痛,溏、瘕、洩,水閉、黃疸,不能臥,強立股膝內腫厥,足大指不用。為此諸病,盛則瀉之,虛則補之,熱則疾之,寒則留之,陷下則灸之,不盛不虛,以經取之。盛者,寸口大三倍於人迎,虛者,寸口反小於人迎也。
心手少陰之脈,起於心中,出屬心繫,下膈絡小腸;其支者,從心繫上挾咽,系目系;其直者,復從心繫卻上肺,下出腋下,下循臑內後廉,行太陰心主之後,下肘內,循臂內後廉,抵掌後銳骨之端,入掌內後廉,循小指之內出其端。是動則病嗌干心痛,渴而欲飲,是為臂厥。是主心所生病者,目黃脅痛,臑臂內後廉痛厥,掌中熱痛。為此諸病,盛則瀉之,虛則補之,熱則疾之,寒則留之,陷下則灸之,不盛不虛,以經取之。盛者寸口大再倍於人迎,虛者寸口反小於人迎也。

小腸手太陽之脈,起於小指之端,循手外側上腕,出踝中,直上循臂骨下廉,出肘內側兩筋之間,上循臑外後廉,出肩解,繞肩胛,交肩上,入缺盆絡心,循嚥下膈,抵胃屬小腸;其支者,從缺盆循頸上頰,至目銳眥,卻入耳中;其支者,別頰上(出頁)抵鼻,至目內眥,斜絡於顴。是動則病嗌痛頷腫,不可以顧,肩似拔,臑似折。是主液所生病者,耳聾目黃頰腫,頸頷肩臑肘臂外後廉痛。為此諸病,盛則瀉之,虛則補之,熱則疾之,寒則留之,陷下則灸之,不盛不虛,以經取之。盛者人迎大再倍於寸口,虛者人迎反小於寸口也。

膀胱足太陽之脈,起於目內眥,上額交巔;其支者,從巔至耳上角;其直者,從巔入絡腦,還出別下項,循肩髆內,挾脊抵腰中,入循膂,絡腎屬膀胱;其支者,從腰中下挾脊貫臀,入膕中;其支者,從髆內左右,別下貫胛,挾脊內,過髀樞,循髀外從後廉下合膕中,以下貫踹內,出外踝之後,循京骨,至小指外側。是動則病沖頭痛,目似脫,項如拔,脊痛,腰似折,髀不可以曲,膕如結,踹如裂,是為踝厥。是主筋所生病者,痔瘧狂顛疾,頭囟項痛,目黃淚出鼽衄,項背腰尻膕踹腳皆痛,小指不用。為此諸病,盛則瀉之,虛則補之,熱則疾之,寒則留之,陷下則灸之,不盛不虛,以經取之。盛者人迎大再倍於寸口,虛者人迎反小於寸口也。

腎足少陰之脈,起於小指之下,邪走足心,出於然谷之下,循內踝之後,別入跟中,以上踹內,出膕內廉,上股內後廉,貫脊屬腎絡膀胱;其直者,從腎上貫肝膈,入肺中,循喉嚨,挾舌本;其支者,從肺出絡心,注胸中。是動則病飢不欲食,面如漆柴,咳唾則有血,喝喝而喘,坐而欲起,目(目巟)(目巟)如無所見,心如懸若飢狀,氣不足則善恐,心惕惕如人將捕之,是為骨厥。是主腎所生病者,口熱舌干,咽腫上氣,嗌干及痛,煩心心痛,黃疸腸澼,脊股內後廉痛,痿厥嗜臥,足下熱而痛。為此諸病,盛則瀉之,虛則補之,熱則疾之,寒則留之,陷下則灸之,不盛不虛,以經取之。灸則強食生肉,緩帶披髮,大杖重履而步。盛者寸口大再倍於人迎,虛者寸口反小於人迎也。
心主手厥陰心包絡之脈,起於胸中,出屬心包絡,下膈,歷絡三膲;其支者,循胸出脅,下腋三寸,上抵腋,下循臑內,行太陰少陰之間,入肘中,下臂行兩筋之間,入掌中,循中指出其端;其支者,別掌中,循小指次指出其端。是動則病手心熱,臂肘攣急,腋腫,甚則胸脅支滿,心中憺憺大動,面赤目黃,喜笑不休。是主脈所生病者,煩心心痛,掌中熱。為此諸病,盛則瀉之,虛則補之,熱則疾之,寒則留之,陷下則灸之,不盛不虛,以經取之。盛者寸口大一倍於人迎,虛者寸口反小於人迎也。

三焦手少陽之脈,起於小指次指之端,上出兩指之間,循手錶腕,出臂外兩骨之間,上貫肘,循臑外上肩,而交出足少陽之後,入缺盆,布羶中,散落心包,下膈,循屬三焦;其支者,從羶中上出缺盆,上項,系耳後,直上出耳上角,以屈下頰至(出頁);其支者,從耳後入耳中,出走耳前,過客主人前,交頰,至目銳眥。是動則病耳聾渾渾焞焞,嗌腫喉痹。是主氣所生病者,汗出,目銳眥痛,頰痛,耳後肩臑肘臂外皆痛,小指次指不用。為此諸病,盛則瀉之,虛則補之,熱則疾之,寒則留之,陷下則灸之,不盛不虛,以經取之。盛者人迎大一倍於寸口,虛者人迎反小於寸口也。

膽足少陽之脈,起於目銳眥,上抵頭角,下耳後,循頸行手少陽之前,至肩上,卻交出手少陽之後,入缺盆;其支者,從耳後入耳中,出走耳前,至目銳眥後;其支者,別銳眥,下大迎,合於手少陽,抵於(出頁),下加頰車,下頸合缺盆以下胸中,貫膈絡肝屬膽,循脅裡,出氣街,繞毛際,橫入髀厭中;其直者,從缺盆下腋,循胸過季脅,下合髀厭中,以下循髀陽,出膝外廉,下外輔骨之前,直下抵絕骨之端,下出外踝之前,循足跗上,入小指次指之間;其支者,別跗上,入大指之間,循大指岐骨內出其端,還貫爪甲,出三毛。是動則病口苦,善太息,心脅痛不能轉側,甚則面微有塵,體無膏澤,足外反熱,是為陽厥。是主骨所生病者,頭痛頷痛,目銳眥痛,缺盆中腫痛,腋下腫,馬刀俠癭,汗出振寒,瘧,胸脅肋髀膝外至脛絕骨外髁前及諸節皆痛,小指次指不用。為此諸病,盛則瀉之,虛則補之,熱則疾之,寒則留之,陷下則灸之,不盛不虛,以經取之。盛者人迎大一倍於寸口,虛者人迎反小於寸口也。

肝足厥陰之脈,起於大指叢毛之際,上循足跗上廉,去內踝一寸,上踝八寸,交出太陰之後,上膕內廉,循股陰入毛中,過陰器,抵小腹,挾胃屬肝絡膽,上貫膈,布脅肋,循喉嚨之後,上入頏顙,連目系,上出額,與督脈會於巔;其支者,從目系下頰裡,環唇內;其支者,復從肝別貫膈,上注肺。是動則病腰痛不可以俯仰,丈夫(疒貴)疝,婦人少腹腫,甚則嗌干,面塵脫色。是主肝所生病者,胸滿嘔逆飧洩,狐疝遺溺閉癃。為此諸病,盛則瀉之,虛則補之,熱則疾之,寒則留之,陷下則灸之,不盛不虛,以經取之。盛者寸口大一倍於人迎,虛者寸口反小於人迎也。

手太陰氣絕則皮毛焦,太陰者行氣溫於皮毛者也,故氣不榮則皮毛焦,皮毛焦則津液去皮節,津液去皮節者則爪枯毛折,毛折者則毛先死,丙篤丁死,火勝金也。手少陰氣絕則脈不通,脈不通則血不流,血不流則髦色不澤,故其面黑如漆柴者,血先死,壬篤癸死,水勝火也。足太陰氣絕者則脈不榮肌肉,唇舌者肌肉之本也,脈不榮則肌肉軟,肌肉軟則舌萎人中滿,人中滿則唇反,唇反者肉先死,甲篤乙死,木勝土也。足少陰氣絕則骨枯,少陰者冬脈也,伏行而濡骨髓者也,故骨不濡則肉不能著也,骨肉不相親則肉軟卻,肉軟卻故齒長而垢發無澤,發無澤者骨先死,戊篤己死,土勝水也。足厥陰氣絕則筋絕,厥陰者肝脈也,肝者筋之合也,筋者聚於陰氣,而脈絡於舌本也,故脈弗榮則筋急,筋急則引舌與卵,故唇青舌卷卵縮則筋先死,庚篤辛死,金勝木也。五陰氣俱絕則目系轉,轉則目運,目運者為志先死,志先死則遠一日半死矣。六陽氣絕,則陰與陽相離,離則腠理髮洩,絕汗乃出,故旦佔夕死,夕佔旦死。

經脈十二者,伏行分肉之間,深而不見;其常見者,足太陰過於外踝之上,無所隱故也。諸脈之浮而常見者,皆絡脈也。六經絡手陽明少陽之大絡,起於五指間,上合肘中。飲酒者,衛氣先行皮膚,先充絡脈,絡脈先盛,故衛氣已平,營氣乃滿,而經脈大盛。脈之卒然動者,皆邪氣居之,留於本末;不動則熱,不堅則陷且空,不與眾同,是以知其何脈之動也。雷公曰:何以知經脈之與絡脈異也?黃帝曰:經脈者常不可見也,其虛實也以氣口知之,脈之見者皆絡脈也。雷公曰:細子無以明其然也。黃帝曰:諸絡脈皆不能經大節之間,必行絕道而出,入復合於皮中,其會皆見於外。故諸刺絡脈者,必刺其結上,甚血者雖無結,急取之以瀉其邪而出其血,留之發為痹也。凡診絡脈,脈色青則寒且痛,赤則有熱。胃中寒,手魚之絡多青矣;胃中有熱,魚際絡赤;其暴黑者,留久痹也;其有赤有黑有青者,寒熱氣也;其青短者,少氣也。凡刺寒熱者皆多血絡,必間日而一取之,血盡乃止,乃調其虛實;其小而短者少氣,甚者瀉之則悶,悶甚則僕不得言,悶則急坐之也。

手太陰之別,名曰列缺,起於腕上分間,並太陰之經直入掌中,散入於魚際。其病實則手銳掌熱,虛則欠(去欠),小便遺數,取之去腕半寸,別走陽明也。手少陰之別,名曰通裡,去腕一寸半,別而上行,循經入於心中,系舌本,屬目系。其實則支膈,虛則不能言,取之掌後一寸,別走太陽也。手心主之別,名曰內關,去腕二寸,出於兩筋之間,循經以上,繫於心包絡。心繫實則心痛,虛則為頭強,取之兩筋間也。手太陽之別,名曰支正,上腕五寸,內注少陰;其別者,上走肘,絡肩髃。實則節弛肘廢,虛則生肬,小者如指痂疥,取之所別也。手陽明之別,名曰偏歷,去腕三寸,別入太陰;其別者,上循臂,乘肩髃,上曲頰偏齒;其別者,入耳合於宗脈。實則齲聾,虛則齒寒痹隔,取之所別也。手少陽之別,名曰外關,去腕二寸,外遶臂,注胸中,合心主。病實則肘攣,虛則不收,取之所別也。足太陽之別,名曰飛揚,去踝七寸,別走少陰。實則鼽窒頭背痛,虛則鼽衄,取之所別也。足少陽之別,名曰光明,去踝五寸,別走厥陰,下絡足跗。實則厥,虛則痿躄,坐不能起,取之所別也。足陽明之別,名曰豐隆,去踝八寸,別走太陰;其別者,循脛骨外廉,上絡頭項,合諸經之氣,下絡喉嗌。其病氣逆則喉痹瘁瘖,實則狂巔,虛則足不收脛枯,取之所別也。足太陰之別,名曰公孫,去本節之後一寸,別走陽明;其別者,入絡腸胃。厥氣上逆則霍亂,實則腸中切痛,虛則鼓脹,取之所別也。足少陰之別,名曰大鍾,當踝後繞跟,別走太陽;其別者,並經上走於心包,下外貫腰脊。其病氣逆則煩悶,實則閉癃,虛則腰痛,取之所別者也。足厥陰之別,名曰蠡溝,去內踝五寸,別走少陽;其別者,徑脛上睾,結於莖。其病氣逆則睾腫卒疝,實則挺長,虛則暴癢,取之所別也。任脈之別,名曰尾翳,下鳩尾,散於腹。實則腹皮痛,虛則癢搔,取之所別也。督脈之別,名曰長強,挾膂上項,散頭上,下當肩胛左右,別走太陽,入貫膂。實則脊強,虛則頭重,高搖之,挾脊之有過者,取之所別也。脾之大絡,名曰大包,出淵腋下三寸,布胸脅。實則身盡痛,虛則百節盡皆縱,此脈若羅絡之血者,皆取之脾之大絡脈也。凡此十五絡者,實則必見,虛則必下,視之不見,求之上下,人經不同,絡脈異所別也。

\section{經別第十一}

黃帝問於岐伯曰:余聞人之合於天道也,內有五藏,以應五節五色五時五味五位也;外有六府,以應六律,六律建陰陽諸經而合之十二月、十二辰、十二節、十二經水、十二時、十二經脈者,此五藏六府之所以應天道。夫十二經脈者,人之所以生,病之所以成,人之所以治,病之所以起,學之所始,工之所止也,粗之所易,上之所難也。請問其離合出入奈何?岐伯稽首再拜曰:明乎哉問也!此粗之所過,上之所息也,請卒言之。

足太陽之正,別入於膕中,其一道下尻五寸,別入於肛,屬於膀胱,散之腎,循膂當心入散;直者,從膂上出於項,復屬於太陽,此為一經也。足少陰之正,至膕中,別走太陽而合,上至腎,當十四顀,出屬帶脈;直者,系舌本,復出於項,合於太陽,此為一合。成以諸陰之別,皆為正也。

足少陽之正,繞髀入毛際,合於厥陰;別者,入季脅之間,循胸裡屬膽,散之上肝貫心,以上挾咽,出頤頷中,散於面,系目系,合少陽於外眥也。足厥陰之正,別跗上,上至毛際,合於少陽,與別俱行,此為二合也。

足陽明之正,上至髀,入於腹裡,屬胃,散之脾,上通於心,上循咽出於口,上頞(出頁),還系目系,合於陽明也。足太陰之正,上至髀,合於陽明,與別俱行,上結於咽,貫舌中,此為三合也。

手太陽之正,指地,別於肩解,入腋走心,系小腸也。手少陰之正,別入於淵腋兩筋之間,屬於心,上走喉嚨,出於面,合目內眥,此為四合也。

手少陽之正,指天,別於巔,入缺盆,下走三焦,散於胸中也。手心主之正,別下淵腋三寸,入胸中,別屬三焦,出循喉嚨,出耳後,合少陽完骨之下,此為五合也。

手陽明之正,從手循膺乳,別於肩髃,入柱骨,下走大腸,屬於肺,上循喉嚨,出缺盆,合於陽明也。手太陰之正,別入淵腋少陰之前,入走肺,散之太陽,上缺盆,循喉嚨,復合陽明,此六合也。




\section{經水第十二}

黃帝問於岐伯曰:經脈十二者,外合於十二經水,而內屬於五藏六府。夫十二經水者,其有大小、深淺、廣狹、遠近各不同,五藏六府之高下、小大、受谷之多少亦不等,相應奈何?夫經水者,受水而行之;五藏者,合神氣魂魄而藏之;六府者,受谷而行之,受氣而揚之;經脈者,受血而營之。合而以治奈何?刺之深淺,灸之壯數,可得聞乎?
岐伯答曰:善哉問也!天至高,不可度,地至廣,不可量,此之謂也。且夫人生於天地之間,六合之內,此天之高、地之廣也,非人力之所度量而至也。若夫八尺之士,皮肉在此,外可度量切循而得之,其死可解剖而視之,其藏之堅脆,府之大小,谷之多少,脈之長短,血之清濁,氣之多少,十二經之多血少氣,與其少血多氣,與其皆多血氣,與其皆少血氣,皆有大數。其治以針艾,各調其經氣,固其常有合乎?

黃帝曰:余聞之,快於耳,不解於心,願卒聞之。岐伯答曰:此人之所以參天地而應陰陽也,不可不察。足太陽外合於清水,內屬於膀胱,而通水道焉。足少陽外合於渭水,內屬於膽。足陽明外合於海水,內屬於胃。足太陰外合於湖水,內屬於脾。足少陰外合於汝水,內屬於腎。足厥陰外合於澠水,內屬於肝。手太陽外合於淮水,內屬於小腸,而水道出焉。手少陽外合於漯水,內屬於三焦。手陽明外合於江水,內屬於大腸。手太陰外合於河水,內屬於肺。手少陰外合於濟水,內屬於心。手心主外合於漳水,內屬於心包。凡此五藏六府十二經水者,外有源泉而內有所稟,此皆內外相貫,如環無端,人經亦然。故天為陽,地為陰,腰以上為天,腰以下為地。故海以北者為陰,湖以北者為陰中之陰,漳以南者為陽,河以北至漳者為陽中之陰,漯以南至江者為陽中之太陽,此一隅之陰陽也,所以人與天地相參也。
黃帝曰:夫經水之應經脈也,其遠近淺深,水血之多少各不同,合而以刺之奈何?岐伯答曰:足陽明,五藏六府之海也,其脈大血多,氣盛熱壯,刺此者不深弗散,不留不瀉也。足陽明刺深六分,留十呼,足太陽深五分,留七呼。足少陽深四分,留五呼。足太陰深三分,留四呼。足少陰深二分,留三呼。足厥陰深一分,留二呼。手之陰陽,其受氣之道近,其氣之來疾,其刺深者皆無過二分,其留皆無過一呼。其少長大小肥瘦,以心撩之,命曰法天之常。灸之亦然。灸而過此者得惡火,則骨枯脈澀;刺而過此者,則脫氣。
黃帝曰:夫經脈之大小,血之多少,膚之厚薄,肉之堅脆,及膕之大小,其可為量度乎?岐伯答曰:其可為量度者,取其中度也,不甚脫肉而血氣不衰也。若失度之人,痟瘦而形肉脫者,惡可以量度刺乎。審切循捫按,視其寒溫盛衰而調之,是謂因適而為之真也。

\section{經筋第十三}

足太陽之筋,起於足小指,上結於踝,邪上結於膝,其下循足外踝,結於踵,上循跟,結於膕;其別者,結於踹外,上膕中內廉,與膕中並上結於臀,上挾脊上項;其支者,別入結於舌本;其直者,結於枕骨,上頭下顏,結於鼻;其支者,為目上網,下結於頄;其支者,從腋後外廉,結於肩髃;其支者,入腋下,上出缺盆,上結於完骨;其支者,出缺盆,邪上出於頄。其病小指支,跟腫痛,膕攣,脊反折,項筋急,肩不舉,腋支,缺盆中紐痛,不可左右搖。治在燔針劫刺,以知為數,以痛為輸,名曰仲春痹也。
足少陽之筋,起於小指次指,上結外踝,上循脛外廉,結於膝外廉;其支者,別起外輔骨,上走髀,前者結於伏兔之上,後者結於尻;其直者,上乘(月少)季協,上走腋前廉,繫於膺乳,結於缺盆;直者,上出腋,貫缺盆,出太陽之前,循耳後,上額角,交巔上,下走頷,上結於頄;支者,結於目眥為外維。其病小指次指支轉筋,引膝外轉筋,膝不可屈伸,膕筋急,前引髀,後引尻,即上乘(月少)季脅痛,上引缺盆膺乳頸,維筋急,從左之右,右目不開,上過右角,並蹻脈而行,左絡於右,故傷左角,右足不用,命曰維筋相交。治在燔針劫刺,以知為數,以痛為輸,名曰孟春痹也。
足陽明之筋,起於中三指,結於跗上,邪外上加於輔骨,上結於膝外廉,直上結於髀樞,上循脅,屬脊;其直者,上循骭,結於膝;其支者,結於外輔骨,合少陽;其直者,上循伏兔,上結於髀,聚於陰器,上腹而布,至缺盆而結,上頸,上挾口,合於頄,下結於鼻,上合於太陽,太陽為目上網,陽明為目下網;其支者,從頰結於耳前。其病足中指支,脛轉筋,腳跳堅,伏兔轉筋,髀前腫,(疒貴)疝,腹筋急,引缺盆及頰,卒口僻,急者目不合,熱則筋縱,目不開。頰筋有寒,則急引頰移口;有熱則筋弛縱緩,不勝收故僻。治之以馬膏,膏其急者,以白酒和桂,以涂其緩者,以桑鉤鉤之,即以生桑灰置之坎中,高下以坐等,以膏熨急頰,且飲美酒,噉美炙肉,不飲酒者,自強也,為之三拊而已。治在燔針劫刺,以知為數,以痛為輸,名曰季春痹也。
足太陰之筋,起於大指之端內側,上結於內踝;其直者,絡於膝內輔骨,上循陰股,洛於髀,聚於陰器,上腹,結於齊,循腹裡,結於肋,散於胸中;其內者,著於脊。其病足大指支,內踝痛,轉筋痛,膝內輔骨痛,陰股引髀而痛,陰器紐痛,下引臍兩脅痛,引膺中脊內痛。治在燔針劫刺,以知為數,以痛為輸,命曰孟秋痹也。
足少陰之筋,起於小指之下,並足太陰之筋邪走內踝之下,結於踵,與太陽之筋合而上結於內輔之下,並太陰之筋而上循陰股,結於陰器,循脊內挾膂,上至項,結於枕骨,與足太陽之筋合。其病足下轉筋,及所過而結者皆痛及轉筋。病在此者主癇瘛及痙,在外者不能俯,在內者不能仰,故陽病者腰反折不能俯,陰病者不能仰。治在燔針劫刺,以知為數,以痛為輸,在內者熨引飲藥。此筋折紐,紐發數甚者,死不治,名曰仲秋痹也。
足厥陰之筋,起於大指之上,上結於內踝之前,上循脛,上結內輔之下,上循陰股,結於陰器,絡諸筋。其病足大指支,內踝之前痛,內輔痛,陰股痛轉筋,陰器不用,傷於內則不起,傷於寒則陰縮入,傷於熱則縱挺不收。治在行水清陰氣。其病轉筋者,治在燔針劫刺,以知為數,以痛為輸,命曰季秋痹也。
手太陽之筋,起於小指之上,結於腕,上循臂內廉,結於肘內銳骨之後,彈之應小指之上,入結於腋下;其支者,後走腋後廉,上繞肩胛,循頸出走太陽之前,結於耳後完骨;其支者,入耳中;直者,出耳上,下結於頷,上屬目外眥。其病小指支,肘內銳骨後廉痛,循臂陰入腋下,腋下痛,腋後廉痛,繞肩胛引頸而痛,應耳中鳴痛,引頷目瞑,良久乃得視,頸筋急則為筋瘻頸腫。寒熱在頸者,治在燔針劫刺之,以知為數,以痛為輸,其為腫者,復而銳之。名曰仲夏痹也。
手少陽之筋,起於小指次指之端,結於腕,上循臂,結於肘,上繞臑外廉,上肩走頸,合手太陽;其支者,當曲頰入系舌本;其支者,上曲牙,循耳前,屬目外眥,上乘頷,結於角。其病當所過者即支轉筋,舌卷。治在燔針劫刺,以知為數,以痛為輸,名曰季夏痹也。
手陽明之筋,起於大指次指之端,結於腕,上循臂,上結於肘外,上臑,結於髃;其支者,繞肩胛,挾脊;直者,從肩髃上頸;其支者,上頰,結於(九頁);直者,上出手太陽之前,上左角,絡頭,下右頷。其病當所過者支痛及轉筋,肩不舉頸,不可左右視。治在燔針劫刺,以知為數,以痛為輸,名曰孟夏痹也。
手太陰之筋,起於大指之上,循指上行,結於魚後,行寸口外側,上循臂,結肘中,上臑內廉,入腋下,出缺盆,結肩前髃,上結缺盆,下結胸裡,散貫賁,合賁下,抵季脅。其病當所過者支轉筋痛,甚成息賁,脅急吐血。治在燔針劫刺,以知為數,以痛為輸,名曰仲冬痹也。
手心主之筋,起於中指,與太陰之筋並行,結於肘內廉,上臂陰,結腋下,下散前後挾脅;其支者,入腋,散胸中,結於臂。其病當所過者支轉筋,前及胸痛息賁。治在燔針劫刺,以知為數,以痛為輸,名曰孟冬痹也。
手少陰之筋,起於小指之內,側結於銳骨,上結肘內廉,上入腋,交太陰,挾乳裡,結於胸中,循臂,下繫於臍。其病內急,心承伏梁,下為肘網。其病當所過者支轉筋,筋痛。治在燔針劫刺,以知為數,以痛為輸。其成伏粱唾血膿者,死不治。經筋之病,寒則反折筋急,熱則筋弛縱不收,陰痿不用。陽急則反折,陰急則俯不伸。焠刺者,刺寒急也,熱則筋縱不收,無用燔針。名曰季冬痹也。
足之陽明,手之太陽,筋急則口目為噼,眥急不能卒視,治皆如右方也




\section{骨度第十四}

黃帝問於伯高曰:脈度言經脈之長短,何以立之?伯高曰:先度其骨節之大小廣狹長短,而脈度定矣。黃帝曰:願聞眾人之度,人長七尺五寸者,其骨節之大小長短各幾何?伯高曰:頭之大骨圍二尺六寸,胸圍四尺五寸,腰圍四尺二寸。

發所覆者,顱至項尺二寸,發以下至頤長一尺,君子終折。結喉以下至缺盆中長四寸。缺盆以下至(骨曷)(骨亏)長九寸。過則肺大,不滿則肺小。(骨曷)(骨亏])以下至天樞長八寸,過則胃大,不及則胃小。天樞以下至橫骨長六寸半,過則迴腸廣長,不滿則狹短。橫骨長六寸半,橫骨上廉以下至內輔之上廉長一尺八寸,內輔之上廉以下至下廉長三寸半,內輔下廉下至內踝長一尺三寸,內踝以下至地長三寸,膝膕以下至跗屬長一尺六寸,跗屬以下至地長三寸,故骨圍大則太過,小則不及。

角以下至柱骨長一尺,行腋中不見者長四寸,腋以下至季脅長一尺二寸,季脅以下至髀樞長六寸,髀樞以下至膝中長一尺九寸,膝以下至外踝長一尺六寸,外踝以下至京骨長三寸,京骨以下至地長一寸。

耳後當完骨者廣九寸,耳前當耳門者廣一尺三寸,兩顴之間相去七寸,兩乳之間廣九寸半,兩髀之間廣六寸半。足長一尺二寸,廣四寸半。

肩至肘長一尺七寸,肘至腕長一尺二寸半,腕至中指本節長四寸,本節至其末長四寸半。項發以下至背骨長二寸半,膂骨以下至尾骶二十一節長三尺,上節長一寸四分,分之一奇分在下,故上七節至於膂骨九寸八分分之七。

此眾人骨之度也,所以立經脈之長短也。是故視其經脈之在於身也,其見浮而堅,其見明而大者,多血;細而沉者,多氣也。



\section{五十營第十五}

黃帝曰:余願聞五十營,奈何?岐伯答曰:天週二十八宿,宿三十六分,人氣行一週,千八分。日行二十八宿,人經脈上下、左右、前後二十八脈,週身十六丈二尺,以應二十八宿,漏水下百刻,以分晝夜。
故人一呼,脈再動,氣行三寸,一吸,脈亦再動,氣行三寸,呼吸定息,氣行六寸。十息氣行六尺,日行二分。二百七十息,氣行十六丈二尺,氣行交通於中,一週於身,下水二刻,日行二十五分。五百四十息,氣行再周於身,下水四刻,日行四十分。二千七百息,氣行十週於身,下水二十刻,日行五宿二十分。一萬三千五百息,氣行五十營於身,水下百刻,日行二十八宿,漏水皆盡,脈終矣。
所謂交通者,並行一數也,故五十營備,得盡天地之壽矣,凡行八百一十丈也。




\section{營氣第十六}

黃帝曰:營氣之道,內谷為寶。谷入於胃,乃傳之肺,流溢於中,布散於外,精專者行於經隧,常營無已,終而復始,是謂天地之紀。故氣從太陰出,注手陽明,上行注足陽明,下行至跗上,注大指間,與太陰合,上行抵髀。從脾注心中,循手少陰出腋下臂,注小指,合手太陽,上行乘腋出(出頁)內,注目內眥,上巔下項,合足太陽,循脊下尻,下行注小指之端,循足心注足少陰,上行注腎,從腎注心,外散於胸中。循心主脈出腋下臂,出兩筋之間,入掌中,出中指之端,還注小指次指之端,合手少陽,上行注羶中,散於三焦,從三焦注膽,出脅注足少陽,下行至跗上,復從跗注大指間,合足厥陰,上行至肝,從肝上注肺,上循喉嚨,入頏顙之竅,究於畜門。其支別者,上額循巔下項中,循脊入骶,是督脈也,絡陰器,上過毛中,入臍中,上循腹裡,入缺盆,下注肺中,復出太陰。此營氣之所行也,逆順之常也。




\section{脈度第十七}

黃帝曰:願聞脈度。岐伯答曰:手之六陽,從手至頭,長五尺,五六三丈。手之六陰,從手至胸中,三尺五寸,三六一丈八尺,五六三尺,合二丈一尺。足之六陽,從足上至頭八尺,六八四丈八尺。足之六陰,從足至胸中,六尺五寸,六六三丈六尺,五六三尺,合三丈九尺。蹻脈從足至目,七尺五寸,二七一丈四尺,二五一尺,合一丈五尺。督脈任脈各四尺五寸,二四八尺,二五一尺,合九尺。凡都合一十六丈二尺,此氣之大經隧也。經脈為裡,支而橫者為絡,絡之別者為孫,盛而血者疾誅之,盛者瀉之,虛者飲藥以補之。

五藏常內閱於上七竅也,故肺氣通於鼻,肺和則鼻能知臭香矣;心氣通於舌,心和則舌能知五味矣;肝氣通於目,肝和則目能辨五色矣;脾氣通於口,脾和則口能知五穀矣;腎氣通於耳,腎和則耳能聞五音矣。五藏不和則七竅不通,六府不和則留為癰。故邪在府則陽脈不和,陽脈不和則氣留之,氣留之則陽氣盛矣。陽氣太盛則陰脈不利,陰脈不利則血留之,血留之則陰氣盛矣。陰氣太盛,則陽氣不能榮也,故曰關。陽氣太盛,則陰氣弗能榮也,故曰格。陰陽俱盛,不得相榮,故曰關格。關格者,不得盡期而死也。

黃帝曰:蹻脈安起安止?何氣榮水?岐伯答曰:蹻脈者,少陰之別,起於然骨之後,上內踝之上,直上循陰股入陰,上循胸裡入缺盆,上出人迎之前,入頄,屬目內眥,合於太陽、陽蹻而上行,氣並相還則為濡目,氣不榮則目不合。

黃帝曰:氣獨行五藏,不榮六府,何也?岐伯答曰:氣之不得無行也,如水之流,如日月之行不休,故陰脈榮其藏,陽脈榮其府,如環之無端,莫知其紀,終而復始。其流溢之氣,內溉藏府,外濡腠理。

黃帝曰:蹻脈有陰陽,何脈當其數?岐伯答曰:男子數其陽,女子數其陰,當數者為經,其不當數者為絡也。




\section{營衛生會第十八}

黃帝問於岐伯曰:人焉受氣?陰陽焉會?何氣為營?何氣為衛?營安從生?衛於焉會?老壯不同氣,陰陽異位,願聞其會。岐伯答曰:人受氣於谷,谷入於胃,以傳與肺,五藏六府,皆以受氣,其清者為營,濁者為衛,營在脈中,衛在脈外,營周不休,五十而復大會。陰陽相貫,如環無端。衛氣行於陰二十五度,行於陽二十五度,分為晝夜,故氣至陽而起,至陰而止。故曰:日中而陽隴為重陽,夜半而陰隴為重陰。故太陰主內,太陽主外,各行二十五度,分為晝夜。夜半為陰隴,夜半後而為陰衰,平旦陰盡而陽受氣矣。日中而陽隴,日西而陽衰,日入陽盡而陰受氣矣。夜半而大會,萬民皆臥,命曰合陰,平旦陰盡而陽受氣,如是無已,與天地同紀。

黃帝曰:老人之不夜瞑者,何氣使然?少壯之人不晝瞑者,何氣使然?岐伯答曰:壯者之氣血盛,其肌肉滑,氣道通,營衛之行,不失其常,故晝精而夜瞑。老者之氣血衰,其肌肉枯,氣道澀,五藏之氣相搏,其營氣衰少而衛氣內伐,故晝不精,夜不瞑。

黃帝曰:願聞營衛之所行,皆何道從來?岐伯答曰:營出於中焦,衛出於下焦。黃帝曰:願聞三焦之所出。岐伯答曰:上焦出於胃上口,並咽以上,貫膈而布胸中,走腋,循太陰之分而行,還至陽明,上至舌,下足陽明,常與營俱行於陽二十五度,行於陰亦二十五度,一週也,故五十度而復大會於手太陰矣。黃帝曰:人有熱飲食下胃,其氣未定,汗則出,或出於面,或出於背,或出於身半,其不循衛氣之道而出何也?岐伯曰:此外傷於風,內開腠理,毛蒸理洩,衛氣走之,固不得循其道,此氣慓悍滑疾,見開而出,故不得循其道,故命曰漏洩。
黃帝曰;願聞中焦之所出,岐伯答曰:中焦亦並胃中,出上焦之後,此所受氣者,泌糟粕,蒸津液,化其精微,上注於肺脈,乃化而為血,以奉生身,莫貴於此,故獨得行於經隧,命曰營氣。黃帝曰:夫血之與氣,異名同類,何謂也?岐伯答曰:營衛者精氣也,血者神氣也,故血之與氣,異名同類焉。故奪血者無汗,奪汗者無血,故人生有兩死而無兩生。
黃帝曰:願聞下焦之所出。岐伯答曰:下焦者,別迴腸,注於膀胱而滲入焉。故水谷者,常並居於胃中,成糟粕,而俱下於大腸,而成下焦,滲而俱下,濟泌別汁,循下焦而滲入膀胱焉。黃帝曰:人飲酒,酒亦入胃,谷未熟而小便獨先下何也?岐伯答曰:酒者熟谷之液也,其氣悍以清,故後谷而入,先谷而液出焉。黃帝曰:善。余聞上焦如霧,中焦如漚,下焦如瀆,此之謂也。

\section{四時氣第十九}

黃帝問於岐伯曰:夫四時之氣,各不同形,百病之起,皆有所生,灸刺之道,何者為定?岐伯答曰:四時之氣,各有所在,灸刺之道,得氣穴為定。故春取經血脈分肉之間,甚者深刺之,間者淺刺之。夏取盛經孫絡,取分間絕皮膚。秋取經腧,邪在府,取之合。冬取井滎,必深以留之。

溫瘧汗不出,為五十九痏。風(疒水)膚脹,為五十七痏,取皮膚之血者,盡取之。飧洩,補三陰之上,補陰陵泉,皆久留之,熱行乃止。轉筋於陽治其陽,轉筋於陰治其陰,皆卒刺之。徒(疒水),先取環谷下三寸,以鈹針針之,已刺而筩之,而內之,入而復之,以盡其(疒水),必堅,來緩則煩悗,來急則安靜,間日一刺之,(疒水)盡乃止。飲閉藥,方刺之時徒飲之,方飲無食,方食無飲,無食他食,百三十五日。著痹不去,久寒不已,卒取其三里骨為干。腸中不便,取三里,盛瀉之,虛補之。癘風者,素刺其腫上,已刺,以銳針針其處,按出其惡氣,腫盡乃止,常食方食,無食他食。
腹中常鳴,氣上衝胸,喘不能久立,邪在大腸,刺肓之原、巨虛上廉、三里。小腹控睾、引腰脊,上衝心,邪在小腸者,連睾系,屬於脊,貫肝肺,絡心繫。氣盛則厥逆,上衝腸胃,熏肝,散於肓,結於臍。故取之肓原以散之,刺太陰以予之,取厥陰以下之,取巨虛下廉以去之,按其所過之經以調之。善嘔,嘔有苦,長太息,心中憺憺,恐人將捕之,邪在膽,逆在胃,膽液洩則口苦,胃氣逆則嘔苦,故曰嘔膽。取三里以下胃氣逆,則刺少陽血絡以閉膽逆,卻調其虛實以去其邪。飲食不下,膈塞不通,邪在胃脘,在上脘則刺抑而下之,在下脘則散而去之。小腹痛腫,不得小便,邪在三焦約,取之太陽大絡,視其絡脈與厥陰小絡結而血者,腫上及胃脘,取三里。

覩其色,察其以,知其散復者,視其目色,以知病之存亡也。一其形,聽其動靜者,持氣口人迎以視其脈,堅且盛且滑者病日進,脈軟者病將下。諸經實者病三日已。氣口候陰,人迎候陽也。



\section{五邪第二十}

邪在肺,則病皮膚痛,寒熱,上氣喘,汗出,咳動肩背。取之膺中外腧,背三節五藏之傍,以手疾按之,快然,乃刺之,取之缺盆中以越之。

邪在肝,則兩脅中痛,寒中,惡血在內,行善掣,節時腳腫,取之行間以引脅下,補三里以溫胃中,取血脈以散惡血,取耳間青脈,以去其掣。

邪在脾胃,則病肌肉痛。陽氣有餘,陰氣不足,則熱中善飢;陽氣不足,陰氣有餘,則寒中腸鳴腹痛。陰陽俱有餘,若俱不足,則有寒有熱。皆調於三里。

邪在腎,則病骨痛陰痹。陰痹者,按之而不得,腹脹腰痛,大便難,肩背頸項痛,時眩。取之湧泉、崑崙,視有血者盡取之。

邪在心,則病心痛喜悲,時眩僕,視有餘不足而調之其輸也。


\section{寒熱病第二十一}

皮寒熱者,不可附席,毛髮焦,鼻槁臘,不得汗。取三陽之絡,以補手太陰。肌寒熱者,肌痛,毛髮焦而唇槁臘,不得汗。取三陽於下以去其血者,補足太陰以出其汗。骨寒熱者,病無所安,汗注不休。齒未槁,取其少陰於陰股之絡;齒已槁,死不治。骨厥亦然。骨痹,舉節不用而痛,汗注煩心,取三陰之經,補之。身有所傷血出多,及中風寒,若有所墮墜,四支懈惰不收,名曰體惰。取其小腹臍下三結交。三結交者,陽明、太陰也,臍下三寸關元也。厥痹者,厥氣上及腹。取陰陽之絡,視主病也,瀉陽補陰經也。

頸側之動脈人迎。人迎,足陽明也,在嬰筋之前。嬰筋之後,手陽明也,名曰扶突。次脈,足少陽脈也,名曰天牖。次脈,足太陽也,名曰天柱。腋下動脈,臂太陰也,名曰天府。陽迎頭痛,胸滿不得息,取之人迎。暴瘖氣鞭,取扶突與舌本出血。暴聾氣蒙,耳目不明,取天牖。暴攣癇眩,足不任身,取天柱。暴癉內逆,肝肺相搏,血溢鼻口,取天府。此為天牖五部。

臂陽明有入頄遍齒者,名曰大迎,下齒齲取之,臂惡寒補之,不惡寒瀉之。足太陽有入頄遍齒者,名曰角孫,上齒齲取之,在鼻與頄前。方病之時其脈盛,盛則瀉之,虛則補之。一曰取之出鼻外。足陽明有挾鼻入於面者,名曰懸顱,屬口,對入系目本,視有過者取之,損有餘,益不足,反者益。其足太陽有通項入於腦者,正屬目本,名曰眼系,頭目苦痛,取之在項中兩筋間,入腦乃別。陰蹻、陽蹻,陰陽相交,陽入陰,陰出陽,交於目銳眥,陽氣盛則瞋目,陰氣盛則瞑目。

熱厥取足太陰、少陽,皆留之;寒厥取足陽明、少陰於足,皆留之。舌縱涎下,煩悗,取足少陰。振寒灑灑,鼓頷,不得汗出,腹脹煩悗,取手太陰。刺虛者,刺其去也;刺實者,刺其來也。春取絡脈,夏取分腠,秋取氣口,冬取經輸,凡此四時,各以時為齊。絡脈治皮膚,分腠治肌肉,氣口治筋脈,經輸治骨髓、五藏。身有五部:伏免一;腓二,腓者腨也;背三;五藏之腧四;項五。此五部有癰疽者死。病始手臂者,先取手陽明、太陰而汗出;病始頭首者,先取項太陽而汗出;病始足脛者,先取足陽明而汗出。臂太陰可汗出,足陽明可汗出。故取陰而汗出甚者,止之於陽;取陽而汗出甚者,止之於陰。凡刺之害,中而不去則精洩,不中而去則致氣;精洩則病甚而恇,致氣則生為癰疽也。

\section{癲狂第二十二}

目眥外決於面者,為銳眥;在內近鼻者為內眥;上為外眥,下為內眥。癲疾始生,先不樂,頭重痛,視舉目赤,甚作極已,而煩心,候之於顏,取手太陽、陽明、太陰,血變而止。癲疾始作而引口啼呼喘悸者,候之手陽明、太陽,左強者攻其右,右強者攻其左,血變而止。癲疾始作先反僵,因而脊痛,候之足太陽、陽明、太陰、手太陽,血變而止。
治癲疾者,常與之居,察其所當取之處。病至,視之有過者瀉之,置其血於瓠壺之中,至其發時,血獨動矣,不動,灸窮骨二十壯。窮骨者,骶骨也。
骨癲疾者,顑齒諸腧分肉皆滿,而骨居,汗出煩悗。嘔多沃沫,氣下洩,不治。筋癲疾者,身倦攣急大,刺項大經之大杼脈。嘔多沃沫,氣下洩,不治。脈癲疾者,暴僕,四肢之脈皆脹而縱。脈滿,盡刺之出血;不滿,灸之挾項太陽,灸帶脈於腰相去三寸,諸分肉本輸。嘔多沃沫,氣下洩,不治。癲疾者,疾發如狂者,死不治。
狂始生,先自悲也,喜忘苦怒善恐者,得之憂飢,治之取手太陰、陽明,血變而止,及取足太陰、陽明。狂始發,少臥不飢,自高賢也,自辨智也,自尊貴也,善罵詈,日夜不休,治之取手陽明、太陽、太陰、舌下少陰,視之盛者,皆取之,不盛,釋之也。狂言、驚、善笑、好歌樂、妄行不休者,得之大恐,治之取手陽明、太陽、太陰。狂,目妄見、耳妄聞、善呼者,少氣之所生也,治之取手太陽、太陰、陽明、足太陰、頭、兩顑。狂者多食,善見鬼神,善笑而不發於外者,得之有所大喜,治之取足太陰、太陽、陽明,後取手太陰、太陽、陽明。狂而新發,未應如此者,先取曲泉左右動脈,及盛者見血,有頃已,不已,以法取之,灸骨骶二十壯。

風逆暴四肢腫,身漯漯,唏然時寒,飢則煩,飽則善變,取手太陰表裡,足少陰、陽明之經,肉清取滎,骨清取井、經也。厥逆為病也,足暴清,胸若將裂,腸若將以刀切之,煩而不能食,脈大小皆澀,暖取足少陰,清取足陽明,清則補之,溫則瀉之。厥逆腹脹滿,腸鳴,胸滿不得息,取之下胸二脅咳而動手者,與背腧以手按之立快者是也。內閉不得溲,刺足少陰、太陽與骶上以長針,氣逆則取其太陰、陽明、厥陰,甚取少陰、陽明動者之經也。少氣,身漯漯也,言吸吸也,骨痠體重,懈惰不能動,補足少陰。短氣,息短不屬,動作氣索,補足少陰,去血絡也。




\section{熱病第二十三}

偏枯,身偏不用而痛,言不變,志不亂,病在分腠之間,巨針取之,益其不足,損其有餘,乃可復也。痱之為病也,身無痛者,四肢不收,智亂不甚,其言微知,可治,甚則不能言,不可治也。病先起於陽,後入於陰者,先取其陽,後取其陰,浮而取之。

熱病三日而氣口靜、人迎躁者,取之諸陽,五十九刺,以瀉其熱而出其汗,實其陰以補其不足者。身熱甚,陰陽皆靜者,勿刺也;其可刺者,急取之,不汗出則洩。所謂勿刺者,有死征也。熱病七日八日,脈口動喘而短者,急刺之,汗且自出,淺刺手大指間。熱病七日八日,脈微小,病者溲血,口中干,一日半而死,脈代者,一日死。熱病已得汗出,而脈尚躁,喘且復熱,勿刺膚,喘甚者死。熱病七日八日,脈不躁,躁不散數,後三日中有汗;三日不汗,四日死。未曾汗者,勿腠刺之。

熱病先膚痛,窒鼻充面,取之皮,以第一針,五十九,苛軫鼻,索皮於肺,不得索之火,火者心也。熱病先身澀,倚而熱,煩悗,干唇口嗌,取之皮,以第一針,五十九,膚脹口乾,寒汗出,索脈於心,不得索之水,水者腎也。熱病嗌干多飲,善驚,臥不能起,取之膚肉,以第六針,五十九,目青,索肉於脾,不得索之木,木者,肝也。熱病面青腦痛,手足躁,取之筋間,以第四針於四逆,筋目浸,索筋於肝,不得索之金,金者,肺也,熱病數驚,而狂,取之脈,以第四針,急瀉有餘者,癲疾毛髮去,索血於心,不得索之水,水者,腎也。熱病身重骨痛,耳聾而好瞑,取之骨,以第四針,五十九,刺骨病不食,齒耳青,索骨於腎,不得索之土,土者,脾也。熱病不知所痛,耳聾不能自收,口乾,陽熱甚,陰頗有寒者,熱在髓,死,不可治。熱病頭痛,顳目,脈痛善,厥熱病也,取之以第三針,視有餘不足,寒熱痔熱病,體重,腸中熱,取之以第四針,於其俞及下諸指間,索氣於胃絡,得氣也,熱病挾齊急痛,胸脅滿,取之湧泉與陰陵泉,取以第四針,針嗌裡。熱病而汗且出,及脈順可汗者,取之魚際大淵大都大白,瀉之則熱去,補之則汗出,汗出太甚,取內踝上橫脈以止之。熱病已得汗而脈尚躁盛,此陰脈之極也,死。其得汗而脈靜者,生。熱病者脈尚盛躁而不得汗者,此陽脈之極也,死。脈盛躁得汗靜者,生。熱病不可刺者有九,一曰,汗不出,大顴發赤噦者,死,二曰,洩而腹滿甚者,死。三曰,目不明,熱不已者,死。四曰,老人嬰兒,熱而腹滿者,死。五曰,汗不出,嘔下血者,死。六曰,舌本爛,熱不已者,死。七曰,而,汗不出,出不至足者,死。八曰,髓熱者,死。九曰,熱而痙者,死,腰折,齒噤也。凡此尢九者,不可刺也。所謂五十九刺者,兩手外內側各三,凡十二,五指間各一,凡八,足亦如是。頭入發一寸傍三分各三,凡六。更入發三寸邊五,凡十。耳前後口下者各一,項中一,凡六。巔上一,囟會一,髮際一。廉泉一,風池二,天柱二。第三節氣滿胸中喘息,取足太陰大指之端,去爪甲如韭葉,寒則留之,熱則疾之,氣下乃止。心疝暴痛,取足太陰厥陰,盡刺去其血絡。喉痹舌卷,口中干,煩心,心痛,臂內廉痛,不可及頭,取手小指次指爪甲下,去端如韭葉。目中赤痛,從內始,取之陰。風痙身反折,先取足太陽及中及血絡出血,中有寒,取三里。癃,取之陰及三毛上及血絡出血,男子如蠱,女子如,身體腰脊如解,不欲飲食,先取湧泉見血,視跗上盛者,盡見血也。



\section{厥病第二十四}

厥頭痛,面若腫起而煩心,取之足陽明太陰。厥頭痛,頭脈痛,心悲善泣,視頭動脈反盛者,刺盡去血,後調足厥陰。厥頭痛,貞貞頭痛而重,瀉頭上五行行五,先取手少陰,後取足少陰。厥頭痛,意善忘,按之不得,取頭面左右動脈,後取足太陰。厥頭痛,項先痛,腰脊為應,先取天柱,後取足太陽。厥頭痛,頭痛甚,耳前後脈湧有熱,瀉出其血,後取足少陽。真頭痛,頭痛甚,腦盡痛,手足寒至節,死不治。頭痛不可取於俞者,有所擊墮,惡血在於內,若肉傷,痛未已,可則刺,不可遠取也。頭痛不可刺者,大痹為惡,日作者,可令少愈,不可已,頭半寒痛,先取手少陽陽明,後取足少陽陽明。

厥心痛,與背相控善,如從後觸其心,傴僂者,腎心痛也,先取京骨崑崙。發狂不已,取然谷。厥心痛,腹脹胸滿,心尤痛甚,胃心痛也,取之大都太白。厥心痛,痛如以錐針刺其心,心痛甚者,脾心痛也,取之然谷太溪。厥心痛,色蒼蒼如死狀,終日不得太息,肝心痛也,取之行間太沖。厥心痛,臥若徒居,心痛,間動作,痛益甚,色不變,肺心痛也,取之魚際太淵。真心痛,手足清至節,心痛甚,旦發夕死,夕發旦死。心痛不可刺者,中有盛聚,不可取於俞。腸中有蟲瘕及蛟有,皆不可取以小針。心腸痛,作痛,膿聚,往來上下行,痛有休止,腹熱喜渴涎出者,是蛟有也。以手聚按而堅持之,無令得移,以大針刺之,久持之,蟲不動,乃出針也。腹痛。形中上者。

耳聾無聞,取中耳,耳鳴,取耳前動脈。耳痛不可刺者,耳中有膿,若有干盯,耳無聞也。耳聾取手小指次指爪甲上與肉交者,先取手,後取足。耳鳴取手中指爪甲上,左取右,右取左,先取手,後取足。足髀不可舉,側而取之,在樞谷中,以員利針,大針不可刺。病注下血,取曲泉。風痹淫濼,病不可已者,足如履冰,時如入湯中,股脛淫樂,煩心頭痛,時嘔時,眩已汗出,久則目眩,悲以喜恐,短氣,不樂,不出三年,死矣。



\section{病本第二十五}

先病而後逆者,治其本,先逆而後病者,治其本,先寒而後生病者,治其本。先病而後生寒者,治其本。先熱而後生病者,治其本。先洩而後生他病者,治其本。必且調之,乃治其他病。先病而後中滿者,治其標。先病後洩者,治其本。先中滿而後煩心者,治其本。有客氣有同氣,大小便不利,治其標。大小便利,治其本。病發而有餘,本而標之,先治其本,後治其標。病發而不足,標而本之,先治其標,後治其本。謹詳察間甚,以意調之,間者並行,甚為獨行。先小大便不利而後生他病者,治其本也。



\section{雜病第二十六}

厥挾脊而痛者,至頂,頭沉沉然,目然,腰脊強,取足太陽中血絡。厥胸滿面腫,漯漯,然暴言難,甚則不能言,取足陽明。厥氣走喉而不能言,手足清,大便不利,取足少陰。厥而腹向向然,多寒氣,腹中,便溲難,取足太陰。
嗌干,口中熱如膠,取足少陰,膝中痛,取犢鼻,以員利針,發而間之,針大如,刺膝無疑。喉痹不能言,取足陽明。能言,取手陽明。瘧不渴,間日而作,取足陽明。渴而日作,取手陽明。齒痛不惡清飲,取足陽明。惡清飲,取手陽明。聾而不痛者,取足少陽。聾而痛者。取手陽明。而不止,杯血流,取足太陽。杯血,取手太陽,不已,刺宛骨下,不已,刺中出血。腰痛,痛上寒,取足太陽陽明。痛上熱,取足厥陰。不可以仰,取足少陽。中熱而喘,取足少陰中血絡。喜怒而不欲食,言益小,刺足太陰。怒而多言,刺足少陽。痛,刺手陽明與之盛脈出血。項痛不可仰,刺足太陽。不可以顧,刺手太陽也。
小腹滿大,上走胃,至心,淅淅身時寒熱,小便不利,取足厥陰。腹滿,大便不利,腹大,亦上走胸嗌,喘息喝喝然,取足少陰。腹滿食不化,腹向向然,不能大便,取足太陰。心痛引腰脊,欲嘔,取足少陰。心痛腹脹,嗇嗇然,大便不利,取足太陰。
心痛引背,不得息,刺足少陰,不已,取手少陽。心痛引小腹滿,上下無常處,便溲難,刺足厥陰。心痛,但短氣不足以息,刺手太陰。心痛,當九節刺之,按,已刺按之,立已。不已,上下求之,得之立已。
痛,刺足陽明曲周動脈見血,立已。不已,按人迎於經,立已。氣逆上,刺膺中陷者與下胸動脈,腹痛,刺齊左右動脈,已刺按之,立已。不已,刺氣街,已按刺之,立已。痿厥為四末,乃疾解之,日二,不仁者,十日而知,無休,病已止。噦以草刺鼻,嚏,嚏而已。無息而疾迎引之,立已。大驚之,亦可已。



\section{周痹第二十七}

黃帝問於岐伯曰:周痹之在身也,上下移徒隨脈,其上下左右相應,間不容空,願聞此痛,在血脈之中邪,將在分肉之間乎,何以致是。其痛之移也,間不及下針,其痛之時,不及定治,而痛已止矣,何道使然,願聞其故。

岐伯答曰:此眾痹也。非周痹也。黃帝曰:願聞眾痹。岐伯對曰:此各在其處,更發更止,更居更起,以右應左,以左應右。非能周也,更發更休也。黃帝曰:善。刺之奈何?岐伯對曰:刺此者,痛雖已止,必刺其處,勿令復起。

帝曰:善。願聞周痹何如?岐伯對曰:周痹者,在於血脈之中,隨脈以上,隨脈以下,不能左右,各當其所。黃帝曰:刺之奈何?岐伯對曰:痛從上下者,先刺其下以過之,後刺其上以脫之,痛從下上者,先刺其上以過之,後刺其下以脫之。

黃帝曰:善。此痛安生。何因而有名。岐伯對曰:風寒濕氣,客於外分肉之間,迫切而為沫,沫得寒則聚,聚則排分肉而分裂也,分裂則痛,痛則神歸之,神歸之則熱,熱則痛解,痛解則厥,厥則他痹發,發則如是。

帝曰:善。余已得其意矣。此內不在藏,而外未發於皮,獨居分肉之間,真氣不能周,故命曰周痹。故刺痹者,必先切循其下之六經,視其虛實,及大絡之血結而不通,及虛而脈陷空者而調之,熨而通之其手堅,轉引而行之。黃帝曰:善。余已得其意矣。亦得其事也。九者,經巽之理,十二經脈陰陽之病也。




\section{口問第二十八}

黃帝閒居,辟左右而問於岐伯曰:余已聞九針之經,論陰陽逆順,六經已畢,願得口問。岐伯避席再拜曰:善乎哉問也,此先師之所口傳也。黃帝曰:願聞口傳。岐伯答曰:夫百病之始生也,皆生於風雨寒暑,陰陽喜怒,飲食居處,大驚卒恐,則血氣分離,陰陽破散,經絡厥絕,脈道不通,陰陽相逆,衛氣稽留,經脈虛空,血氣不次,乃失其常,論不在經者,請道其方。

黃帝曰:人之欠者,何氣使然。岐伯答曰:衛氣晝日行於陽,夜半則行於陰,陰者主夜,夜者臥,陽者主上,陰者主下,故陰氣積於下,陽氣未盡,陽引而上,陰引而下,陰陽相引,故數欠,陽氣盡,陰氣盛則目瞑,陰氣盡而陽氣盛,盛則寤矣。瀉足少陰,補足太陽。
黃帝曰:入之噦者,何氣使然。岐伯曰:谷入於胃,胃氣上注於肺,今有故寒氣與新谷氣,俱還入於胃,新故相亂,真邪相攻,氣並相逆,復出於胃,故為噦,補手太陰,瀉足少陰。
黃帝曰:人之唏者,何氣使然。岐伯曰:此陰氣盛而陽氣虛,陰氣疾而陽氣徐,陰氣盛而陽氣絕,故為唏,補足太陽,瀉足少陰。
黃帝曰:人之振寒者,何氣使然。岐伯曰:寒氣客於皮膚,陰氣盛,陽氣虛,故為振寒寒慄,補諸陽。
黃帝曰:人之噫者,何氣使然。岐伯曰:寒氣客於胃,厥逆從下上散,復出於胃,故為噫,補足太陰陽明,一曰補眉本也。
黃帝曰:人之嚏者,何氣使然。岐伯曰:陽氣和利,滿於心,出於鼻,故為嚏,補足太陽滎眉本,一曰眉上也。
黃帝曰:人之者,何氣使然。岐伯曰:胃不實則諸脈虛,諸脈虛則筋脈懈惰,筋脈懈惰則行陰用力,氣不能復,故為,因其所在,補分肉間。
黃帝曰:人之哀而泣涕出者,何氣使然。岐伯曰:心者,五藏六府之主也。目者,宗脈之所聚也,上液之道也。口鼻者,氣之門戶也。故悲哀愁憂則心動,心動則五藏六府皆搖,搖則宗脈感,宗脈感則液道開,液道開,故泣涕出焉液者,所以灌精濡空竅者也。故上液之道開,則泣,泣不止則液竭,液竭則精不灌,精不灌則目無所見矣,故命曰奪精,補天柱經俠頸。
黃帝曰:人之太息者,何氣使然。岐伯曰:憂思則心繫急,心繫急則氣道約,約則不利,故太息以伸出之,補手少陰心主,足少陽留之也。
黃帝曰:人之涎下者,何氣使然。岐伯曰:飲食者,皆入於胃,胃中有熱則蟲動,蟲動則胃緩,胃緩則廉泉開,故涎下,補足少陰。
黃帝曰:人之耳中鳴者,何氣使然。岐伯曰:耳者,宗脈之所聚也,故胃中空則宗脈虛,虛則下溜,脈有所竭者,故耳鳴,補客主人,手大指爪甲上與肉交者也。
黃帝曰:人之自舌者,何氣使然。此厥逆走上,脈氣輩至也,少陰氣至則舌,少陽氣至則頰,陽明氣至則矣,視主病者,則補之。

凡此十二邪者,皆奇邪之走空竅者也,故邪之所在,皆為不足,故上氣不足,腦為之不滿,耳為之苦鳴,頭為之苦頃,目為之眩,中氣不足,溲便為之變,腸為之苦鳴。下氣不足,則乃為痿厥心,補足外踝下留之。黃帝曰:治之奈何?岐伯曰:腎主為欠,取足少陰。肺主為噦,取手太陰。足少陰唏者,陰與陽絕,故補足太陽,瀉足少陰。振寒者,補諸陽。噫者,補足太陰陽明。嚏者,補足太陽眉本。因其所在,補分肉間。泣出補天柱經俠頸,俠頸者,頭中分也。太息補手少陰心主,足少陽留之。涎下補足少陰。耳鳴補客主人,手大指爪甲上與肉交者。自舌,視主病者,則補之。目眩頭頃,補足外踝下留之。痿厥心,刺足大指間上二寸留之。一曰足外踝下留之。

\section{師傳第二十九}

黃帝曰:余聞先師,有所心藏,弗著於方,余願聞而藏之,則而行之,上以治民,下以治身,使百姓無病,上下和親,德澤下流,子孫無憂,傳於後世,無所終時,可得聞乎?岐伯曰:遠乎哉問也。夫治與民,自治,治彼與治此,治小與治大,治國與治家,未有逆而能治之也,夫惟順而已矣。順者,非獨陰陽脈,論氣之逆順也,百姓人民,皆欲順其志也。
黃帝曰:順之奈何?岐伯曰:入國問俗,入家問諱,上堂問禮,臨病人問所便。黃帝曰:便病人奈何?岐伯曰:夫中熱消癉則便寒,寒中之屬則便熱,胃中熱則消谷,令人懸心善,齊以上皮熱。腸中熱,則出黃如糜,齊以下皮寒。胃中寒,則腹脹,腸中寒,則腸鳴飧洩。胃中寒,腸中熱,則脹而且洩。胃中熱,腸中寒,則疾小腹痛脹。黃帝曰:胃欲寒飲,腸欲熱飲,兩者相逆,便之奈何?且夫王公大人,血食之君,驕恣從欲輕人,而無能禁之,禁之則逆其志,順之則加其病,便之奈何,治之何先。
岐伯曰:人之情,莫不惡死而樂生。告之以其敗,語之以其善,導之以其所便,開之以其所苦,雖有無道之人,惡有不聽者乎?黃帝曰:治之奈何?岐伯曰:春夏先治其標,後治其本,秋冬先治其本,後治其標。黃帝曰:便其相逆者奈何?岐伯曰:便此者,飲食衣服,亦欲適寒溫,寒無淒愴,暑無出汗。食飲者,熱無灼灼,寒無滄滄,寒溫中適,故氣將持,乃不致邪僻也。

黃帝曰:本藏以身形支節肉,候五藏六府之大小焉。今夫王公大人,臨朝即位之君,而問焉,誰可捫循之。而後答乎?岐伯曰:身形支節者,藏府之蓋也,非面部之閱也。黃帝曰:五藏之氣,閱於面者,余已知之矣,以支節知而閱之,奈何?岐伯曰:五藏六府者,肺為之蓋,巨肩陷,咽喉見其外。黃帝曰:善。岐伯曰:五藏六府,心為之主,缺盆為之道,骷骨有餘,以候。黃帝曰:善。岐伯曰:肝者,主為將,使知候外,欲知堅固,視目小大。黃帝曰:善。岐伯曰:脾者,主為衛,使之迎糧,視舌好惡,以知吉凶。黃帝曰:善。岐伯曰:腎者:主為外,使之遠聽,視耳好惡,以知其性。黃帝曰:善。
願聞六府之候。岐伯曰:六府者,胃為之海,廣骸大頸張胸,五穀乃容,鼻隧以長,以候大腸,厚,人中長,以候小腸,目下果大,其膽乃橫,鼻孔在外,膀胱漏洩,鼻柱中央起。三焦乃約,此所以候六府者也。上下三等,藏安且良矣。



\section{決氣第三十}

黃帝曰:余聞人有精氣津液血脈,余意以為一氣耳,今乃辨為六名,余不知其所以然。岐伯曰:兩神相搏,合而成形,常先身生,是謂精。何謂氣。岐伯曰:上焦開發,宣五穀味,熏膚,充身,澤毛,若霧露之溉,是謂氣。何謂津。岐伯曰:腠理髮洩,汗出溱溱,是謂津。何謂液。岐伯曰:谷入氣滿,淖澤注於骨,骨屬屈伸,澤補益腦髓,皮膚潤澤,是謂液。何謂血。岐伯曰:中焦受氣,取汁變化而赤,是謂血。何謂脈。岐伯曰:壅遏滎氣,令無所避,是謂脈。

黃帝曰:六氣者,有餘不足,氣之多少,腦髓之虛實,血脈之清濁,何以知之。岐伯曰:精脫者,耳聾。氣脫者,目不明。津脫者,腠理開,汗大洩,。液脫者,骨屬屈伸不利,色天,腦髓消,脛,耳數鳴。血脫者,色白,夭然不澤,其脈空虛,此其候也。

黃帝曰:六氣者,貴賤何如?岐伯曰:六氣者,各有部主也,其貴賤善惡,可為常主,然五穀與胃為大海也。


\section{腸胃第三十一}

黃帝問於伯高曰:余願聞六府傳谷者,腸胃之小大長短,受谷之多少奈何?伯高曰:請盡言之,谷所從出入淺深遠近長短之度,唇至齒,長九分,口廣二寸半,齒以後至厭,深三寸半,大容五合,舌重十兩,長七寸,廣二寸半。咽門重十兩,廣二寸半,至胃長一尺六寸。
紆曲屈伸之,長二尺六寸,大一尺五寸徑五寸,大容三斗五升。第三節小腸後附脊左環,回周疊積,其注於迴腸者,外附於齊,上回運環十六曲,大二寸半,徑八分分之少半,長三丈三尺。迴腸當齊左環,回周葉積而下,回運環反十六曲,大四寸,徑一寸寸之少半,長二丈一尺。廣腸傳脊,以受迴腸,左環葉脊上下辟,大八寸,徑二寸寸之大半,長二尺八寸。第四節腸胃所入至所出,長六丈四寸四分,回曲環反,三十二曲也。



\section{平人絕谷第三十二}

黃帝曰:願聞人之不食,七日而死,何也?伯高曰:臣請言其故。胃大一尺五寸,徑五寸,長二尺六寸,橫屈受水谷三斗五升,其中之谷,常留二斗,水一斗五升而滿,上焦洩氣,出其精微,悍滑疾,下焦下溉諸腸。

小腸大二寸半,徑八分分之少半,長三丈二尺,受谷二斗四升,水六升三合合之大半。迴腸大四寸,徑一寸寸之少半,長二丈一尺,受谷一斗,水七升半。廣腸大八寸,徑二寸寸之大半,長二尺八寸,受谷九升三合八分合之一。

腸胃之長,凡五丈八尺四寸,受水谷九斗二升一合合之大半,此腸胃所受水谷之數也。平人則不然,胃滿則腸虛,腸滿則胃虛,更虛更滿,故氣得上下,五藏安定,血脈和則精神乃居,故神者水谷之精氣也。故腸胃之中,當留谷二斗,水一斗五升,故平人日再後,後二升半,一日中五升,七日五七三斗五升,而留水谷盡矣。故平人不食飲七日而死者,水谷精氣津液皆盡故也。


\section{海論第三十三}

黃帝問於岐伯曰:余聞刺法於夫子,夫子之所言,不離於滎衛血氣。夫十二經脈者,內屬於府藏,外絡於支節,夫子乃合之於四海乎?岐伯答曰:人亦有四海,十二經水,經水者,皆注於海,海有東南西北,命曰四海。黃帝曰:以人應之奈何?岐伯曰:人有髓海,有血海,有氣海,有水穀之海,凡此四者,以應四海也。黃帝曰:遠乎哉。夫子之合人天地四海也,願聞應之奈何?岐伯答曰:日必先明知陰陽表裡滎輸所在,四海定矣。

黃帝曰:定之奈何?岐伯曰:胃者水穀之海,其輸上在氣街,下至三里。衝脈者,為十二經之海,其輸上在於大杼,下出於巨虛之上下廉。羶中者,為氣之海,其輸上在柱骨之上下,前在於人迎。腦為髓之海,其輸上在於其蓋,下在風府。黃帝曰:凡此四海者,何利何害,何生何敗。岐伯曰:得順者生,得逆者敗,知調者利,不知調者害。

黃帝曰:四海之逆順奈何?岐伯曰:氣海有餘者,氣滿胸中,息面赤。氣海不足,則氣少不足以言。血海有餘,則常想其身大,怫然不知其所病。血海不足,亦常想身小,狹然不知其所病。水穀之海有餘,則腹滿。水穀之海不足,則飢不受谷食。髓海有餘,則輕勁多力,自過其度。髓海不足,則腦轉耳鳴,脛眩冒,目無所見,懈怠安臥。黃帝曰:余已聞逆順,調之奈何?岐伯曰:審守其輸,而調其虛實,無犯其害,順者得復。逆者必敗。黃帝曰:善。



\section{五亂第三十四}

黃帝曰:經脈十二者,別為五行,分為四時,何失而亂,何得而治。岐伯曰:五行有序,四時有分,相順則治,相逆則亂。黃帝曰:何謂相順。岐伯曰:經脈十二者,以應十二月,十二月者,分為四時,四時者,春夏冬秋,其氣各異,滎衛相隨,陰陽已和,清濁不相干,如是則順之而治。
黃帝曰:何謂逆而亂。岐伯曰:清氣在陰,濁氣在陽,滎氣順脈,衛氣逆行,清濁相干,亂於胸中,是謂大。故氣亂於心,則煩心密嘿,首靜伏。亂於肺,則仰喘喝,接手以呼。亂於腸胃,則為霍亂。亂於臂脛,則為四厥。亂於頭,則為厥逆,頭重眩僕。

黃帝曰:五亂者,刺之有道乎?岐伯曰:有道以來,有道以去,審知其道,是謂身寶。黃帝曰:善。願聞其道。岐伯曰:氣在於心者,取之手少陰心主之輸。氣在於肺者,取之手太陰滎足少陰輸。氣在於腸胃者,取之足太陰陽明,不下者,取之三里。氣在於頭者,取之天柱大杼,不知,取足太陽滎輸。氣在於臂足,取之先去血脈,後取其陽明少陽之滎輸。黃帝曰:補瀉奈何?岐伯曰:徐入徐出,謂之導氣,補瀉無形,謂之同精,是非有餘不足也,亂氣之相交也。黃帝曰:允乎哉道,明乎哉論,請著之玉版,命曰治亂也。

\section{脹論第三十五}

黃帝曰:脈之應於寸口,如何而脹。岐伯曰:其脈大堅以澀者,脹也。黃帝曰:何以知藏府之脹也。岐伯曰:陰為藏,陽為府。黃帝曰:夫氣之令人脹也,在於血脈之中邪藏府之內乎?岐伯曰:三者皆存焉。然非脹之舍也。黃帝曰:願聞脹之舍。岐伯曰:夫脹者,皆在於藏府之外,排藏府而郭胸脅,脹皮膚,故命曰:脹。黃帝曰:藏府之在胸脅腹裡之內也,若匣匱之藏禁器也,各有次舍,異名而同處,一域之中,其氣各異,願聞其故。黃帝曰:未解其意。再問岐伯曰:夫胸腹,藏府之郭也。羶中者,心主之宮城也。胃者,太倉也。咽喉小腸者,傳送也。胃之五竅者,閭裡門戶也。廉泉玉英者,津液之道也。故五藏六府者,各有畔界,其病各有形狀。滎氣循脈,衛氣逆為脈脹,衛氣並脈循分為膚脹,三里而瀉,近者一下,遠者三下,無問虛實,工在疾瀉。

黃帝曰:願聞脹形。岐伯曰:夫心脹者,煩心短氣。臥不安。肺脹者,虛滿而喘。肝脹者,脅下滿而痛引小腹。脾脹者,善噦,四支煩,體重不能勝衣,臥不安。腎脹者,腹滿引背,央央然腰髀痛。六府脹。胃脹者,腹滿,胃脘痛,鼻聞焦臭,妨於食,大便難。大腸脹者,腸鳴而痛濯濯,冬日重感於寒,則飧洩不化。小腸脹者,少腹脹,引腰而痛。膀胱脹者,少腹滿而氣癃。三焦脹者,氣滿於皮膚中,輕輕然而不堅。膽脹者,脅下痛脹,口中苦,善太息。凡此諸脹者,其道在一,明知逆順,針數不失,瀉虛補實,神去其室,致邪失正,真不可定,之所敗,謂之天命,補虛瀉實,神歸其室。久塞其空,謂之良工。

黃帝曰:脹者焉生,何因而有。岐伯曰:衛氣之在身也,常然並脈循分肉,行有逆順,陰陽相隨,乃得天和,五藏更始,四時有序,五穀乃化,然後厥氣在上,滎衛留止,寒氣逆上,真邪相攻,兩氣相搏,乃合為脹也。黃帝曰:善。何以解惑。岐伯曰:合之於真,三合而得。帝曰: 善。黃帝問於岐伯曰:脹論言無問虛實,工在疾瀉,近者一下,遠者三下,今有其三而不下者,其過焉在。岐伯對曰:此言陷於肉盲,而中氣穴者也。不中氣穴,則氣內閉,針不陷盲,則氣不行,上越中肉,則衛氣相亂,陰陽相逐,其於脹也,當瀉不瀉,氣故不下,三而不下,必更其道,氣下乃止,不下復始,可以萬全,烏有殆者乎,其於脹也,必審其,當瀉則瀉,當補則補,如鼓應桴,惡有不下者乎?



\section{五癃津液別第三十六}

黃帝問於岐伯曰:水谷入於口,輸於腸胃,其液別為五,天寒衣薄,則為溺與氣,天熱衣厚則為汗,悲哀氣並則為泣,中熱胃緩則為唾,邪氣內逆則氣為之閉塞而不行,不行則為水脹,余知其然也,不知其所由生,願聞其道。
岐伯曰:水谷皆入於口,其味有五,各注其海,津液各走其道,故三焦出氣,以溫肌肉,充皮膚,為其津,其流而不行者為液。天暑衣厚則腠理開,故汗出,寒留於分肉之間,聚沫則為痛,天寒則腠理閉,氣濕不行,水下留於膀胱,則為溺與氣。五藏六府,心為之主,耳為之聽,目為之候,肺為之相,肝為之將,脾為之衛,腎為之主外。故五藏六府之津液,盡上滲於目,心悲氣並,則心繫急,心繫急則肺舉,肺舉則液上溢。夫心繫與肺,不能盡舉,乍上乍下,故而泣出矣。中熱則胃中消谷,消谷則蟲上下作,腸胃充郭,故胃緩,胃緩則氣逆,故唾出。
五穀之津液和合而為膏者,內滲入於骨空,補益腦髓,而下流於陰陽。陰陽不和,則使液溢而下流於陰,髓液皆減而下,下過度則虛,虛,故腰背痛而脛。陰陽氣道不通,四海塞閉,三焦不瀉,津液不化,水谷並於腸胃之中,別於迴腸,留於下焦,不得滲膀胱,則下焦脹,水溢則為水脹,此津液五別之逆順也。

\section{五閱五使第三十七}

黃帝問於岐伯曰:余聞刺有五官五閱以觀五氣,五氣者,五藏之使也,五時之副也,願聞其五使當安出。岐伯曰:五官者,五藏之閱也。黃帝曰:願聞其所出,令可為常。岐伯曰:脈出於氣口,色見於明堂,五色更出,以應五時,各如其常,經氣入藏,必當治裡。帝曰:善。五色獨決於明堂乎?岐伯曰:五官已辨,闕庭必張,乃立明堂,明堂廣大,蕃蔽見外,方壁高基,引垂居外,五色乃治,平博廣大,壽中百歲。見此者,刺之必已,如是之人者,血氣有餘,肌肉堅致,故可苦以針。

黃帝曰:願聞五官。岐伯曰:鼻者,肺之官也。目者,肝之官也。口者,脾之官也。舌者,心之官也。耳者,腎之官也。黃帝曰:以官何候。岐伯曰:以候五藏。故肺病者,喘息鼻張。肝病者,青。脾病者,黃。心病者,舌卷短,顴赤。腎病者,顴與顏黑。黃帝曰:五脈安出,五色安見,其常色殆者何如?岐伯曰:五官不辨,闕庭不張,小其明堂,蕃蔽不見,又埤其,下無基,垂角去外,如是者,雖平常殆,況加病哉。黃帝曰:五色之見於明堂,以觀五藏之氣,左右高下,各有形乎?岐伯曰:五藏之在中也,各以次舍左右上下,各如其度也。

\section{逆順肥瘦第三十八}

黃帝問於岐伯曰:余聞針道於夫子,眾多畢悉矣。夫子之道,應若失,而據未有堅然者也。夫子之問學熟乎,將審察於物而生之乎?岐伯對曰:聖人之為道者,上合於天,下合於地,中合於人事,必有明法,以起度數,法式檢押,乃後可傳焉。故匠人不能釋尺寸而意短長,廢繩墨而起平水也,工人不能置規而為員,去矩而為方。知用此者,固自然之物,易用之教,逆順之常也。黃帝曰:願聞自然奈何?岐伯曰:臨深決水,不用工力,而水可竭也,循掘決沖,而經可通也,此言氣之滑澀,血之清濁,行之逆順也。

黃帝曰:願聞人之黑白肥瘦小長,各有數乎?岐伯曰:年質壯大,血氣充盈,膚革堅固,因加以邪,刺此者,深而留之,此肥人也。廣肩,腋項肉薄,皮厚而黑色,臨臨然,其血黑以濁,其氣澀以遲,其為人也,貪而於取與。刺此者,深而留之,多益之數也。黃帝曰:刺瘦人奈何?岐伯曰:瘦人者,皮薄色少,肉廉廉然,薄輕言,其血清氣滑,易脫於氣,易損於血,刺此者,淺而疾之。黃帝曰:刺常人奈何?岐伯曰:視其白黑,各為調之,其端正惇厚者,其血氣和調,刺此者,無失常數也。黃帝曰:刺壯士真骨者,奈何?岐伯曰:刺壯士真骨,堅肉緩節監監然,此人重則氣澀血濁,刺此者,深而留之,多益其數。勁則氣滑血清,刺此者,淺而疾之。黃帝曰:刺嬰兒奈何?岐伯曰:嬰兒者,其肉脆,血少氣弱,刺此者,以毫針,淺刺而疾髮針,日再可也。

黃帝曰:臨深決水奈何?岐伯曰:血清氣濁,疾瀉之,則氣竭焉。黃帝曰:循掘決沖,奈何?岐伯曰:血濁氣澀,疾瀉之,則經可通也。黃帝曰:脈行之逆順,奈何?岐伯曰:手之三陰。從藏走手,手之三陽,從手走頭,足之三陽,從頭走足,足之三陰,從足走腹。黃帝曰:少陰之脈獨下行,何也?岐伯曰:不然,夫衝脈者,五藏六府之海也,五藏六府皆稟焉。其上者,出於頏顙。滲諸陽,灌諸精。其下者,注少陰之大絡,出於氣街,循陰股內廉,入中,伏行骨內,下至內踝之後屬而別。其下者,並於少陰之經,滲三陰,其前者,伏行出跗屬,下循跗,入大指間,滲諸絡而溫肌肉。故別絡結則跗上不動,不動則厥,厥則寒矣。黃帝曰:何以明之。岐伯曰:以言導之,切而驗之,其非必動,然後乃可明逆順之行也。黃帝曰:窘乎哉,聖人之為道也,明於日月,微於毫,其非夫子,孰能道之也。



\section{血絡論第三十九}

黃帝曰:願聞其奇邪而不在經者。岐伯曰:血絡是也。黃帝曰:刺血絡而僕者,何也?血出而射者,何也?血少黑而濁者,何也?血出清而半為汁者,何也?髮針而腫者,何也?血出若多若少而面色蒼蒼者,何也?髮針而面色不變而煩者,何也?多出血而不動搖者,何也?願聞其故。
岐伯曰:脈氣盛而血虛者,刺之則脫氣,脫氣則僕。血氣俱盛而陰氣多者,其血滑,刺之則射。陽氣畜積,久留而不瀉者,血黑以濁,故不能射。新飲而液滲於絡,而未合和於血也,故血出而汁別焉。其不新飲者,身中有水,久則為腫。陰氣積於陽,其氣因於絡,故刺之血未出而氣先行,故腫。陰陽之氣,其新相得而未和合,因而瀉之,則陰陽俱脫,表裡相離,故脫色而蒼蒼然。刺之血出多,色不變而煩者,刺絡而虛經,虛經之屬於陰者。陰脫故煩悶。陰陽相得而合為痹者,此為內溢於經,外注於絡,如是者,陰陽俱有餘,雖多出血而弗能虛也。

黃帝曰:相之奈何?岐伯曰:血脈者,盛堅橫以赤,上下無常處,小者如針,大者如筋,則而瀉之萬全也,故無失數矣,失數而反,各如其度。黃帝曰:針入而肉著,何也?岐伯曰:熱氣因於針,則針熱,熱則肉著於針,故堅焉。

\section{陰陽清濁第四十}

黃帝曰:余聞十二經脈,以應十二經水者,其五色各異,清濁不同,人之血氣若一,應之奈何?岐伯曰:人之血氣,苟能若一,則天下為一矣,惡有亂者乎?黃帝曰:余聞一人,非問天下之眾。岐伯曰:夫一人者,亦有亂氣,天下之眾,亦有亂人,其合為一耳。黃帝曰:願聞人氣之清濁。岐伯曰:受谷者濁,受氣者清。清者注陰,濁者注陽。濁而清者,上出於咽。清而濁者,則下行。清濁相干,命曰亂氣。

黃帝曰:夫陰清而陽濁,濁者有清,清者有濁,清濁別之奈何?岐伯曰:氣之大別,清者上注於肺,濁者下走於胃,胃之清氣,上出於口,肺之濁氣,下注於經,內積於海。黃帝曰:諸陽皆濁,何陽獨甚乎?岐伯曰:手太陽獨受陽之濁,手太陰獨受陰之清,其清者上走空竅,其濁者獨下行諸經,諸陰皆清,足太陰獨受其濁。

黃帝曰:治之奈何?岐伯曰:清者其氣滑,濁者其氣澀,此氣之常也。故刺陰者,深而留之,刺陽者,淺而疾之,清濁相干者,以數調之也。

\section{陰陽系日月第四十一}

黃帝曰:余聞天為陽,地為陰,日為陽,月為陰,其合之於人,奈何?岐伯曰:腰以上為天,腰以下為地,故天為陽,地為陰。故足之十二經脈以應十二月,月生於水,故在下者為陰。手之十指,以應十日,日主火,故在上者為陽。

黃帝曰:合之於脈,奈何?岐伯曰:寅者,正月之生陽也,主左足之少陽。未者,六月,主右足之少陽。卯者,二月,主左足之太陽。午者,五月,主右足之太陽。辰者,三月,主左足之陽明。巳者,四月,主右足之陽明,此兩陽合於前,故曰陽明。申者,七月之生陰也,主右足之少陰。丑者,十二月,主左足之少陰。酉者,八月,主右足之太陰。子者,十一月,主左足之太陰。戍者,九月,主右足之厥陰。亥者,十月,主左足之厥陰,此兩陰交盡,故曰厥陰。

甲主左手之少陽,己主右手之少陽,乙主左手之太陽,戊主右手之太陽,丙主左手之陽明,丁主右手之陽明,此兩火併合,故為陽明。庚主右手之少陰,癸主左手之少陰,辛主右手之太陰,壬主左手之太陰。

故足之陽者,陰中之少陽也。足之陰者,陰中之太陰也。手之陽者,陽中之太陽也。手之陰者,陽中之少陰也。腰以上者為陽,腰以下者為陰。其於五藏也,心為陽中之太陽,肺為陽中之少陰,肝為陰中之少陽,脾為陰中之至陰,腎為陰中之太陰。

黃帝曰:以治奈何?岐伯曰:正月二月三月,人氣在左,無刺左足之陽。四月五月六月,人氣在右,無刺右足之陽。七月八月九月,人氣在右,無刺右足之陰。十月十一月十二月,人氣在左,無刺左足之陰。黃帝曰:五行以東方甲乙木王春,春者,蒼色,主肝,肝者,足厥陰也。今乃以甲為左手之少陽,不合於數,何也?岐伯曰:此天地之陰陽也,非四時五行之以次行也。且夫陰陽者,有名而無形,故數之可十,推之可百,數之可千,推之可萬,此之謂也。



\section{病傳第四十二}

黃帝曰:余受九針於夫子,而私覽於諸方,或有導引行氣喬摩炙熨刺炳飲藥之一者,可獨守耶,將盡行之乎?岐伯曰:諸方者,眾人之方也,非一人之所盡行也。黃帝曰:此乃所謂守一勿失,萬物畢者也。今余已聞陰陽之要,虛實之理,頃移之過,可治之屬,願聞病之變化,淫傳絕敗而不可治者,可得聞乎?岐伯曰:要乎哉問道,昭乎其如日醒,窘乎其如夜瞑,能被而服之,神與俱成,畢將服之,神自得之,生神之理,可著於竹帛,不可傳於子孫。黃帝曰:何謂日醒。岐伯曰:明於陰陽,如惑之解,如醉之醒。黃帝曰:何謂夜瞑。岐伯曰:乎其無聲,漠乎其無形,折毛髮理,正氣橫頃,淫恤衍,血脈傳溜,大氣入藏,腹痛下淫,可以致死,不可以致生。

黃帝曰:大氣入藏,奈何?岐伯曰:病先發於心,一日而之肺,三日而之肝,五日而之脾,三日不已,死,冬夜半,夏日中。病先發於肺,三日而之肝,一日而之脾,五日而之胃,十日不已,死,冬日入,夏日出。病先發於肝,三日而之脾,五日而之胃,三日而之腎,三日不已,死,冬日入,夏蚤食。病先發於脾,一日而之胃,二日而之腎,三日而之膂膀胱,十日不已,死,冬人定,夏晏食。病先發於胃,五日而之腎,三日而之膂膀胱,五日而上之心,二日不已,死,冬夜半,夏日。病先發於腎,三日而之膂膀胱,三日而上之心,三日而之小腸,三日不已,死,冬大晨,夏早晡。病先發於膀胱,五日而之腎,一日而之小腸,一日而之心,二日不已,死,冬雞鳴,夏下晡。諸病以次相傳,如是者,皆有死期,不可刺也,間一藏及二三四藏者,乃可刺也。



\section{淫邪發夢第四十三}

黃帝曰:願聞淫恤衍,奈何?岐伯曰:正邪從外襲內,而未有定舍,反淫於藏,不得定處,與滎衛俱行,而與魂魄飛揚,使人臥不得安而喜夢。氣淫於府,則有餘於外,不足於內,氣淫於藏,則有餘於內,不足於外。

黃帝曰:有餘不足有形乎?岐伯曰:陰氣盛,則夢涉大水而恐懼。陽氣盛,則夢大火而燔。陰陽俱盛,則夢相殺。上盛則夢飛,下盛則夢墮,甚則夢取,甚飽則夢予。肝氣盛,則夢怒。肺氣盛,則夢恐懼,哭泣,飛揚。心氣盛,則夢善笑,恐畏。脾氣盛,則夢歌樂,身體重不舉。腎氣盛,則夢腰脊兩解不屬。凡此十二盛者,至而瀉之,立已。

厥氣客於心,則夢見邱山煙火。客於肺,則夢飛揚。見金鐵之奇物。客於肝,則夢山林樹木。客於脾,則夢見邱陵大澤,壞屋風雨。客於腎,則夢臨淵,沒居水中。客於膀胱,則夢遊行。客於胃,則夢飲食。客於大腸,則夢田野。客於小腸,則夢聚邑沖衢。客於膽,則夢訟自刳。客於陰器,則夢接內。客於項,則夢斬首。客於脛,則夢行走而不能前,及居深地苑中。客於股肱,則夢禮節拜起。客於胞植,則夢便。凡此十五不足者,至而補之立已也。
順氣一日分為四時第四十四

黃帝曰:夫百病之所始生者,必起於燥濕寒暑風雨陰陽喜怒飲食居然,氣合而有形,得藏而有名,余知其然也。夫百病者,多以旦慧晝安,夕加夜甚,何也?岐伯曰:四時之氣使然。黃帝曰:願聞四時之氣。岐伯曰:春生夏長,秋收冬藏,是氣之常也,人亦應之。以一日分為四時,朝則為春,日中為夏,日入為秋,夜半為冬,朝則人氣始生,病氣衰,故旦慧。日中人氣長,長則勝邪,故安。夕則人氣始衰,邪氣始生,故加。夜半人氣入藏,邪氣獨居於身,故甚也。黃帝曰:其時有反者何也?岐伯曰:是不應四時之氣,藏獨主其病者,是必以藏氣之所不勝時者甚,以其所勝時者起也。黃帝曰:治之奈何?岐伯曰:順天之時,而病可與期,順者為工,逆者為。

黃帝曰:善。余聞刺有五變,以主五輸,願聞其數。岐伯曰:人有五藏,五藏有五變,五變有五輸,故五五二十五輸,以應五時。黃帝曰:願聞五變。岐伯曰:肝為牡藏,其色青,其時春,其音角,其味酸,其日甲乙。心為牡藏,其色赤,其時夏,其日丙丁,其音徵,其味苦。脾為牝藏,其色黃,其時長夏,其日戊己,其音宮,其味甘。肺為牝藏,其色白,其音商,其時秋,其日庚辛,其味辛。腎為牝藏,其色黑,其時冬,其日壬癸,其音羽,其味咸,是為五變。

黃帝曰:以主五輸奈何?藏主冬,冬刺井。色主春,春刺滎。時主夏,夏刺輸。音主長夏,長夏刺經。味主秋,秋刺合,是謂五變,以主五輸。黃帝曰:諸原安合,以致六輸。岐伯曰:原獨不應五時,以經合之,以應其數,故六六三十六輸。黃帝曰:何謂藏主冬,時主夏,音主長夏,味主秋,色主春,願聞其故。岐伯曰:病在藏者,取之井。病變於色者,取之滎。病時間時甚者,取之輸。病變於音者,取之經。經滿而血者,病在胃,乃以飲食不節得病者,取之於合,故命曰味主合。是謂五變也。



\section{外揣第四十五}

黃帝曰:余聞九針九篇,余親授其調,頗得其意。夫九針者,始於一而終於九,然未得其要道也。夫九針者,小之則無內,大之則無外,深不可為下,高不可為蓋,恍惚無窮,流溢無極,余知其合於天道人事四時之變也,然余願雜之毫毛,渾為一,可乎?岐伯曰:明乎哉問也,非獨針道焉,夫治國亦然。黃帝曰:余願聞針道,非國事也。岐伯曰:夫治國者,夫惟道焉,非道,何可小大深淺,雜合而為一乎?

黃帝曰:願卒聞之。岐伯曰:日與月焉,水與鏡焉,鼓與響焉。夫日月之明,不失其影,水鏡之察,不失其形,鼓響之應,不後其聲,動搖則應和,盡得其情。黃帝曰:窘乎哉,昭昭之明不可蔽,其不可蔽,不失陰陽也。合而察之,切而驗之,見而得之,若清水明鏡之不失其形也。五音不彰,五色不明,五藏波蕩,若是則內外相襲,若鼓之應桴,響之應聲,影之應形。故遠者司外揣內,近者,司內揣外,是謂陰陽之極,天地之蓋,請藏之靈蘭之室,弗敢使洩也。


\section{五變第四十六}

黃帝問於少俞曰:余聞百疾之始期也,必生於風雨寒暑,循毫毛而入腠理,或復還,或留止,或為風腫汗出,或為消癉,或為寒熱,或為留痹,或為積聚,奇邪淫溢,不可勝數,願聞其故。夫同時得病,或病此,或病彼,意者天之為人生風乎,何其異也。少俞曰:夫天之生風者,非以私百姓也,其行公平正直,犯者得之,避者得無殆,非求人而人自犯之。黃帝曰:一時遇風,同時得病,其病各異,願聞其故。少俞曰:善乎哉問,請論以比匠人,匠人磨斧斤,礪刀削,砍材,木之陰陽,尚有堅脆,堅者不入,脆者皮施,至其交節,而缺斤斧焉。夫一木之中,堅脆不同,堅者則剛,脆者易傷,況其材木之不同,皮之厚薄,汁之多少,而各異耶。夫木之蚤花先生葉者,遇春霜烈風,則花落而葉萎。久曝大旱,則脆木薄皮者,枝條汁少而葉萎。久陰淫雨,則薄皮多汁者,皮潰而漉。卒風暴起,則剛脆之木,枝折杌傷。秋霜疾風,則剛脆之木,根搖而葉落。凡此五者,各有所傷,況於人乎?黃帝曰:以人應木,奈何?少俞答曰:木之所傷也,皆傷其枝,枝之剛脆而堅,未成傷也。人之有常病也,亦因其骨節皮膚腠理之不堅固者,邪之所舍也,故常為病也。

黃帝曰:人之善病風厥漉汗者,何以候之,少俞答曰:肉不堅,腠理疏,則善病風。黃帝曰:何以候肉之不堅也。少俞答曰:肉不堅,而無分理,理者理,理而皮不致者,腠理疏,此言其渾然者。
黃帝曰:人之善病消癉者,何以候之。少俞答曰:五藏皆柔弱者,善病消癉。黃帝曰:何以知五藏之柔弱也。少俞答曰:夫柔弱者,必有剛強,剛強多怒,柔者易傷也。黃帝曰:何以候柔弱之與剛強。少俞答曰:此人皮膚薄而目堅固以深者,長沖直揚,其心剛,剛則多怒,怒則氣上逆,胸中畜積,血氣逆留,皮充肌,血脈不行,轉而為熱,熱則消肌膚,故為消癉,此言其人暴剛而肌肉弱者也。
黃帝曰:人之善病寒熱者,何以候之。少俞答曰:小骨弱肉者,善病寒熱。黃帝曰:何以候骨之小大,肉之堅脆,色之不一也。少俞答曰:顴骨者,骨之本也,顴大則骨大,顴小則骨小,皮膚薄而其肉無,其臂懦懦然,其地色殆然,不與其天同色,汗然獨異,此其候也。然後臂薄者,其髓不滿,故善病寒熱也。
黃帝曰:何以候人之善病痹者,少俞答曰:理而肉不堅者,善病痹。黃帝曰:痹之高下有處乎?少俞答曰:欲知其高下者,各視其部。
黃帝曰:人之善病腸中積聚者,何以候之。少俞答曰:皮膚薄而不澤,肉不堅而淖澤,如此,腸胃惡,惡則邪氣留止,積聚乃傷,脾胃之間,寒溫不次,邪氣稍至,畜積留止,大聚乃起。

黃帝曰:余聞病形,已知之矣,願聞其時。少俞答曰:先立其年,以知其時,時高則起,時下則殆,雖不陷下,當年有沖通,其病必起,是謂因形而生病,五變之紀也。



\section{本藏第四十七}

黃帝問於岐伯曰:人之血氣精神者,所以奉生而周於性命者也。經脈者,所以行血氣而滎陰陽,濡筋骨,利關節者也。衛氣者,所以溫分肉,充皮膚,肥腠理,司關闔者也。志意者,所以御精神,收魂魄,適寒溫,和喜怒者也。是故血和則經脈流行,滎覆陰陽,筋骨勁強,關節清利矣。衛氣和則分肉解利,皮膚調柔,腠理緻密矣。志意和則精神專直,魂魄不散,悔怒不起,五藏不受邪矣。寒溫和則六府化谷,風痹不作,經脈通利,支節得安矣。此人之常平也。五藏者,所以藏精神血氣魂魄者也。六府者,所以化水谷而行津液者也。此人之所以具受於天也,無智愚賢不肖,無以相倚也。然有其獨盡天壽,而無邪僻之病,百年不衰,雖犯風雨卒寒大暑,猶有弗能害也。有其不離屏蔽室內,無怵之恐,然猶不免於病,何也,願聞其故。岐伯曰:窘乎哉問也。五藏者,所以參天地,副陰陽,而運四時,化五節者也。五藏者,固有小大高下堅脆端正偏頃者,六府亦有小大長短厚薄結直緩急,凡此二十五者,各不同,或善或惡,或吉或凶,請言其方。

心小則安,邪弗能傷,易傷以憂。心大則憂不能傷,易傷於邪。心高則滿於肺中,而善忘,難開以言。心下則藏外,易傷於寒,易恐以言。心堅則藏安守固,心脆則善病消癉熱中,心端正則和利難傷,心偏頃則操持不一。無守司也。
肺小則少飲,不病喘喝。肺大則多飲,善病胸痹喉痹逆氣。肺高則上氣喘息。肺下則居賁迫肺,善脅下痛。肺堅則不病上氣。肺脆則苦病消癉易傷。肺端正則和利難傷。肺偏頃則胸偏痛也。
肝小則藏安,無脅下之痛。肝大則逼胃,迫咽則苦膈中,且脅下痛。肝高則上支賁切脅,為息賁。肝下則逼胃,脅下空,脅下空則易受邪。肝堅則藏安難傷。肝脆則善病消癉易傷。肝端正則和利難傷。肝偏頃則脅下痛也。
脾小則藏安,難傷於邪也。脾大則苦湊而痛,不能疾行。脾高則引季脅而痛。脾下則下加於大腸,下加於大腸則藏苦受邪,脾堅則藏安難傷。脾脆則善病消癉易傷。脾端正則和利難傷。脾偏傾則善滿善脹也。
腎小則藏安難傷。腎大則善病腰痛,不可以仰,易傷以邪。腎高則苦背膂痛,不可以仰。腎下則腰尻痛,不可以仰,為狐疝。腎堅則不病腰背痛。腎脆則苦病消癉易傷。腎端正則和利難傷。腎偏頃則苦腰尻痛也。凡此二十五變者,人之所苦常病也。

黃帝曰:何以知其然也。岐伯曰:赤色小理者,心小。理者,心大。無者,心高。小短舉者,心下,長者,心下堅。弱小以薄者,心脆。直下不舉者,心端正。倚一方者,心偏頃也。白色小理者,肺小。理者,肺大。巨肩反膺陷喉者,肺高。合腋張脅者,肺下。好肩背厚者,肺堅。肩背薄者,肺脆。背膺厚者。肺端正。脅偏疏者,肺偏頃也。青色小理者,肝小。理者,肝大。廣胸反者,肝高。合脅兔者,肝下。胸脅好者,肝堅。脅骨弱者,肝脆。膺腹好相得者,肝端正。脅骨偏舉者,肝偏頃也。黃色小理者,脾小。理者,脾大。揭唇者,脾高。唇下縱者,脾下。唇堅者,脾堅。唇大而不堅者,脾脆。唇上下好者,脾端正。唇偏舉者,脾偏頃也。黑色小理者,腎小。理者,腎大。高耳者,腎高。耳後陷者,腎下。耳堅者,腎堅。耳薄不堅者,腎脆。耳好前居牙車者,腎端正。耳偏高者,腎偏頃也。凡此諸變者,持則安,減則病也。
黃帝曰:善。然非余之所問也。願聞人之有不可病者,至盡天壽,雖有深憂大恐,怵惕之志,猶不能減也。甚寒大熱,不能傷也。其有不離屏蔽室內,又無怵惕之恐,然不免於病者,何也,願聞其故。岐伯曰:五藏六府,邪之舍也,請言其故。五藏皆小者,少病,苦焦心,大愁憂。五藏皆大者,緩於事,難使以憂。五藏皆高者,好高舉措。五藏皆下者,好出人下。五藏皆堅者,無病。五藏皆脆者,不離於病。五藏皆端正者,和利得人心。五藏皆偏頃者,邪心而善盜,不可以為人,平反覆言語也。

黃帝曰:願聞六府之應。岐伯答曰:肺合大腸,大腸者,皮其應。心合小腸,小腸者,脈其應。肝合膽,膽者,筋其應。脾合胃,胃者,肉其應。腎合三焦膀胱,三焦膀胱者,腠理毫毛其應。
黃帝曰:應之奈何?岐伯曰:肺應皮,皮厚者,大腸厚,皮薄者,大腸薄,皮緩腹裡大者,大腸大而長,皮急者,大腸急而短,皮滑者,大腸直,皮肉不相離者,大腸結。心應脈,皮厚者,脈厚,脈厚者,小腸厚。皮薄者,脈薄。脈薄者,小腸薄。皮緩者,脈緩,脈緩者,小腸大而長。皮薄而脈衝小者,小腸小而短。諸陽經脈皆多紆屈者,小腸結。脾應肉,肉堅大者,胃厚,肉麼者,胃薄,肉小而麼者胃不堅,肉不稱身者,胃下,胃下者,下脘約不利。肉不堅者,胃緩。肉無小裹累者,胃急。肉多少裹累者,胃結。胃結者,上脘約不利也。肝應爪,爪厚色黃者,膽厚。爪薄色紅者,膽薄。爪堅色青者,膽急。爪濡色赤者,膽緩。爪直色白無約者,膽直。爪惡色黑多絞者,膽結也。腎應骨,密理厚皮者,三焦膀胱厚。理薄皮者,三焦膀胱薄。疏腠理者,三焦膀胱緩。皮急而無毫毛者,三焦膀胱急。毫毛美而者,三焦膀胱直。稀毫毛者,三焦肪胱結也。
黃帝曰:厚薄美惡皆有形,願聞其所病。岐伯答曰:視其外應,以知其內藏,則知所病矣。

\section{禁服第四十八}

雷公問於黃帝曰:細子得受業,通於九針六十篇,旦暮勤服之,近者編絕,久者簡垢,然尚諷誦弗置,未盡解於意矣,外揣言渾為一,未知所謂也。夫大則無外,小則無內,大小無極,高下無度,之奈何?士之才力,或有厚薄,智慮褊淺,不能博大深奧,自強於學若細子,細子恐其散於後世,絕於子孫,敢問約之奈何?黃帝曰:善乎哉問也。此先師之所禁,坐私傳之也,割臂歃血之盟也,子若欲得之,何不齋乎,雷公再拜而起曰:請聞命於是也。乃齋宿三日而請曰:敢問今日正陽,細子願以受盟。黃帝乃與俱入齋室,割臂歃血。黃帝親祝曰:今日正陽,歃血傳方,敢有背此言者,反受其殃。雷公再拜曰:細子受之。黃帝乃左握其手,右受之書,曰:慎之慎之,吾為子言之。

凡刺之理,經脈為始,滎其所行,知其度量,內刺五藏,外刺六府,審察衛氣,為百病母,調其虛實,虛實乃止,瀉其血絡,血盡不殆矣。雷公曰:此皆細子之所以通,未知其所約也。

黃帝曰:夫約方者,猶約囊也,囊滿而弗約,則輸洩,方成弗約,則神與弗俱。雷公曰:願為下材者,弗滿而約之。黃帝曰:未滿而知約之以為工,不可以為天下師。雷公曰:願聞為工。

黃帝曰:寸口主中,人迎主外,兩者相應,俱往俱來,若引繩大小齊等,春夏人迎微大,秋冬寸口微大,如是者名曰平人。
人迎大一倍於寸口,病在足少陽,一倍而躁,在手少陽。人迎二倍,病在足太陽,二倍而躁,病在手太陽。人迎三倍,病在足陽明,三倍而躁,病在手陽明。盛則為熱,虛則為寒,緊則為痛痹,代則乍甚乍問。盛則瀉之,虛則補之,緊痛則取之分肉,代則取血絡,具飲藥,陷下則灸之,不盛不虛,以經取之,名曰經刺。人迎四倍者,且大且數,名曰溢陽,溢陽為外格,死不治。必審按其本末,察其寒熱,以驗其藏府之病。
寸口大於人迎一倍,病在足厥陰,一倍而躁,病在手心主。寸口二倍,病在足少陰,二倍而躁。在手少陰。寸口三倍,病在足太陰,三倍而躁,病在手太陰。盛則脹滿,寒中食不化,虛則熱中,出糜少氣,溺色變,緊則痛痹,代則乍痛乍止。盛則瀉之,虛則補之,緊則先刺而後灸之,代則取血絡,而後調之,陷下則徒灸之,陷下者,脈血絡於中,中有著血,血寒,故宜灸之,不盛不虛,以經取之。名曰經刺,寸口四倍者名曰內關,內關者,且大且數,死不治。必審察其本末之寒溫,以驗其藏府之病。

通其滎輸,乃可傳於大數。大數曰:盛則徒瀉之。虛則徒補之,緊則灸刺,且飲藥,陷下則徒灸之,不盛不虛,以經取之。所謂經治者,飲藥,亦曰灸刺。脈急則引,脈大以弱,則欲安靜,用力無勞也。



\section{五色第四十九}

雷公問於黃帝曰:五色獨決於明堂乎,小子未知其所謂也。黃帝曰:明堂者,鼻也。闕者,眉間也。庭者,顏也。蕃者頰側也。蔽者,耳門也。其間欲方大,去之十步,皆見於外,如是者壽,必中百歲。雷公曰:五官之辯,奈何?黃帝曰:明堂骨高以起平以直,五藏次於中央,六府挾其兩側,首面上於闕庭,王宮在於下極,五藏安於胸中,真色以致,病色不見,明堂潤澤以清,五官惡得無辯乎?雷公曰:其不辯者,可得聞乎?黃帝曰:五色之見也,各出其色部,部骨陷者,必不免於病矣。其色部乘襲者,雖病者甚,不死矣。雷公曰:官五色奈何?黃帝曰:青黑為痛,黃赤為熱,白為寒,是謂五官。
雷公曰:病之益甚與其方衰,如何。黃帝曰:外內皆在焉,切其脈口滑小緊以沉者,病益甚,在中。人迎氣大緊以浮者,其病益甚,在外。其脈口浮滑者,病日進。人迎沉而滑者,病日損。其脈口滑以沉者,病日進,在內。其人迎脈滑盛以浮者,其病日進,在外。脈之浮沉及人迎與寸口氣小大等者,病難已。病之在藏,沉而大者,易已,小為逆。病在府,浮而大者,其病易已。人迎盛堅者,傷於寒,氣口盛堅者,傷於食。
雷公曰:以色言病之間甚,奈何?黃帝曰:其色以明,沉夭者為甚,其色上行者,病益甚,其色下行,如雲徹散者,病方已。五色各有藏部,有外部,有內部也。色從外部走內部者,其病從外走內。其色從內走外者,其病從內走外。病生於內者,先治其陰,後治其陽,反者益甚。其病生於陽者,先治其外,後治其內,反者益甚。其脈滑大以代而長者,病從外來,目有所見,志有所惡,此陽氣之並也,可變而已。第四節雷公曰:小子聞風者,百病之始也,厥逆者,寒濕之起也,別之奈何?黃帝曰:常候闕中,薄澤為風,沖濁為痹,在地為厥,此其常也,各以其色言其病。雷公曰:人不病卒死,何以知之。黃帝曰:大氣入於藏府者,不病而卒死矣。雷公曰:病小愈而卒死者,何以知之。黃帝曰:赤色出兩顴,大如母指者,病雖小愈,必卒死。黑色出於庭,大如母指,必不病而卒死。雷公再拜曰:善哉,其死有期乎?黃帝曰:察色以言其時。

雷公曰:善乎,願卒聞之。黃帝曰:庭者,首面也。闕上者,咽喉也。闕中者,肺也。下極者,心也。直下者,肝也。肝左者,膽也。下者,脾也。方上者,胃也。中央者,大腸也。挾大腸者,腎也。當腎者,齊也。面王以上者,小腸也。面王以下者,膀胱子處也。顴者,肩也。顴後者,腎也。臂下者,手也。目內上者,膺乳也。挾繩而上者,背也。循牙車以下者,股也。中央者,膝也。膝以下者,脛也。當脛以下者,足也。巨分者,股裡也。巨屈者,膝臏也。此五藏六府支節之部也。
各有部分,有部分,用陰和陽,用陽和陰,當明部分,萬舉萬當,能別左右,是謂大道,男女異位,故曰陰陽,審察澤夭,謂之良工。沉濁為內,浮澤為外,黃赤為風,青黑為痛,白為寒,黃而膏潤為膿,赤甚者為血痛,甚為攣,寒甚為皮不仁。五色各見其部,察其浮沉,以知淺深,察其澤夭,以觀成敗,察其散搏,以知遠近,視色上下,以知病處,積神於心,以知往今。故相氣不微,不知是非,屬意勿去,乃知新故。色明不,沉夭為甚,不明不澤,其病不甚。其色散,駒駒然未有聚,其病散而氣痛,聚未成也。腎乘心,心先病,腎為應,色皆如是。男子色在於面王,為小腹痛,下為卵痛,其圜直為莖痛,高為本,下為首,狐疝陰之屬也。女子在於面王,為膀胱子處之病,散為痛,搏為聚,方員左右,各如其色形。其隨而下至胝,為淫,有潤如膏狀,為暴食不。左為左,右為右,其色有邪,聚散而不端,面色所指者也。色者,青黑赤白黃,皆端滿有別鄉。別鄉赤者,其色赤,大如榆莢,在面王為不日。其色上銳,首空上向,下銳下向,在左右如法,以五色命藏,青為肝,赤為心,白為肺,黃為脾,黑為腎。肝合筋,心合脈,肺合皮,脾合肉,腎合骨也。

\section{論勇第五十}

黃帝問於少俞曰:有人於此,並行並立,其年之長少等也,衣之厚薄均也,卒然遇烈風暴雨,或病或不病,或皆病,或皆不病,其故何也?少俞曰:帝問何急?黃帝曰:願盡聞之。少俞曰:春青風,夏陽風,秋涼風,各寒風。凡此四時之風者,其所病各不同形。黃帝曰:四時之風,病人如何?少俞曰:黃色薄皮弱肉者,不勝春之虛風;白色薄皮弱肉者,不勝夏之虛風;青色薄皮弱肉,不勝秋之虛風;赤色薄皮弱肉,不勝冬之虛風也。黃帝曰:黑色不病乎?少俞曰:黑色而皮厚肉堅,固不傷於四時之風。其皮薄而肉不堅,色不一者,長夏至而有虛風者,病矣。其皮厚而肌肉堅者,長夏至而有虛風,不病矣。其皮厚而肌肉堅者,必重感於寒,外內皆然,乃病。黃帝曰:善。

黃帝曰:夫人之忍痛與不忍痛者,非勇怯之分也。夫勇士之不忍痛者,見難則前,見痛則止;夫怯士之忍痛者,聞難則恐,遇痛不動。夫勇士之忍痛者,見難不恐,遇痛不動;夫怯士之不忍痛者,見難與痛,目轉面盼,恐不能言,失氣驚,顏色變化,乍死乍生。余見其然也,不知其何由,願聞其故。少俞曰:夫忍痛與不忍痛者,皮膚之薄厚,肌肉之堅脆緩急之分也,非勇怯之謂也。黃帝曰:願聞勇怯之所由然。少俞曰:勇士考,目深以固,長衡直暢,三焦理橫,其心端直,其肝大以堅,其膽滿以傍,怒則氣盛而胸張,肝舉而膽橫,毗裂而目揚,毛起而面蒼,此勇士之由然者也。黃帝曰:願聞怯士之所由然。少俞曰:怯士者,目大而不減,陰陽相失,其焦理縱,(骨曷)(骨亏)短而小,肝系緩,其膽不滿而縱,腸胃挺,脅下空,雖方大怒,氣不能滿其胸,肝肺雖舉,氣衰復下,故不能久怒,此怯士之所由然者也。黃帝曰;怯士之得酒,怒不避勇士者,何藏使然?少俞曰:酒者,水谷之精,熟谷之液也,其氣慓悍,其入於胃中,則胃脹,氣上逆,滿於胸中,肝浮膽橫。當是之時,固比於勇士,氣衰則悔。與勇士同類,不知避之,名曰酒悖也。



\section{背俞第五十一}

黃帝問於岐伯曰:願聞五藏之俞,出於背者。岐伯曰:胸中大俞,在杼骨之端,肺俞在三焦之間,心俞在五焦之間,膈俞在七焦之間,肝俞在九焦之間,脾俞在十一焦之間,腎俞在十四焦之間,皆挾脊相去三寸所,則欲得而驗之,按其處,應在中而痛解,乃其俞也。灸之則可,刺之則不可。氣盛則瀉之,虛則補之。以火補者,無吹其火,須自滅也。以火瀉者,疾吹其火,傳其艾,須其火滅也。



\section{衛氣第五十二}

黃帝曰:五藏者,所以藏精神魂魄者也。六府者,所以受水谷而化行物者也。其氣內干五藏,而外絡支節。其浮氣之不循經者,為衛氣。其精氣之行於經者,為滎氣。陰陽相隨,外內相貫,如環之無端,亭亭淳淳乎,孰能窮之。然其分別陰陽,皆有標本虛實所離之處。能別陰陽十二經者,知病之所生。候虛實之所在者,能得病之高下。知六府之氣街者,能知解結契紹於門戶。能知虛實堅軟者,知補瀉之所在。能知六經標本者,可以無惑於天下。

岐伯曰:博哉,聖帝之論,臣請盡意悉言之。足太陽之本,在跟以上五寸中,標在兩絡命門,命門者,目也。足少陽之本,在竅陰之間,標在窗籠之前,窗籠者,耳也。足少陰之本,在內踝下上三寸中,標在背俞與舌下兩脈也。足厥陰之本,在行間上五寸所,標在背俞也。足陽明之本,在厲兌,標在人迎頰挾頏顙也。足太陰之本,在中封前上四寸之中,標在背俞與舌本也。手太陽之本,在外踝之後,標在命門之上一寸也。手少陽之本,在小指次指之間上二寸,標在耳後上角下外也。手陽明之本,在肘骨中,上至別陽,標在顏下合鉗上也。手太陰之本,在寸口之中,標在腋內動也。手少陰之本,在銳骨之端,標在背俞也。手心主之本,在掌後兩筋之間二寸中,標在腋下下三寸也。凡候此者,下虛則厥,下盛則熱,上虛則眩,上盛則熱痛。故實者,絕而止之,虛者,引而起之。

請言氣街,胸氣有街,腹氣有街,頭氣有街,脛氣有街,故氣在頭者,止之於腦。氣在胸者,止之膺與背俞。氣在腹者,止之背俞,與衝脈於齊左右之動脈者。氣在脛者,止之於氣街,與承山踝上以下。取此者,用毫針,必先按而在,久應於手,乃剌而予之。所治者,頭痛眩僕,腹痛中滿暴脹,及有新積。痛可移者,易已也,積不痛,難已也。



\section{論痛第五十三}

黃帝問於少俞曰:筋骨之強弱,肌肉之堅脆,皮膚之厚薄,腠理之疏密,各不同,其於針石火之痛如何。腸胃之厚薄堅脆亦不等,其於毒藥何如?願盡聞之。少俞曰:人之骨強筋弱肉緩皮膚厚者,耐痛,其於針石之痛火亦然。黃帝曰:其耐火者,何以知之。少俞答曰:加以黑色而美骨者,耐火。黃帝曰:其不耐針石之痛者,何以知之。少俞曰:堅肉薄皮者,不耐針石之痛,於火亦然。
黃帝曰:人之病,或同時而傷,或易已,或難已,其故何如?少俞曰:同時而傷,其身多熱者,易已。多寒者,難已。
黃帝曰:人之勝毒,何以知之。少俞曰:胃厚色黑大骨及肥者,皆勝毒,故其瘦而薄胃者,皆不勝毒也。



\section{天年第五十四}

黃帝問於岐伯曰:願聞人之始生,何氣為基,何立而為,何失而死,何得而生。岐伯曰:以母為基,以父為,失神者死,得神者生也。黃帝曰:何者為神。岐伯曰:血氣已和,滎衛已通,五藏已成,神氣舍心,魂魄畢具,乃成為人。

黃帝曰:人之壽夭各不同,或夭壽,或卒死,或病久,願聞其道。岐伯曰:五藏堅固,血脈和調,肌肉解利,皮膚緻密,滎衛之行,不失其常,呼吸微徐,氣度以行,六府化谷,津液布揚,各如其常,故能長久。黃帝曰:人之壽百歲而死,何以致之。岐伯曰:使道隧以長,基高以方,通調滎衛,三部三里起,骨高肉滿,百歲乃得終。

黃帝曰:其氣之盛衰,以至其死,可得聞乎?岐伯曰:人生十歲,五藏始定,血氣已通,其氣在下,故好走。二十歲,血氣始盛,肌肉方長,故好趨。三十歲,五藏大定,肌肉堅固,血脈盛滿,故好步。四十歲,五藏六府十二經脈,皆大盛以平定,腠理始疏,滎華頹落,發頗頒白,平盛不搖,故好坐。五十歲,肝氣始衰,肝葉始薄,膽汁始滅,目始不明。六十歲,心氣始衰,善憂悲,血氣懈惰,故好臥。七十歲,脾氣虛,皮膚枯。八十歲,肺氣衰,魄離,故言善誤。九十歲,腎氣焦,四藏經脈空虛。百歲,五藏皆虛,神氣皆去,形骸獨居而終矣。

黃帝曰:其不能終壽而死者,何如?岐伯曰:其五藏皆不堅,使道不長,空外以張,喘息暴疾,又卑基,薄脈少血,其肉不石,數中風寒,血氣虛,脈不通,真邪相攻,亂而相引,故中壽而盡也。



\section{逆順第五十五}

黃帝問於伯高曰:余聞氣有逆順,脈有盛衰,刺有大約,可得於聞乎?伯高曰:氣之逆順者,所以應天地陰陽四時五行也。脈之盛衰者,所以候血氣之虛實有餘不足也。刺之大約者,必明知病之可刺,與其未可刺,與其已不可剌也。

黃帝曰:候之奈何?伯高曰:兵法曰:無迎逢逢之氣,無擊堂堂之陣。刺法曰:無刺之熱,無刺漉漉之汗,無刺渾渾之脈,無刺病與脈相逆者。黃帝曰:候其可刺奈何?伯高曰:上工,刺其未病者也。其次,刺其未盛者也。其次,刺其已衰者也。下工,刺其方襲者也,與其形之盛者也,與其病之與脈相逆者也。故曰,方其盛也,勿敢毀傷,刺其已衰,事必大昌。故曰,上工治未病,不治已病,此之謂也。



\section{五味第五十六}

黃帝曰:願聞谷氣有五味,其入五藏,分別奈何?伯高曰:胃者,五藏六府之海也,水谷皆入於胃,五藏六府,皆稟氣於胃。五味各走其所喜,谷味酸,先走肝,谷味苦,先走心,谷味甘,先走脾,谷味辛,先走肺,谷味咸,先走腎,谷氣津液已行,滎衛大通,乃化糟粕,以次傳下。

黃帝曰:滎衛之行奈何?伯高曰:谷始入於胃,其精微者,先出於胃之兩焦,以溉五藏,別出兩行,滎衛之道。其大氣之搏而不行者,積於胸中,命曰氣海,出於肺,循喉咽,故呼則出,吸則入。天地之精氣,其大數常出三入一,故谷不入半日則氣衰,一日則氣少矣。

黃帝曰:谷之五味,可得聞乎?伯高曰:請盡言之。五穀,米甘,麻酸,大豆咸,麥苦,黃黍辛。五果,棗甘,李酸,栗咸,杏苦桃辛。五畜,牛甘,犬酸,豬咸,羊苦,雞辛。五菜,葵甘,韭酸,藿咸,薤苦,蔥辛。五色,黃色宜甘,青色宜酸,黑色宜咸,赤色宜苦,白色宜辛。凡此五者,各有所宜。

所謂五色者,脾病者,宜食米飯牛肉棗葵。心病者,宜食麥羊肉杏薤。腎病者。宜食大豆黃卷豬肉栗藿。肝病者,宜食麻犬肉李韭。肺病者,宜食黃黍雞肉桃蔥。五禁,肝病禁辛,心病禁咸,脾病禁酸,腎病禁甘,肺病禁苦。肝色青,宜食甘,米飯牛肉棗葵皆甘。心色赤,宜食酸,犬肉麻李韭皆酸。脾色黃,宜食咸,大豆豕肉栗藿皆咸。肺色白,宜食苦,麥羊肉杏薤皆苦。腎色黑,宜食辛,黃黍雞肉桃蔥皆辛。



\section{水脹第五十七}

黃帝問於岐伯曰:水與膚脹,鼓脹,腸覃,石瘕,石水,何以別之。岐伯答曰:水始起也,目窠上微腫,如新臥起之狀,其頸脈動,時,陰股間寒,足脛腫,腹乃大,其水已成矣。以手按其腹,隨手而起,如裹水之狀,此其候也。
黃帝曰:膚脹何以候之。岐伯曰:膚脹者,寒氣客於皮膚之間,鼓鼓然不堅,腹大,身盡腫,皮厚,按其腹。而不起,腹色不變,此其候也。
鼓脹何如?岐伯曰:腹脹身皆大,大與膚脹等也,色蒼黃,腹筋起,此其候也。
腸覃何如?岐伯曰:寒氣客於腸外,與衛氣相搏,氣不得滎,因有所繫,癖而內著,惡氣乃起,肉乃生。其始生也,大如雞卵,稍以益大,至其成如懷子之狀,久者離藏,按之則堅,推之則移,月事以時下,此其候也。
石瘕何如?岐伯曰:石瘕生於胞中,寒氣客於子門,子門閉塞,氣不得通,惡血當瀉不瀉,杯以留止,日以益大,狀如懷子,月事不以時下,皆生於女子,可導而下。
黃帝曰:膚脹鼓脹,可刺耶。岐伯曰:先瀉其脹之血絡,後調其經,刺去其血絡也。

\section{賊風第五十八}

黃帝曰:夫子言賊風邪氣之傷人也,令人病焉,今有其不離屏蔽,不出室穴之中,卒然病者,非不離賊風邪氣,其故何也?岐伯曰:此皆嘗有所傷於濕氣,藏於血脈之中,分肉之間,久留而不去,若有所墮墜,惡血在內而不去。卒然喜怒不節,飲食不適,寒溫不時,腠理閉而不通,其開而遇風寒,則血氣凝結,與故邪相襲,則為寒痹。其有熱則汗出,汗出則受風,雖不遇賊風邪氣,必有因加而發焉。

黃帝曰:夫子之所言者,皆病人之所自知也,其無所遇邪氣,又無怵之所志,卒然而病者,其故何也,唯因鬼神之事乎?岐伯曰:此亦有故,邪留而未發,因而志有所惡,及有所慕,血氣內亂,兩氣相搏,其所從來者微,視之不見,聽而不聞,故似鬼神。黃帝曰:其祝而已者,其故何也?岐伯曰:先巫者,因知百病之勝,先知其病之所從生者,可祝而已也。



\section{衛氣失常第五十九}

黃帝曰:衛氣之留於腹中,積不行,苑蘊不得常所,使人支脅胃中滿,喘呼逆息者,何以去之。伯高曰:其氣積於胸中者,上取之。積於腹中者,下取之。上下皆滿者,傍取之。黃帝曰:取之奈何?伯高對曰:積於上,瀉大迎天突喉中。積於下者,瀉三里與氣街。上下皆滿者,上下取之,與季脅之下一寸。重者,雞足取之,診視其脈大而弦急,及絕不至者,及腹皮急甚者,不可刺也。黃帝曰:善。

黃帝問於伯高曰:何以知皮肉氣血筋骨之病也。伯高曰:色起兩眉薄澤者,病在皮。色青黃赤白黑者,病在肌肉。滎氣濡然者,病在血氣。目色青黃赤白黑者,病在筋。耳焦枯,受塵垢,病在骨。黃帝曰:病形何如,取之奈何?伯高曰:夫百病變化,不可勝數,然皮有部,肉有柱,血氣有輸,骨有屬。黃帝曰:願聞其故。伯高曰:皮之部,輸於四末。肉之柱,在臂脛諸陽分肉之間,與足少陰分間。血氣之輸,輸於諸絡,氣血留居則盛而起。筋部無陰無陽,無左無右,候病所在。骨之屬者,骨空之所以受益而益腦髓者也。黃帝曰:取之奈何?伯高曰:夫病變化,浮沉深淺,不可勝窮,各在其處,病間者淺之,甚者深之,間者少之,甚者眾之,隨變而調氣,故曰上工。

黃帝問於伯高曰:人之肥瘦大小寒溫,有老壯少小,別之奈何?伯高對曰:人年五十已上為老,二十已上為壯,十八已上為少,六歲已上為小。黃帝曰:何以度知其肥瘦。伯高曰:人有肥有膏有肉。黃帝曰:別此奈何?伯高曰:肉堅,皮滿者,肥。肉不堅,皮緩者,膏。皮肉不相離者,肉。黃帝曰:身之寒溫何如?伯高曰:膏者,其肉淖而理者,身寒。細理者,身熱脂者,其肉堅,細理者熱,理者寒。黃帝曰:其肥瘦大小奈何?伯高曰:膏者,多氣而皮縱緩,故能縱腹垂腴。肉者,身體容大。脂者,其身收小。黃帝曰:三者之氣血多少何如?伯高曰:膏者,多氣,多氣者,熱,熱者,耐寒。肉者,多血則充形,充形則平。脂者,其血清,氣滑少,故不能大。此別於眾人者也。黃帝曰:眾人奈何?伯高曰:眾人皮肉脂膏,不能相加也,血與氣,不能相多,故其形不小不大,各自稱其身,命曰眾人。黃帝曰:善。治之奈何?伯高曰:必先別其三形,血之多少,氣之清濁,而後調之。治無失常經。是故膏人縱腹垂腴,肉人者,上下容大,脂人者,雖脂不能大也。



\section{玉版第六十}

黃帝曰:余以小針為細物也,夫子乃言上合之於天,下合之於地,中合之於人,余以為過針之意矣,願聞其故。岐伯曰:何物大於天乎,夫大於針者,惟五兵者焉。五兵者,死之備也。非生之具,且夫人者,天地之鎮也,其不可不參乎?夫治民者,亦唯針焉,夫針之與五兵,其孰小乎?

黃帝曰:病之生時,有喜怒不測,飲食不節,陰氣不足,陽氣有餘,滎氣不行,乃發為癰疽。陰陽不通,兩熱相搏,乃化為膿,小針能取之乎?岐伯曰:聖人不能使化者為之,邪不可留也。故兩軍相當,旗幟相望,白刃陳於中野者,此非一日之謀也。能使其民令行,禁止士卒無白刃之難者,非一日之教也,須臾之得也。夫至使身被癰疽之病,膿血之聚者,不亦離道遠乎?夫癰疽之生,膿血之成也,不從天下,不從地出,積微之所生也。故聖人自治於未有形也,愚者遭其已成也。
黃帝曰:其已形,不予遭,膿已成,不予見,為之奈何?岐伯曰:膿已成,十死一生,故聖人勿使已成,而明為良方,著之竹帛,使能者踵而傳之後世,無有終時者,為其不予遭也。黃帝曰:其已有膿血而後遭乎?不道之以小針治乎?岐伯曰:以小治小者,其功小,以大治大者,多害,故其已成膿血者,其唯砭石鈹鋒之所取也。

黃帝曰:多害者其不可全乎?岐伯曰:其在逆順焉。黃帝曰:願聞逆順。岐伯曰:以為傷者,其白眼青,黑眼小,是一逆也。內藥而嘔者,是二逆也。腹痛喝甚,是三逆也。肩項中不便,是四逆也。音嘶色脫,是五逆也。除此五者,為順矣。黃帝曰:諸病皆有逆順,可得聞乎?岐伯曰:腹脹身熱脈大,是一逆也。腹鳴而滿,四支清洩,其脈大。是二逆也。而不止,脈大。是三逆也。且溲血脫形,其脈小勁,是四逆也。脫形,身熱,脈小以疾,是謂五逆也。如是者,不過十五日而死矣。其腹大脹,四末清。形脫,洩甚,是一逆也。腹脹便血,其脈大,時絕,是二逆也。溲血,形肉脫,脈搏,是三逆也。嘔血,胸滿引背,脈小而疾,是四逆也。嘔,腹脹且飧洩,其脈絕,是五逆也。如是者,不過一時而死矣,工不察此者而刺之,是謂逆治。

黃帝曰:夫子之言針甚駿,以配天地,上數天文,下度地紀,內別五藏,外次六府,經脈二十八會,盡有周紀,能殺生人,不能起死者,子能反之乎?岐伯曰:能殺生人,不能起死者也。黃帝曰:余聞之,則為不仁。然願聞其道,弗行於人。岐伯曰:是明道也。其必然也,其如刀劍之可以殺人。如飲酒使人醉也,雖勿胗,猶可知矣。黃帝曰:願卒聞之。岐伯曰:人之所受氣者,谷也。谷之所注者,胃也。胃者,水谷氣血之海也。海之所行雲氣者,天下也。胃之所出氣血者,經隧也。經隧者,五藏六府之大絡也,迎而奪之而已矣。黃帝曰:上下有數乎?岐伯曰:迎之五里,中道而止,五至而已,五往而藏之氣盡矣。故五五二十五,而竭其輸矣。此所謂奪其天氣者也,非能絕其命而頃其壽者也。黃帝曰:願卒聞之。岐伯曰:門而刺之者,死於家中,入門而刺之者,死於堂上。黃帝曰:善乎方,明哉道,請著之玉版,以為重寶,傳之後世,以為刺禁,令民勿敢犯也。



\section{五禁第六十一}

黃帝問於岐伯曰:余聞刺有五禁,何謂五禁。岐伯曰:禁其不可刺也。黃帝曰:余聞刺有五奪。岐伯曰:無瀉其不可奪者也。黃帝曰:余聞刺有五過。岐伯曰:補瀉無過其度。黃帝曰:余聞刺有五逆。岐伯曰:病與脈相逆,命曰五逆。黃帝曰:余聞刺有九宜。岐伯曰:明知九針之論,是謂九宜。
黃帝曰:何謂五禁,願聞其不可刺之時。岐伯曰:甲乙日自乘,無刺頭,無發蒙於耳內。丙丁日自乘,無振埃於肩喉廉泉。戊己日自乘四季,無刺腹去爪瀉水。庚辛日自乘,無刺關節於股膝,壬癸日自乘,無刺足脛,是謂五禁。
黃帝曰:何謂五奪。岐伯曰:形肉已奪,是一奪也。大奪血之後,是二奪也。大汗出之後,是三奪也。大洩之後,是四奪也。新產及大血,是五奪也,此皆不可瀉。
黃帝曰:何謂五逆。岐伯曰:熱病脈靜,汗已出,脈盛躁,是一逆也。病洩脈洪大,是二逆也。著痹不移,肉破,身熱,脈偏絕,是三逆也。淫而奪形,身熱,色夭然白,及後下血杯,血杯篤重,是謂四逆也。寒熱奪形,脈堅搏,是謂五逆也。

\section{動輸第六十二}

黃帝曰:經脈十二,而手太陰足少陰陽明,獨動不休,何也?岐伯曰:是明胃脈也。

胃為五藏六府之海,其清氣上注於肺,肺氣從太陰而行之,其行也,以息往來,故人一呼,脈再動,一吸,脈亦再動,呼吸不已,故動而不止。黃帝曰:氣之過於寸口也,上十焉息,下八焉伏,何道從還,不知其極。岐伯曰:氣之離藏也,卒然如弓弩之發,如水之下岸,上於魚以反衰,其餘氣衰散以上逆,故其行微。

黃帝曰:足之陽明,何因而動。岐伯曰:胃氣上注於肺,其悍氣上衝頭者,循咽上走空竅,循眼系,入絡腦,出,下客主人循牙車,合陽明,並下人迎,此胃氣別走於陽明者也。故陰陽上下,其動也若一。故陽病而陽脈小者為逆,陰病而陰脈大者,為逆,故陰陽俱靜俱動,若引繩相頃者,病。

黃帝曰:足少陰何因而動。岐伯曰:衝脈者,十二經之海也,與少陰之大絡,起於腎下,出於氣街,循陰股內廉,邪入中,循脛股內廉,並少陰之經,下入內踝之後。入足下。其別者,邪入踝,出屬跗上,入大指之間,注諸絡,以溫足脛,此脈之常動者也。

黃帝曰:滎衛之行也,上下相貫,如環之無端。今有其卒然遇邪氣,及逢大寒,手足懈惰,其脈陰陽之道,相輸之會,行相失也。氣何由還。岐伯曰:夫四末陰陽之會者,此氣之大絡也。四街者,氣之徑路也,故絡絕則徑通,四末解則氣從合,相輸如環。黃帝曰:善。此所謂如環無端,莫知其紀,此之謂也。



\section{五味論第六十三}

黃帝問於少俞曰:五味入於口也,各有所走,各有所病。酸走筋,多食之,令人癃。咸走血,多食之,令人喝。辛走氣,多食之,令人洞心。苦走骨,多食之,令人變嘔。甘走肉,多食之,令人心。余知其然也,不知其何由,願聞其故。

少俞答曰:酸入於胃,其氣澀以收,上之兩焦,弗能出入也,不出即留於胃中,胃中和溫,則下注膀胱,膀胱之脆薄以懦,得酸則縮綣,約而不通,水道不行,故癃。陰者,積筋之所終也,故酸入而走筋矣。
黃帝曰:咸走血,多食之,令人喝,何也?少俞曰:咸入於胃,其氣上走中焦,注於脈,則血氣走之,血與咸相得,則凝,凝則胃中汁注之,注之則胃中竭,竭則咽路焦,故舌本干而善喝。血脈者,中焦之道也,故咸入而走血矣。
黃帝曰:辛走氣,多食之,令人洞心,何也?少俞曰:辛入於胃,其氣走於上焦,上焦者,受氣而滎諸陽者也,姜韭之氣薰之,滎衛之氣,不時受之,久留心下,故洞心。辛與氣俱行,故辛入而與汗俱出。
黃帝曰:苦走骨,多食之,令人變嘔,何也?少俞曰:苦入於胃,五穀之氣,皆不能勝苦,苦入下脘,三焦之道,皆閉而不通,故變嘔。齒者,骨之所終也,故苦入而走骨,故入而復出知其走骨也。
黃帝曰:甘走肉,多食之,令人心,何也?少俞曰:甘入於胃,其氣弱小,不能上至於上焦,而與谷留於胃中者,令人柔潤者也,胃柔則緩,緩則蠱動,蠱動則令人心。其氣外通於肉,故甘走肉。



\section{陰陽二十五人第六十四}

黃帝曰:余聞陰陽之人,何如?伯高曰:天地之間,六合之內,不離乎五,人亦應之。故五五二十五人之政,而陰陽之人不與焉。其滎又不合於眾者也,余已知之矣,願聞二十五人之形,血氣之所生別,而以候從外知內,何如?岐伯曰:悉乎哉問也,此先師之也,雖伯高猶不能明之也。黃帝避席遵循而卻曰:余聞之。得其人弗教,是謂重失,得而之,天將厭之。余願得而明之,金匱藏之,不敢揚之。岐伯曰:先立五形金木水火土,別其五色,異其五形之人,而二十五人具矣。
黃帝曰:願卒聞之。岐伯曰:慎之慎之,臣請言之。木形之人,比於上角,似於蒼帝,其為人,蒼色,小頭,長面,大肩背,直身,小手,足好,有才,勞心,少力,多憂,勞於事。能春夏,不能秋冬,感而病生,足厥陰佗佗然。大角之人,比於左足少陽,少陽之上遺遺然,左角之人,比於右足少陽,少陽之下隨隨然。角之人,比於右足少陽,少陽之上推推然。判角之人,比於左足少陽,少陽之下栝栝然。火形之人,比於上徵,似於赤帝。其為人,赤色,廣,脫面,小頭,好肩背髀腹,小手足,行安地,疾心,行搖,肩背肉滿,有氣,輕財,少信,多慮,見事明,好顏,急心,不壽暴死。能春夏,不能秋冬,秋冬感而病生手少陰,核核然。質徵之人,比於左手太陽,太陽之上肌肌然。少徵之人,比於右手太陽,太陽之下,然。右徵之人,比於右手太陽,太陽之上,鮫鮫然。質判之人,比於左手太陽,太陽之下,支支頤頤然。土形之人,比於上宮,似於上古黃帝。其為人,黃色,員面,大頭,美肩背,大腹,美股脛,小手足,多肉,上下相稱,行安地,舉足浮安,心好利人,不喜權勢,善附人也。能秋冬,不能春夏,春夏感而病生,足太陰惇惇然。大宮之人,比於左足陽明,陽明之上婉婉然。加宮之人,比於左足陽明,陽明之下坎坎然,少宮之人,比於右足陽明,陽明之上樞樞然。左宮之人,比於右足陽明,陽明之下兀兀然。金形之人,比於上商,似於白帝。其為人,方面,白色,小頭,小肩背,小腹,小手足,如骨發踵外,骨輕,身清廉,急心靜悍,善為吏,能秋冬,不能春夏,春夏感而病生,手太陰惇惇然。商之人,比於左手陽明,陽明之上廉廉然。右商之人,比於左手陽明,陽明之下脫脫然。左商之人,比於右手陽明,陽明之上監監然。少商之人,比於右手陽明,陽明之下嚴嚴然。水形之人,比上羽,似於黑帝。其為人,黑色面不平,大頭廉頤,小肩,大腹,動手足,發行搖身,下尻長背,延延然,不敬畏,善欺給人戮死,能秋冬,不能春夏,春夏感而病,生足少陰,汗汗然。大羽之人,比於右足太陽,太陽之上,頰頰然。少羽之人,比於左足太陽,太陽之下,紆紆然。眾之為人,比於右足太陽,太陽之下,然。桎之為人,比於左足太陽,太陽之上,安安然。是故五形之人二十五變者,眾之所以相欺者是也。
黃帝曰:得其形,不得其色,何如?岐伯曰:形勝色,色勝形者,至其勝時年加,感則病行,失則憂矣。形色相得者,富貴大樂。黃帝曰:其形色相勝之時,年加可知乎?岐伯曰:凡年忌下上之人,大忌常加七歲,十六歲,二十五歲,三十四歲,四十三歲,五十二歲,六十一歲,皆人之大忌,不可不自安也。感則病行,失則憂矣,當此之時,無為奸事,是謂年忌。
黃帝曰:夫子之言,脈之上下,血氣之候,以知形氣,奈何?岐伯曰:足陽明之上,血氣盛則髯美長,血少氣多則髯短,故氣少血多則髯少,血氣皆少則無髯,兩吻多畫。足陽明之下,血氣盛則下毛美長至胸。血多氣少則下毛美短至齊,行則善高舉足,足指少肉,足善寒,血少氣多,則肉而善瘃,血氣皆少,則無毛,有則稀枯瘁,善痿厥足痹。足少陽之上,氣血盛則通髯美長,血多氣少則通髯美短,血少氣多則少須,血氣皆少則無須,感於寒濕則善痹,骨痛,爪枯也。足少陽之下,血氣盛則脛毛美長,外踝肥,血多氣少則脛毛美短,外踝皮堅而厚,血少氣多則毛少,外踝皮薄而軟,血氣皆少則無毛,外踝瘦無肉。足太陽之上,血氣盛則美眉,眉有毫毛,血多氣少則惡眉,面多少理,血少氣多則面多肉,血氣和則美色。足太陽之下,血氣盛則跟肉滿,踵堅,氣少血多則踵跟空,血氣皆少則善轉筋踵下痛。第四節手陽明之上,血氣盛則髭美,血少氣多則髭惡。血氣皆少則無髭,手陽明之下,血氣盛則腋下毛美,手魚肉以溫,氣血皆少則手瘦以寒。手少陽之上,血氣盛則眉美以長,耳色美,血氣皆少則耳焦惡色。手少陽之下,血氣盛則手卷多肉以溫,血氣皆少則寒以瘦,氣少血多則瘦以多脈。第六節手太陽之上,血氣盛則口多須,面多肉以平,血氣皆少則面瘦惡色。手太陽之下,血氣盛則掌肉充滿,血氣皆少則掌瘦以寒。

黃帝曰:二十五人者,刺之有約乎?岐伯曰:美眉者,足太陽之脈,氣血多。惡眉者,血氣少。其肥而澤者,血氣有餘。肥而不澤者,氣有餘,血不足。瘦而無澤者,氣血俱不足。審察其形氣有餘不足而調之,可以知逆順矣。
黃帝曰:刺其諸陰陽,奈何?岐伯曰:按其寸口人迎,以調陰陽,切循其經絡之凝澀結而不通者,此於身皆為痛痹,甚則不行,故凝澀。凝澀者,致氣以溫之,血和乃止。其結絡者,脈結血不行,決之乃行,故曰:氣有餘於上者,導而下之,氣不足於上者,推而休之,其稽留不至者,因而迎之,必明於經隧,乃能持之。寒與熱爭者,導而行之,其宛陳血不結者,則而予之。必先明知二十五人,則血氣之所在,左右上下,刺約畢也。

\section{五音五味第六十五}

右徵與少徵,調右手太陽上。左商與左徵,調左手陽明上。少徵與太宮,調左手陽明上。右角與太角,調右足少陽下。太徵與少徵,調左手太陽上。眾羽與少羽,調右足太陽下。少商與右商,調右手太陽下。桎羽與眾羽,調右足太陽下。少宮與太宮,調右足陽明下。判角與少角,調右足少陽下。商與上商,調右足陽明下。商與上角,調左足太陽下。
上徵與右徵同,穀麥,畜羊,果杏。手少陰藏心,色赤,味苦,時夏。上羽與太羽同,谷大豆,畜彘,果栗。足少陰藏腎,色黑,味咸,時冬。上宮與太宮同,谷稷,畜牛,果棗。足太陰藏脾,色黃,味甘,時季夏。上商與右商同,谷黍,畜雞,果桃。手太陰藏肺,色白味辛,時秋。上角與太角同。谷麻,畜犬,果李。足厥陰藏肝,色青,味酸,時春。
太宮與上角同,右足陽明上。左角與太角同,左足陽明上。少羽與太羽同,右足太陽下。左商與右商同,左手陽明下。加宮與太宮同,左足少陽上。質判與太宮同,左手太陽下。判角與太角同,左足少陽下。太羽與太角同,右足太陽上。太角與太宮同,右足少陽上。
右徵,少徵,質徵,上徵,判徵。右角,角,上角,太角,判角。右商,少商,商,上商,左商。少宮,上宮,太宮,加宮,左宮。眾羽,桎羽,上羽,太羽,少羽。

黃帝曰:婦人無須者,無血氣乎?岐伯曰:衝脈任脈,皆起於胞中,上循背裡,為經絡之海。其浮而外者,循腹右,上行會於咽喉,別而絡唇口,血氣盛則充膚熱肉,血獨盛則澹滲皮膚,生毫毛。今婦人之生,有餘於氣,不足於血,以其數脫血也。沖任之脈,不滎口,故須不生焉。黃帝曰:士人有傷於陰,陰氣絕而不起,陰不用,然其須不去,其故何也?宦者獨去,何也,願聞其故。岐伯曰:宦者去其宗筋,傷其衝脈,血瀉不復,皮膚內結,口不滎,故須不生。黃帝曰:其有天宦者,未嘗被傷,不脫於血,然其須不生,其故何也?岐伯曰:此天之所不足也,其任沖不盛,宗筋不成,有氣無血,口不滎,故須不生。黃帝曰:善乎哉,聖人之通萬物也,蓋日月之光影,音聲鼓響,聞其聲而知其形,其非夫子,孰能明萬物之精。
是故聖人視其顏色,黃赤者,多熱氣,青白者,少熱氣,黑色者,多血少氣。美眉者,太陽多血,通髯極須者,少陽多血,美眉者,陽明多血,此其時然也。夫人之常數,太陽常多血少氣,少陽常多氣少血,陽明常多血多氣,厥陰常多氣少血,少陰常多氣少血,太陰常多血少氣,此天之常數也。


\section{百病始生第六十六}

黃帝問於岐伯曰:夫百病之始生也,皆生於風雨寒暑,清濕喜怒。喜怒不節則傷藏,風雨則傷上,清濕則傷下三部之氣,所傷異類,願聞其會。岐伯曰:三部之氣各不同,或起於陰,或起於陽,請言其方。喜怒不節,則傷藏,藏傷則病起於陰也。清濕襲虛,則病起於下,風雨襲虛,則病起於上,是謂三部。至於其淫不可勝數。黃帝曰:余固不能數,故問先師,願卒聞其道。岐伯曰:風雨寒熱,不得虛邪,不能獨傷人,卒然逢疾風暴雨而不病者,蓋無虛,故邪不能獨傷人,此必因虛邪之風,與其身形,兩虛相得,乃客其形。兩實相逢,眾人肉堅。其中於虛邪也,因於天時,與其身形,參以虛實,大病乃成。氣有定舍,因處為名,上下中外,分為三員。

是故虛邪之中人也,始於皮膚,皮膚緩則腠理開,開則邪從毛髮入,入則抵深,深則毛髮立,毛髮立則淅然,故皮膚痛。留而不去,則傳舍於絡脈,在絡之時,痛於肌肉,其痛之時息,大經乃代。留而不去,傳舍於經,在經之時,灑淅喜驚。留而不去,傳舍於輸,在輸之時,六氣不通四支,則支節痛,腰脊乃強。留而不去,傳舍於伏沖之脈,在伏沖之時,體重身痛。留而不去,傳舍於腸胃,在腸胃之時,賁響腹脹,脹多寒則腸鳴飧洩,食不化,多熱則溏出麋。留而不出,傳舍於腸胃之外,募原之間,留著於脈,稽留而不去,息而成積,或著孫脈,或著絡脈,或著經脈,或著輸脈,或著於伏沖之脈,或著於膂筋,或著於腸胃之募原,上連於緩筋,邪氣淫,不可勝論。

黃帝曰:願盡聞其所由然。岐伯曰:其著孫絡之脈而成積者,其積往來上下臂手孫絡之居也,浮而緩,不能句積而止之,故往來移行腸胃之間,水湊滲注灌,濯濯有音,有寒則滿雷引,故時切痛。其著於陽明之經,則挾齊而居,飽食則益大,則益小。其著於緩筋也,似陽明之積,飽食則痛,則安。其著於腸胃之募原也,痛而外連於緩筋,飽食則安,則痛。其著於伏沖之脈者,揣之應手而動,發手則熱氣下於兩股如湯沃之狀。其著於膂筋,在腸後者,則積見,飽則積不見,按之不得。其著於輸之脈者,閉塞不通,津液不下,孔竅干塞,此邪氣之從外入內從上下也。

黃帝曰:積之始生,至其已成,奈何?岐伯曰:積之始生,得寒乃生,厥乃成積也。黃帝曰:其成積奈何?岐伯曰:厥氣生足,生脛寒,脛寒則血脈凝澀,血脈凝澀則寒氣上入於腸胃,入於腸胃則脹,脹則腸外之汁沫迫聚不得散,日以成積。卒然多食飲,則腸滿,起居不節,用力過度,則絡脈傷,陽絡傷則血外溢。血外溢則血,陰絡傷則血內溢,血內溢則後血,腸胃之絡傷,則血溢於腸外,腸外有寒,汁沫與血相搏,則併合凝聚不得散,而積成矣。卒然外中於寒,若內傷於憂怒,則氣上逆,氣上逆則六輸不通,溫氣不行,凝血蘊裹而不散,津液澀滲,著而不去,而積皆成矣。黃帝曰:其生於陰者,奈何?岐伯曰:憂思傷心,重寒傷肺,忿怒傷肝,醉以入房,汗出當風傷脾,用力過度,若入房汗出浴,則傷腎,此內外三部之所生病者也。
黃帝曰:善。治之奈何?岐伯答曰:察其所痛,以知其應,有餘不足,當補則補,當瀉則瀉,無逆天時,是謂至治。



\section{行針第六十七}

黃帝問於岐伯曰:余聞九針於夫子,而行之於百姓,百姓之血氣,各不同形,或神動而氣先針行,或氣與針相逢,或針已出,氣獨行,或數刺乃知,或髮針而氣逆,或數刺病益劇,凡此六者,各不同形,願聞其方,
岐伯曰:重陽之人其神易動,其氣易往也。黃帝曰:何謂重陽之人。岐伯曰:重陽之人,高高,言語善疾,舉足善高,心肺之藏氣有餘,陽氣滑盛而揚,故神動而氣先行。
黃帝曰:重陽之人而神不先行者,何也?岐伯曰:此人頗有陰者也。黃帝曰:何以知其頗有陰也。岐伯曰:多陽者,多喜,多陰者,多怒,數怒者,易解。故曰頗有陰,其陰陽之離合難,故其神不能先行也。
黃帝曰:其氣與針相逢,奈何?岐伯曰:陰陽和調,而血氣淖澤滑利,故針入而氣出疾而相逢也。
黃帝曰:針已出而氣獨行者,何氣使然。岐伯曰:其陰氣多而陽氣少,陰氣沉而陽氣浮者內藏,故針已出,氣乃隨其後,故獨行也。
黃帝曰:數刺乃知。何氣使然。岐伯曰:此人之多陰而少陽,其氣沉而氣往難,故數刺乃知也。
黃帝曰:針入而氣逆者,何氣使然。岐伯曰:其氣逆與其數刺病益甚者,非陰陽之氣,浮沉之勢也,此皆之所敗,工之所失,其形氣無過焉。



\section{上膈第六十八}

黃帝曰:氣為上膈者,食飲入而還出,余已知之矣。蟲為下膈,下膈者,食時乃出,余未得其意,願卒聞之。岐伯曰:喜怒不適,食飲不節,寒溫不時,則寒汁流於腸中,流於腸中則蟲寒,蟲寒則積聚,守於下管,則腸胃充郭,衛氣不滎,邪氣居之,人食則蟲上食,蟲上食則下管虛,下管虛則邪氣勝之,積聚以留,留則癰成,癰成則下管約,其癰在管內者,即而痛深,其癰在外者,則癰外而痛浮,癰上皮熱。

黃帝曰:刺之奈何?岐伯曰:微按其癰,視氣所行,先淺刺其傍,稍內益深,還而刺之,無過三行,察其沉浮,以為深淺,已刺必熨,令熱入中,日使熱內,邪氣益衰,大癰乃潰。伍以參禁,以除其內,恬無為,乃能行氣,後以咸苦,化谷乃下矣。



\section{憂恚無言第六十九}

黃帝問於少師曰:人之卒然憂恚,而言無音者,何道之塞,何氣出行,使音不彰,願聞其方。少師答曰:咽喉者,水谷之道也。喉嚨者,氣之所以上下者也。會厭者,音聲之戶也。口者,音聲之扇也。舌者,音聲之機也。懸雍垂者,音聲之關也。頏顙者,分氣之所洩也。橫骨者,神氣所使主發舌者也。故人之鼻洞涕出不收者,頏顙不開,分氣失也。是故厭小而疾薄,則發氣疾,其開闔利,其出氣易。其厭大而厚,則開闔難,其氣出遲,故重言也。人卒然無音者,寒氣客於厭,則厭不能發,發不能下至,其開闔不致,故無音。

黃帝曰:刺之奈何?岐伯曰:足之少陰,上繫於舌,絡於橫骨,終於會厭,兩瀉其血脈,濁氣乃辟,會厭之脈,上絡任脈,取之天突,其厭乃發也。



\section{寒熱第七十}

黃帝問於岐伯曰:寒熱瘰在於頸腋者,皆何氣使生。岐伯曰:此皆鼠寒熱之毒氣也,留於脈而不去者也。黃帝曰:去之奈何?岐伯曰:鼠之本,皆在於藏,其末上出於頸腋之間,其浮於脈中,而未內著於肌肉,而外為膿血者,易去也。黃帝曰:去之奈何?岐伯曰:請從其本引其末,可使衰去,而絕其寒熱。審按其道以予之,徐往徐來以去之,其小如麥者,一刺知,三刺而已。黃帝曰:決其生死奈何?岐伯曰:反其目視之,其中有赤脈,上下貫瞳子,見一脈,一歲死,見一脈半,一歲半死,見二脈,二歲死,見二脈半,二歲半死,見三脈,三歲而死,見赤脈不下貫瞳子,可治也。



\section{邪客第七十一}

黃帝問於伯高曰:夫邪氣之客人也,或令人目不瞑不臥出者,何氣使然。伯高曰:五穀入於胃也,其糟粕津液宗氣,分為三隧,故宗氣積於胸中,出於喉嚨,以貫心脈,而行呼吸焉。滎氣者,泌其津液,注之於脈,化以為血,以態四末,內注五藏六府,以應刻數焉。衛氣者,出其悍氣之疾,而先行於四末分肉皮膚之間,而不休者也,晝日行於陽,夜行於陰,常從足少陰之分間,行於五藏六府。今厥氣客於五藏六府,則衛氣獨衛其外,行於陽,不得入於陰,行於陽則陽氣盛,陽氣盛則陽陷,不得入於陰,陰虛,故目不瞑。黃帝曰:善。治之奈何?伯高曰:補其不足,瀉其有餘,調其虛實,以通其道,而去其邪,飲以半夏湯一劑,陰陽已通,其臥立至。黃帝曰:善。此所謂決瀆壅塞,經絡大通,陰陽和得者也,願聞其方。伯高曰:其湯方以流水千里以外者八升,揚之萬遍,取其清五升煮之,炊以葦薪火沸置秫米一升,治半夏五合徐炊,令竭為一升半,去其滓,飲汁一小杯,日三稍益,以知為度。故其病新發者,覆杯則臥,汗出則已矣。久者,三飲而已也。

黃帝問於伯高曰:願聞人之支節,以應天地奈何?伯高答曰:天員地方,人頭員足方,以應之。天有日月,人有兩目。地有九州,人有九竅。天有風雨,人有喜怒。天有雷電,人有音聲。天有四時,人有四支。天有五音,人有五藏。天有六律,人有六府。天有冬夏,人有寒熱。天有十日,人有手十指。辰有十二,人有足十指莖垂以應之,女子不足二節,以抱人形。天有陰陽,人有夫妻。歲有三百六十五日,人有三百六十節。地有高山,人有肩膝。地有深谷,人有腋。地有十二經水,人有十二經脈。地有泉脈,人有衛氣。地有草,人有毫毛。天有晝夜。人有臥起。天有列星,人有牙齒。地有小山,人有小節。地有山石,人有高骨。地有林木,人有募筋。地有聚邑,人有肉。歲有十二月,人有十二節。地有四時不生草,人有無子。此人與天地相應者也。

黃帝問於岐伯曰:余願聞持針之數,內針之理,縱舍之意,皮開腠理,奈何?脈之屈折,出入之處,焉至而出,焉至而止,焉至而徐,焉至而疾,焉至而入。六府之輸於身者,余願盡聞,少序別離之處,離而入陰,別而入陽,此何道而從行,願盡聞其方。岐伯曰:帝之所問,針道畢矣。黃帝曰:願卒聞之。岐伯曰:手太陰之脈,出於大指之端,內屈,循白肉際。至本節之後太淵,留以澹,外屈,上於本節下,內屈,與陰諸絡會於魚際,數脈並注,其氣滑利,伏行壅骨之下,外屈,出於寸口而行,上至於肘內廉,入於大筋之下,內屈,上行陰,入腋下,內屈,走肺。此順行逆數之曲折也。心主之脈,出於中指之端,內屈,循中指內廉以上,留於掌中,伏行兩骨之間,外屈,出兩筋之間,骨肉之際,其氣滑利,上二寸,外屈,出行兩筋之間,上至肘內廉,入於小筋之下,留兩骨之會,上于于胸中,內絡於心脈。黃帝曰:手少陰之脈,獨無俞,何也?岐伯曰:少陰,心脈也。心者,五藏六府之大主也,精神之所舍也,其藏堅固,邪弗能容也,容之則心傷,心傷則神去,神去則死矣。故諸邪之在於心者,皆在於心之包絡。包絡者,心主之脈也,故獨無俞焉。黃帝曰:少陰獨無俞者,不病乎?岐伯曰:其外經病而藏不病,故獨取其經於掌後銳骨之端,其餘脈出入屈折,其行之徐疾,皆如手少陰心主之脈行也,故本俞者,皆其因氣之虛實疾徐以取之,是謂因沖而瀉,因衰而補,如是者,邪氣得去,真氣堅固,是謂因天之序。
黃帝曰:持針縱舍奈何?岐伯曰:必先明知十二經脈之本末,皮膚之寒熱,脈之盛衰滑澀,其脈滑而盛者,病日進,虛而細者,久以持,大以澀者,為痛痹,陰陽如一者,病難治,其本末尚熱者,病尚在,其熱已衰者,其病亦去矣。持其尺,察其肉之堅脆,大小滑澀,寒溫燥濕,因視目之五色,以知五藏,而決死生,視其血脈,察其色,以知其寒熱痛。黃帝曰:持針縱舍,余未得其意也。岐伯曰:持針之道,欲端以正,安以靜,先知虛實,而行疾徐,左手執骨,右手循之,無與肉果,瀉欲端以正,補必閉膚,輔針導氣,邪得淫,真氣得居。黃帝曰:皮開腠理奈何?岐伯曰:因其分肉,左別其膚,微內而徐端之,視神不散,邪氣得去。

黃帝問於岐伯曰:人有八虛,各何以候。岐伯答曰:以候五藏。黃帝曰:候之奈何?岐伯曰:肺心有邪,其氣留於兩肘。肝有邪,其氣流於兩腋。脾有邪,其氣留於兩髀。腎有邪,其氣留於兩。凡此八虛者,皆機關之室,真氣之所過,血絡之所游,邪氣惡血,固不得住留,住留則傷筋絡骨節,機關不得屈伸,故病攣也。



\section{通天第七十二}

黃帝問於少師曰:余嘗聞人有陰陽,何謂陰人,何謂陽人。少師曰:天地之間,六合之內,不離於五,人亦應之,非徒一陰一陽而已也,而略言耳,口弗能明也。黃帝曰:願略聞其意,有賢人聖人,心能備而行之乎?少師曰:蓋有太陰之人,少陰之人,太陽之人,少陽之人,陰陽和平之人。凡五人者,其態不同,其筋骨氣血各不等。

黃帝曰:其不等者,可得聞乎?少師曰:太陰之人,貪而不仁,下齊湛湛,好內而惡出,心和而不發,不務於時,動而後之,此太陰之人也。少陰之人,小貪而賊心,見人有亡,常若有得,好傷好害,見人有滎,乃反慍怒,心疾而無恩,此少陰之人也。太陽之人,居處于于,好言大事,無能而虛說,志發於四野。舉措不顧是非,為事如常自用,事雖敗,而常無悔,此太陽之人也。少陽之人,諦好自貴,有小小官,則高自宜,好為外交,而不內附,此少陽之人也。陰陽和平之人,居處安靜,無為懼懼,無為欣欣,婉然從物,或與不爭,與時變化,尊則謙謙,譚而不治,是謂至治。

古之善用針艾者,視人五態,乃治之,盛者瀉之,虛者補之。黃帝曰:治人之五態奈何?少師曰:太陰之人,多陰而無陽,其陰血濁,其衛氣澀,陰陽不和,緩筋而厚皮,不之疾瀉,不能移之。少陰之人,多陰少陽,小胃而大腸,六府不調,其陽明脈小,而太陽脈大,必審調之,其血易脫,其氣易敗也。太陽之人,多陽而少陰,必謹調之,無脫其陰,而瀉其陽,陽重脫者,易狂,陰陽皆脫者,暴死不知人也。少陽之人,多陽少陰,經小而脈大,血在中而氣外,實陰而虛陽,獨瀉其絡脈,則強氣脫而疾,中氣不足,病不起也。陰陽和平之人,其陰陽之氣和,血脈調,謹診其陰陽,視其邪正,安容儀,審有餘不足,盛則瀉之,虛則補之,不盛不虛,以經取之。此所以調陰陽,別五態之人者也。

黃帝曰:夫五態之人者,相與無故,卒然新會,未知其行也,何以別之。少師答曰:眾人之屬,不如五態之人者,故五五二十五人,而五態之人不與焉。五態之人,尤不合於眾者也。黃帝曰:別五態之人奈何?少師曰:太陰之人,其狀然黑色,念然下意,臨臨然長大,然未僂,此太陰之人也。少陰之人,其狀清然竊然,固以陰賊,立而躁,行而似伏,此少陰之人也。太陽之人,其狀軒軒儲儲,反身折,此太陽之人也。少陽之人,其狀立則好仰,行則好搖,其兩臂兩肘,則常出於背,此少陽之人也。陰陽和平之人,其狀委委然,隨隨然,然,愉愉然,然,豆豆然,眾人皆曰君子,此陰陽和平之人也。

\section{官能第七十三}

黃帝問於岐伯曰:余聞九針於夫子,眾多矣,不可勝數。余推而論之,以為一紀,余司誦之,子聽其理,非則語余,請其正道,令可久傳後世無患,得其人乃傳,非其人勿言。岐伯稽首再拜曰:請聽聖王之道。黃帝曰:用針之理,必知形氣之所在,左右上下,陰陽表裡,血氣多少,行之逆順,出入之合。謀伐有過。知解結,知補虛瀉實,上下氣門,明通於四海,審其所在,寒熱淋露,以輸異處,審於調氣,明於經隧,左右支絡,盡知其會。寒與熱爭,能合而調之,虛與實鄰,知決而通之,左右不調,把而行之,明於逆順,乃知可治。陰陽不奇,故知起時,害於本末,察其寒熱,得邪所在,萬刺不殆,知官九針,刺道畢矣。

明於五輸徐疾所在,屈伸出入,皆有條理。言陰與陽,合於五行,五藏六府,亦有所藏,四時八風,盡有陰陽,各得其位,合於明堂,各處色部,五藏六府,察其所痛,左右上下,知其寒溫,何經所在。審皮膚之寒溫滑澀,知其所苦,膈有上下,知其氣所在,先得其道,稀而疏之,稍深以留,故能徐入之。大熱在上,推而下之,從下上者,引而去之。視前痛者,常先取之。大寒在外,留而補之。入於中者,從合瀉之。針所不為,炙之所宜。上氣不足,推而揚之。下氣不足,積而從之。陰陽皆虛,火自當之。厥而寒甚,骨廉陷下,寒過於膝,下陵三里。陰絡所過,得之留止。寒入於中,推而行之。經陷下者,火則當之。結絡堅緊,火所治之。不知所苦,兩之下,男陰女陽,良工所禁,針論畢矣。

用針之法,必有法則,上視天光,下司八正,以辟奇邪,而觀百姓,審於虛實,無犯其邪,是得天之露,遇歲之虛,救而不勝,反受其殃。故曰必知天忌,乃言針意,法於往古,驗於來今,觀於窈冥,通於無窮,之所不見,良工之所貴,莫知其形,若神。邪氣之中人也,灑淅動形,正邪之中人也微,先見於色,不知於其身,若有若亡,若亡若存,有形無形,莫知其情。是故上工之取氣,乃救其萌芽,下工守其已成,因敗其形。是故工之用針也,知氣之所在,而守其明戶,明於調氣,補瀉所在,徐疾之意,所取之處。瀉必用員,切而轉之,其氣乃行,疾而徐出,邪氣乃出,伸而迎之,搖大其穴,氣出乃疾。補必用方,外引其皮,令當其門,左引其樞,右推其膚,微旋而徐推之,必端以正,安以靜,堅心無解,欲微以留,氣下而疾出之,推其皮,蓋其外門,真氣乃存,用針之要,無忘其神。

雷公問於黃帝曰:針論曰,得其人乃傳,非其人勿言,何以知其可傳。黃帝曰:各得其人。任之其能,故能明其事。雷公曰:願聞官能奈何?黃帝曰:明目者,可使視色,聰耳者,可使聽音。捷疾辭語者,可使傳論。語徐而安靜,手巧而心審諦者,可使行針艾,理血氣而調諸逆順,察陰陽而兼諸方。緩節柔筋而心和調者,可使導引行氣。疾毒言語輕人者,可使唾癰病。爪苦手毒,為事善傷者,可使按積抑痹,各得其能,方乃可行,其名乃彰。不得其人,其功不成,其師無名。故曰:得其人乃言,非其人勿傳,此之謂也。手毒者,可使試按龜,置龜於器下,而按其上,五十日而死矣。手甘者,復生如故也。



\section{論疾診尺第七十四}

黃帝問於岐伯曰:余欲無視色持脈,獨調其尺,以言其病,從外知內,為之奈何?岐伯曰:審其尺之緩急小大滑澀,肉之堅脆,而病形定矣。視人之目窠上,微癰如新臥起狀,其頸脈動,時,按其手足上,而不起者,風水膚脹也。尺膚滑,其淖澤者,風也。尺肉弱者,解,安臥脫肉者,寒熱,不治。尺膚滑而澤脂者,風也。尺膚澀者,風痹也。尺膚如枯魚之鱗者,水飲也。尺膚熱甚,脈盛躁者,病溫也。其脈盛而滑者,病且出也。尺膚寒,其脈小者,洩,少氣。尺膚炬然先熱後寒者,寒熱也。尺膚先寒,久大之而熱者,亦寒熱也。肘所獨熱者,腰以上熱。手所獨熱者,腰以下熱。肘前獨熱者,膺前熱。肘後獨熱者,肩背熱。臂中獨熱者,腰腹熱。肘後以下三四寸熱者,腸中有蟲。掌中熱者,腹中熱。掌中寒者,腹中寒。魚上白肉有青血脈者,胃中有寒。尺炬然熱,人迎大者,當奪血。尺堅大,脈小甚,少氣有加,立死。

目赤色者病在心,白在肺,青在肝,黃在脾,黑在腎,黃色不可名者病在胸中。診目痛,赤脈從上下者太陽病,從下上者陽明病,從外走內者少陽病。診寒熱,赤脈上下至瞳子,見一脈,一歲死,見一脈半,一歲半死,見二脈,二歲死,見二脈半,二歲半死,見三脈,三歲死。診齲齒痛,按其陽之來,有過者獨熱,在左左熱,在右右熱,在上上熱,在下下熱。
診血脈者,多赤多熱,多青多痛,多黑為久痹。多赤多黑多青皆見者,寒熱身痛,而色微黃,齒垢黃,爪甲上黃,黃疸也。安臥小便黃赤,脈小而澀者,不嗜食。
人病其寸口之脈與人迎之脈小大等,及其浮沉等者,病難已也。女子手少陰脈動甚者妊子。嬰兒病,其頭毛皆逆上者必死。耳間青脈起者掣痛,大便赤硬飧洩,脈小者手足寒難已。飧洩,脈小,手足溫,洩易已。

四時之變,寒暑之勝,重陰必陽,重陽必陰,故陰主寒,陽主熱。故寒甚則熱,熱甚則寒。故曰:寒生熱,熱生寒,此陰陽之變也。故曰:冬傷於寒,春生癉熱。春傷於風,夏生飧洩腸。夏傷於暑,秋生瘧。秋傷於濕,冬生嗽。是謂四時之序也。



\section{刺節真邪第七十五}

黃帝問於岐伯曰:余聞刺有五節,奈何?岐伯曰:固有五節,一曰振埃,二曰發蒙,三曰去爪,四曰徹衣,五曰解惑。黃帝曰:夫子言五節,余未知其意。岐伯曰:振埃者,刺外,去陽病也。發蒙者,刺府輸,去府病也。去爪者,刺關節支絡也。徹衣者,盡刺諸陽之奇輸也。解惑者,盡知調陰陽,補瀉有餘不足,相傾移也。
黃帝曰:刺節言振埃,夫子乃言刺外經,去陽病,余不知其所謂也,願卒聞之。岐伯曰:振埃者,陽氣大逆,上滿於胸中,憤肩息,大氣逆上,喘喝坐伏,病惡埃煙,不得息,請言振埃,尚疾於振埃。黃帝曰:善。取之何如?岐伯曰:取之天容。黃帝曰:其上氣,窮胸痛者,取之奈何?岐伯曰:取之廉泉。黃帝曰:取之有數乎?岐伯曰:取天容者,無過一里。取廉泉者,血變而止。黃帝曰:善哉。
黃帝曰:刺節言發蒙,余不得其意,夫發蒙者,耳無所聞,目無所見,夫子乃言刺府輸,去府病,何輸使然,願聞其故。岐伯曰:妙乎哉問也,此刺之大約,針之極也,神明之類也,口說書卷,猶不能及也,請言發蒙耳,尚疾於發蒙也。黃帝曰:善。願卒聞之。岐伯曰:刺此者,必於日中,刺其聽宮,中其眸子,聲聞於耳,此其輸也。黃帝曰:善。何謂聲聞於耳。岐伯曰:刺邪以手堅按其兩鼻竅而疾偃,其聲必應於針也。黃帝曰:善。此所謂弗見為之,而無目視,見而取之,神明相得者也。
黃帝曰:刺節言去爪,夫子乃言刺關節支絡,願卒聞之。岐伯曰:腰脊者,身之大關節也。支脛者,人之管以趨翔也。莖垂者,身中之機,陰精之候,津液之道也。故飲食不節,喜怒不時,津液內溢,乃下留於睾,血道不通,日大不休,仰不便,趨翔不能,此病滎然有水,不上不下,鈹石所取,形不可匿,常不得蔽,故命曰去爪。帝曰:善。
黃帝曰:刺節言徹衣,夫子乃言盡刺諸陽之奇輸,未有常處也,願卒聞之。岐伯曰:是陽氣有餘,而陰氣不足,陰氣不足則內熱,陽氣有餘則外熱,內熱相搏,熱於懷炭,外畏綿帛近,不可近身,又不可近席,腠理閉塞,則汗不出,舌焦唇槁,干嗌燥,飲食不讓美惡。黃帝曰:善。取之奈何?岐伯曰:取之於其天府大杼三,又刺中膂,以去其熱,補足手太陰,以去其汗,熱去汗稀,疾於徹衣。黃帝曰:善。
黃帝曰:刺節言解惑,夫子乃言盡知調陰陽,補瀉有餘不足,相傾移也,惑何以解之。岐伯曰:大風在身,血脈偏虛,虛者不足,實者有餘,輕重不得,傾側宛伏,不知東西,不知南北,乍上乍下,乍反乍覆,顛倒無常,甚於迷惑。黃帝曰:善。取之奈何?岐伯曰:瀉其有餘,補其不足,陰陽平復,用針若此,疾於解惑。黃帝曰:善。請藏之靈蘭之室,不敢妄出也。

黃帝曰:余聞刺有五邪,何謂五邪。岐伯曰:病有持癰者,有容大者,有狹小者,有熱者,有寒者,是謂五邪。黃帝曰:刺五邪奈何?岐伯曰:凡刺五邪之方,不過五章,癉熱消滅,腫聚散亡,寒痹益溫。小者益陽,大者必去,請道其方。凡刺癰邪,無迎隴,易俗移性,不得膿,脆道更行,去其鄉,不安處所乃散亡,諸陰陽過癰者,取之其輸。瀉之。凡刺大邪,日以小,洩奪其有餘,乃益虛,剽其通,針其邪,肌肉親,視之無有反其真,刺諸陽分肉間。凡刺小邪,日以大,補其不足,乃無害,視其所在,迎之界,遠近盡至,其不得外,侵而行之,乃自費,刺分肉間。凡刺熱邪,越而蒼,出遊不歸,乃無病,為開通,闢門戶,使邪得出,病乃已。凡刺寒邪,日以溫,徐往徐來,致其神,門戶已閉,氣不分,虛實得調,其氣存也。
黃帝曰:官針奈何?岐伯曰:刺癰者,用鈹針。刺大者,用鋒針。刺小者,用員利針。刺熱者,用針。刺寒者,用毫針也。

請言解論,與天地相應,與四時相副,人參天地,故可為解,下有漸洳,上生葦蒲,此所以知形氣之多少也。陰陽者,寒暑也,熱則滋雨而在上,根少汁,人氣在外,皮膚緩腠理開,血氣減,汗大洩,皮淖澤。寒則地凍水冰,人氣在中,皮膚致,腠理閉,汗不出,血氣強,肉堅澀。當是之時,善行水者,不能往冰。善穿地者,不能鑿凍。善用針者,亦不能取四厥。血脈凝結,堅搏不往來者,亦未可即柔。故行水者,必待天溫,冰釋凍解,而水可行,地可穿也。人脈猶是也。治厥者,必先熨調和其經,掌與腋,肘與腳,項與脊以調之,火氣已通,血脈乃行,然後視其病,脈淖澤者,刺而平之,堅緊者,破而散之,氣下乃止,此所謂以解結者也。用針之類,在於調氣,氣積於胃,以通滎衛,各行其道,宗氣留於海,其下者,注於氣街,其上者,走於息道。故厥在於足,宗氣不下,脈中之血,凝而留止,弗之火調,弗能取之。用針者,必先察其經絡之實虛,切而循之,按而彈之,視其應動者,乃後取之而下之。六經調者,謂之不病,雖病,謂之自已也。一經上實下虛而不通者,此必有橫絡盛加於大經,令之不通,視而瀉之,此所謂解結也。上寒下熱,先刺其項太陽,久留之。已刺則熨項與肩胛,令熱下合乃止,此所謂推而上之者也。上熱下寒,視其虛脈而陷之於經絡者,取之,氣下乃止,此所謂引而下之者也。大熱身,狂而妄見妄聞妄言,視足陽明及大絡取之,虛者補之,血而實者瀉之,因其偃臥,居其頭前,以兩手四指挾按頸動脈,久持之,卷而切,推下至缺盆中,而復止如前,熱去乃止,此所謂推而散之者也。

黃帝曰:有一脈生數十病者,或痛,或癰,或熱,或寒,或癢,或痹,或不仁,變化無窮,其故何也?岐伯曰:此皆邪氣之所生也。黃帝曰:余聞氣者,有真氣,有正氣,有邪氣。何謂真氣。岐伯曰:真氣者,所受於天,與谷氣並而充身也。正氣者,正風也,從一方來,非實風,又非虛風也。邪氣者,虛風之賊傷人也,其中人也深,不能自去。正風者,其中人也淺,合而自去,其氣來柔弱,不能勝真氣,故自去。
虛邪之中人也,灑淅動形,起毫毛而發腠理。其入深,內搏於骨,則為骨痹。搏於筋,則為筋攣。搏於脈中,則為血閉不通,則為癰,搏於肉,與衛氣相搏,陽勝者,則為熱,陰勝者,則為寒,寒則真氣去,去則虛,虛則寒。搏於皮膚之間,其氣外發,腠理開,毫毛搖,氣往來行,則為癢,留而不去,則痹,衛氣不行,則為不仁。虛邪偏客於身半,其入深,內居滎衛,滎衛稍衰,則真氣去,邪氣獨留,發為偏枯。其邪氣淺者,脈偏痛。
虛邪之入於深也深,寒與熱相搏,久留而內著,寒勝其熱,則骨痛肉枯,熱勝其寒。則爛肉腐肌為膿,內傷骨,內傷骨為骨蝕。有所疾前筋,筋屈不得伸,邪氣居其間而不反,發於筋溜。有所結,氣歸之,衛氣留之,不得反,津液久留,合而為腸溜,久者,數歲乃成,以手按之柔。已有所結,氣歸之,津液留之,邪氣中之,凝結日以易甚,連以聚居,為昔瘤,以手按之堅。有所結,深中骨,氣因於骨,骨與氣並,日以益大,則為骨疽。有所結,中於肉,宗氣歸之,邪留而不去,有熱則化而為膿,無熱則為肉疽。凡此數氣者,其發無常處,而有常名也。



\section{衛氣行第七十六}

黃帝問於岐伯曰:願聞衛氣之行,出入之合,何如?岐伯曰:歲有十二月,日有十二辰,子午為經,卯酉為緯,天週二十八宿,而一面七星,四七二十八星,房昴為緯,虛張為經,是故房至畢為陽,昴至心為陰,陽主晝,陰主夜。故衛氣之行,一日一夜五十週於身,晝日行於陽二十五週,夜行於陰二十五週,周於五藏。是故平旦陰盡,陽氣出於目,目張則氣上行於頭,循項下足太陽,循背下至小指之端。其散者,別於目銳,下手太陽,下至手小指之間外側。其散者,別於目銳,下足少陽,注小指次指之間,以上循手少陽之分,側下至小指之間,別者以上至耳前,合於頷脈,注足陽明以下行,至跗上,入五指之間。其散者,從耳下下手陽明,入大指之間,入掌中。其至於足也,入足心,出內踝,下行陰分,復合於目,故為一週。

是故日行一舍,人氣行一週與十分身之八。日行二舍,人氣行三週於身與十分身之六。日行三舍,人氣行於身五週與十分身之四。日行四捨,人氣行於身七周與十分身之二。日行五舍,人氣行於身九周。日行六舍,人氣行於身十週與十分身之八。日行七舍,人氣行於身十二周在身與十分身之六。日行十四捨,人氣二十五週於身有奇分與十分身之二,陽盡於陰,陰受氣矣。

其始入於陰,常從足少陰注於腎,腎注於心,心注於肺,肺注於肝,肝注於脾,脾復注於腎為周。是故夜行一舍,人氣行於陰藏一週與十分藏之八,亦如陽行之二十五週,而復合於目。陰陽一日一夜,合有奇分十分身之四。與十分藏之二。是故人之所以臥起之時有早晏者,奇分不盡故也。
黃帝曰:衛氣之在於身也,上下往來不以期,候氣而刺之,奈何?伯高曰:分有多少,日有長短,春秋冬夏,各有分理,然後常以平旦為紀,以夜盡為始。是故一日一夜,水下百刻,二十五刻者,半日之度也,常如是無已。日入而止,隨日之長短,各以為紀而刺之,謹候其時,病可與期。失時反候者,百病不治。故曰:刺實者,刺其來也。刺虛者,刺其去也。此言氣存亡之時,以候虛實而刺之。是故謹候氣之所在而刺之,是謂逢時。在於三陽,必候其氣在於陽而刺之。病在於三陰,必候其氣在陰分而刺之。水下一刻,人氣在太陽。水下二刻,人氣在少陽。水下三刻,人氣在陽明。水下四刻,人氣在陰分。水下五刻,人氣在太陽。水下六刻,人氣在少陽。水下七刻,人氣在陽明。水下八刻,人氣在陰分。水下九刻,人氣在太陽。水下十刻,人氣在少陽。水下十一刻,人氣在陽明。水下十二刻,人氣在陰分。水下十三刻,人氣在太陽。水下十四刻,人氣在少陽。水下十五刻,人氣在陽明。水下十六刻,人氣在陰分。水下十七刻,人氣在太陽。水下十八刻,人氣在少陽。水下十九刻,人氣在陽明。水下二十刻,人氣在陰分。水下二十一刻,人氣在太陽。水下二十二刻,人氣在少陽。水下二十三刻,人氣在陽明。水下二十四刻,人氣在陰分。水下二十五刻,人氣在太陽。此半日之度也。從房至畢一十四捨水下五十刻,日行半度。回行一舍,水下三刻與七分刻之四。大要曰:常以日之加於宿上也,人氣在太陽,是故日行一舍,人氣行三陽行與陰分,常如是無已,天與地同紀,紛紛,終而復始,一日一夜,水下百刻而盡矣。



\section{九宮八風第七十七}

合八風虛實邪正(下為圖)
立秋(坤)玄委秋分(兌)倉果立冬(干)新洛夏至(離)上天招(中央)搖冬至(坎)葉蟄立夏(巽)陰洛春分(震)倉門立春(艮)天留立秋二(玄委西南方)秋分七(倉果西方)立冬六(新洛西北方)夏至九(上天南方)招搖中央冬至一(葉蟄北方)立夏四(陰洛東南方)春分三(倉門東方)立春八(天留東北方)太一常以冬至之日,居葉蟄之宮四十六日,明日居天留四十六日,明日居倉門四十六日,明日居陰洛四十五日,明日居天宮四十六日,明日居玄委四十六日,明日居倉果四十六日,明日居新洛四十五日,明日復居葉蟄之宮,曰冬至矣。太一日遊,以冬至之日,居葉蟄之宮,數所在日從一處,至九日,復反於一,常如是無已,終而復始。

太一移日天必應之以風雨,以其日風雨則吉,歲美民安少病矣。先之則多雨,後之則多旱。太一在冬至之日有變,佔在君。太一在春分之日有變,佔在相。太一在中宮之日有變,佔在吏。太一在秋分之日有變,佔在將。太一在夏至之日有變,佔在百姓。所謂有變者,太一居五宮之日,病風折樹木,揚沙石,各以其所主,佔貴賤。因視風所從來而佔之,風從其所居之鄉來為實風,主生,長養萬物。從其沖後來為虛風,傷人者也,主殺,主害者,謹候虛風而避之,故聖人日避虛邪之道,如避矢石然,邪弗能害,此之謂也。

是故太一入徙立於中宮,乃朝八風,以佔吉凶也。風從南方來,名曰大弱風,其傷人也,內舍於心,外在於脈,氣主熱。風從西南方來,名曰謀風,其傷人也,內舍於脾,外在於肌,其氣主為弱。風從西方來,名曰剛風,其傷人也,內舍於肺,外在於皮膚,其氣主為燥。風從西北方來,名曰折風,其傷人也,內舍於小腸,外在於手太陽脈,脈絕則溢,脈閉則結不通,善暴死。風從北方來,名曰大剛風,其傷人也,內舍於腎,外在於骨與肩背之膂筋,其氣主為寒也。風從東北方來,名曰凶風,其傷人也,內舍於大腸,外在於兩脅腋骨下及支節。風從東方來,名曰嬰兒風,其傷人也,內舍於肝,外在於筋紐,其氣主為身濕。風從東南方來,名曰弱風,其傷人也,內舍於胃,外在肌肉,其氣主體重。此八風皆從其虛之鄉來,乃能病人,三虛相搏,則為暴病卒死,兩實一虛,病則為淋露寒熱。犯其雨濕之地,則為痿。故聖人避風,如避矢石焉。其有三虛而偏中於邪風,則為擊僕偏枯矣。


\section{九針論第七十八}

黃帝曰:余聞九針於夫子,眾多博大矣,余猶不能寤,敢問九針焉生,何因而有名。岐伯曰:九針者,天地之大數也,始於一而終於九,故曰:一以法天,二以法地,三以法人,四以法時,五以法音,六以法律,七以法星,八以法風,九以法野。黃帝曰:以針應九之數,奈何?岐伯曰:夫聖人之起,天地之數也,一而九之,故以立九野,九而九之,九九八十一,以起黃鐘數焉,以針應數也。

一者,天也,天者,陽也,五藏之應天者肺,肺者,五藏六府之盡也,皮者,肺之合也,人之陽也,故為之治針,必以大其頭而銳其末,令無得深入而陽氣出。二者,地也,人之所以應土者,肉也,故為之治針,必其身而員其末,令無得傷肉分,傷則氣得竭。三者,人也,人之所以成生者,血脈也,故為之治針,必大其身而員其末,令可以按脈勿陷,以致其氣,令邪氣獨出。四者,時也,時者,四時八風之客於經絡之中,為瘤病者也,故為之治針,必其身而鋒其末,令可以瀉熱出血,而痼病竭。五者,音也,音者,冬夏之分,分於子午,陰與陽別,寒與熱爭,兩氣相搏,合為癰膿者也,故為之治針必令其末如劍鋒,可以取大膿。六者,律也,律者,調陰陽四時而合十二經脈,虛邪客於經絡而為暴痹者也,故為之治針,必令尖如,且員且銳,中身微大,以取暴氣。七者,星也,星者,人之七竅,邪之所客於經,而為痛痹,舍於經絡者也,故為之治針,令尖如蚊喙,靜以徐往,微以久留,正氣因之,真邪俱往,出針而養者也。八者,風也,風者,人之股肱八節也,八正之虛風,八風傷人,內舍於骨解腰脊節腠理之間,為深痹也,故為之治針,必長其身,鋒其末,可以取深邪遠痹。九者,野也,野者,人之節解皮膚之間也,淫邪流溢於身,如風水之狀,而溜不能過於機關大節者也,故為之治針,令尖如挺,其鋒微員,以取大氣之不能過於關節者也。

黃帝曰:針之長短有數乎?岐伯曰:一曰針者,取法於巾針,去末寸半,卒銳之,長一寸六分,主熱在頭身也。二曰員針,取法於絮針,其身而卯其鋒,長一寸六分,主治分肉間氣。三曰針,取法於黍粟之銳,長三寸半,主按脈取氣,令邪出。四曰鋒針,取法於絮針,其身鋒其末,長一寸六分,主癰熱出血。五曰鈹針,取法於劍鋒,廣二分半,長四寸,主大癰膿,兩熱爭者也。六曰員利針,取法於針微大其末,反小其身,令可深內也,長一寸六分,主取癰痹者也。七曰毫針,取法於毫毛,長一寸六分,主寒熱痛痹在絡者也。八曰長針,取法於綦針,長七寸,主取深邪遠痹者也。九曰大針,取法於鋒針,其鋒微員,長四寸,主取大氣不出關節者也,針形畢矣。此九針大小長短法也。

黃帝曰:願聞身形,應九野,奈何?岐伯曰:請言身形之應九野也,左足應立春,其日戊寅己丑。左脅應春分,其日乙卯。左手應立夏,其日戊辰己巳。膺喉首頭應夏至,其日丙午。右手應立秋,其日戊申己未。右脅應秋分,其日辛酉。右足應立冬,其日戊戌己亥。腰尻下竅應冬至,其日壬子。六府膈下三藏應中州,其大禁,大禁太一所在之日,及諸戊己。凡此九者,善候八正所在之處,所主左右上下,身體有癰腫者,欲治之,無以其所直之日潰治之,是謂天忌日也。

形樂志苦,病生於脈,治之以炙刺。形苦志樂,病生於筋,治之以熨引。形樂志樂,病生於肉,治之以針石。形苦志苦,病生於咽喝,治之以甘藥。形數驚恐,筋脈不通,病生於不仁,治之以按摩醪藥。是謂形。
五藏氣,心主噫,肺主,肝主語,脾主吞,腎主欠。六府氣,膽為怒,胃為氣逆噦,大腸小腸為洩,膀胱不約為遺溺,下焦溢為水。五味,酸入肝,辛入肺,苦入心,甘入脾,咸入腎,淡入胃,是謂五味。五並,精氣並肝則憂,並心則喜,並肺則憂,並腎則恐,並脾則畏,是謂五精之氣,並於藏也。五惡,肝惡風,心惡熱,肺惡寒,腎惡燥,脾惡濕,此五藏氣所惡也。五液,心主汗,肝主泣,肺主涕,腎主唾,脾主涎,此五液所出也。五勞,久視傷血,久臥傷氣,久坐傷肉,久立傷骨,久行傷筋,此五久勞所病也。五走,酸走筋,辛走氣,苦走血,咸走骨,甘走肉,是謂五走也。五裁,病在筋,無食酸,病在氣,無食辛,病在骨,無食咸,病在血,無食苦,病在肉,無食甘,口嗜而欲食之,不可多也,必自裁也,命曰五裁。五發,陰病發於骨,陽病發於血,陰病發於肉,陽病發於冬,陰病發於夏。五邪,邪入於陽,則為狂,邪入於陰,則為血痹,邪入於陽,轉則為癲疾,邪入於陰,轉則為,陽入之於陰,病靜,陰出之於陽,病喜怒。五藏,心藏神,肺藏魄,肝藏魂。脾藏意,腎藏精志也。五主,心主脈,肺主皮,肝主筋,脾主肌,腎主骨。
陽明多血多氣,太陽多血少氣,少陽多氣少血,太陰多血少氣,厥陰多血少氣,少陰多氣少血,故曰刺陽明出血氣,刺太陽出血惡氣,刺少陽出氣惡血,刺太陰出血惡氣,刺厥陰出血惡氣,刺少陰出氣惡血也。足陽明太陰為表裡,少陽厥陰為表裡,太陽少陰為表裡,是謂足之陰陽也。手陽明太陰為表裡,少陽心主為表裡,太陽少陰為表裡,是謂手之陰陽也。



\section{歲露論第七十九}

黃帝問於岐伯曰:經言夏日傷暑,秋病瘧,瘧之發以時,其故何也?岐伯對曰:邪客於風府,病循膂而下,衛氣一日一夜,常大會於風府,其明日日下一節,故其日作晏。此其先客於脊背也,故每至於風府則腠理開,腠理開則邪氣入,邪氣入則病作,此所以日作尚晏也。衛氣之行風府,日下一節,二十一日,下至尾底,二十二日,入脊內,注於伏沖之脈,其行九日,出於缺盆之中,其氣上行,故其病稍益,至其內搏於五藏,橫連募原,其道遠,其氣深,其行遲,不能日作,故次日乃畜積而作焉。黃帝曰:衛氣每至於風府,腠理乃發,發則邪入焉,其衛氣日下一節,則不當風府,奈何?岐伯曰:風府無常,衛氣之所應,必開其腠理,氣之所舍節,則其府也。黃帝曰:善。夫風之與瘧也,相與同類,而風常在,而瘧特以時休,何也?岐伯曰:風氣留其處,瘧氣隨經絡,沉以內搏,故衛氣應,乃作也。帝曰:善。

黃帝問於少師曰:余聞四時八風之中人也,故有寒暑,寒則皮膚急而腠理閉,暑則皮膚緩而腠理開,賊風邪氣因得以入乎,將必須八正虛邪,乃能傷人乎?少師答曰:不然,賊風邪氣之中人也,不得以時,然必因其開也,其入深,其內極病,其病人也,卒暴,因其閉也,其入淺以留,其病也,徐以遲。黃帝曰:有寒溫和適,腠理不開,然有卒病者,其故何也?少師答曰:帝弗知邪入乎?雖平居,其腠理開閉緩急,其故常有時也。黃帝曰:可得聞乎?少師曰:人與天地相參也,與日月相應也。故月滿則海水西盛,人血氣積,肌肉充,皮膚致,毛髮堅,腠理郗,煙垢著,當是之時,雖遇賊風,其入淺不深。至其月郭空,則海水東盛,人氣血虛,其衛氣去,形獨居,肌肉減,皮膚縱,腠理開,毛髮殘,理薄,煙垢落,當是之時,遇賊風,則其入深,其病人也,卒暴。黃帝曰:其有卒然暴死暴病者,何也?少師答曰:三虛者,其死暴疾也。得三實者邪不能傷人也。黃帝曰:願聞三虛。少師曰:乘年之衰,逢月之空,失時之和,因為賊風所傷,是謂三虛,故論不知三虛,工反為。帝曰:願聞三實。少師曰:逢年之盛,遇月之滿,得時之和,雖有賊風邪氣,不能危之也。黃帝曰:善乎哉論,明乎哉道,請藏之金匱,命曰三實,然此一夫之論也。

黃帝曰:願聞歲之所以皆同病者,何因而然。少師曰:此八正之候也。黃帝曰:候之奈何?少師曰:候此者,常以冬至之日,太一立於葉蟄之宮,其至也,天必應之以風雨者矣。風雨從南方來者,為虛風,賊傷人者也。其以夜半至也,萬民皆臥而弗犯也,故其歲民少病。其以晝至者,萬民懈惰而皆中於虛風,故萬民多病。虛邪入客於骨而不發於外,至其立春,陽氣大發,腠理開,因立春之日,風從西方來,萬民又皆中於虛風,此兩邪相搏,經氣結代者矣。故諸逢其風而遇其雨者,命曰遇歲露焉。因歲之和,而少賊風者,民少病而少死,歲多賊風邪氣,寒溫不和,則民多病而死矣。黃帝曰:虛邪之風,其所傷貴賤何如,候之奈何?少師答曰:正月朔日,太一居天留之宮,其日西北風不雨,人多死矣。正月朔日,平旦北風,春,民多死。正月朔日,平旦北風行,民病多者,十月三也。正月朔日,日中北風,夏,民多死。正月朔日,夕時北風,秋,民多死。終日北風,大病,死者十有六。正月朔日,風從南方來,命曰旱鄉,從西方來,命曰白骨,將國有殃,人多死亡。正月朔日,風從東方來,發屋,揚沙石,國有大災也。正月朔日,風從東南方來,春有死亡。正月朔日,天和溫,不風,糴賤民不病,天寒而風,糴貴民多病。此所謂候歲之風,殘傷人者也。二月丑,不風,民多心腹病。三月戍不溫,民多寒熱。四月已不暑,民多癉病。十月申不寒,民多暴死。諸所謂風者,皆發屋,折樹木,揚沙石,起毫毛,發腠理者也。


\section{大惑論第八十}

黃帝問於岐伯曰:余嘗上於清冷之台,中階而顧,匍匐而前,則惑,余私異之,竊內怪之,獨瞑獨視,安心定氣,久而不解,獨博獨眩,披髮長跪,而視之,後久之不已也。卒然自上,何氣使然。岐伯對曰:五藏六府之精氣,皆上於目,而為之精,精之窠為眼,骨之精瞳子,筋之精為黑眼,血之精為絡,其窠氣之精為白眼,肌肉之精為約,裹擷筋骨血氣之精而與脈並為系,上屬於腦,後出於項中,故邪中於項,因逢其身之虛,其入深,則隨眼系以入於腦,入於腦則腦轉,腦轉則引目系急,目系急則目眩以轉矣。邪其精,其精所中,不相比也,則精散,精散則視岐,視岐見兩物。目者,五藏六府之精也,滎衛魂魄之所常滎也,神氣之所生也,故神勞則魂魄散,志意亂,是故瞳子黑眼法於陰,白眼赤脈法於陽也。故陰陽合傳而精明也。目者,心使也,心者,神之舍也,故神精亂而不轉,卒然見非常處,精神魂魄,散不相得,故曰惑也。黃帝曰:余疑其然。余每之東苑,未嘗不惑,去之則復,余唯獨為東苑勞神乎,何其異也。岐伯曰:不然也。心有所喜,神有所惡,卒然相惑,則精氣亂,視誤,故惑,神移乃復,是故間者為迷,甚者為惑。
黃帝曰:人之善忘者,何氣使然。岐伯曰:上氣不足,下氣有餘,腸胃實而心肺虛,虛則滎衛留於下,久之不以時上,故善忘也。
黃帝曰:人之善飢而不嗜食者,何氣使然。岐伯曰:精氣並於脾,熱氣留於胃,胃熱則消谷,谷消故善,胃氣逆上,則胃脘寒,故不嗜食也。
黃帝曰:病而不得臥者,何氣使然。岐伯曰:衛氣不得入於陰,常留於陽,留於陽則陽氣滿,陽氣滿則陽盛,不得入於陰則陰氣虛,故目不瞑矣。
黃帝曰:病目而不得視者,何氣使然。岐伯曰:衛氣留於陰,不得行於陽,留於陰則陰氣盛,陰氣盛則陰滿,不得入於陽則陽氣虛,故目閉也。
黃帝曰:人之多臥者,何氣使然。岐伯曰:此人腸胃大而皮膚濕而分肉不解焉。腸胃大則衛氣留久,皮膚濕則分肉不解,其行遲。夫衛氣者,晝日常行於陽,夜行於陰,故陽氣盡則臥,陰氣盡則寤。故腸胃大,則衛氣行留久,皮膚涇,分肉不解,則行遲,留於陰也久,其氣不清,則欲瞑,故多臥矣。其腸胃小,皮膚滑以緩,分肉解利,衛氣之留於陽也久,故少瞑焉。黃帝曰:其非常經也,卒然多臥者。何氣使然。岐伯曰:邪氣留於上焦,上焦閉而不通,已食若飲湯,衛氣留久於陰而不行,故卒然多臥焉。
黃帝曰:善。治此諸邪,奈何?岐伯曰:先其藏府,誅其小過,後調其氣,盛者瀉之,虛者補之,必先明知其形志之苦樂,定乃取之。

\section{癰疽第八十一}

黃帝曰:余聞腸胃受谷,上焦出氣,以溫分肉,而養骨節,通腠理。中焦出氣如露,上注溪谷,而滲孫脈,津液和調,變化而赤為血,血和則孫脈先滿,溢乃注於絡脈,皆盈,乃注於經脈。陰陽已張,因息乃行,行有經紀,周有道理,與天合同,不得休止。切而調之,從虛去實,瀉則不足,疾則氣減,留則先後,從虛去虛,補則有餘,血氣已調,形氣乃持。余已知血氣之平與不平,未知癰疽之所從生,成敗之時,死生之期,有遠近何以度之,可得聞乎?岐伯曰:經脈留行不止,與天同度,與地合紀。故天宿失度,日月薄蝕,地經失紀,水道流溢,草不成,五穀不殖,徑路不通,民不往來,巷聚邑居,則別離異處,血氣猶然,請言其故。夫血脈滎衛,周流不休,上應星宿,下應經數,寒邪客經絡之中,則血泣,血泣則不通,不通則衛氣歸之,不得復反,故癰腫寒氣化為熱,熱勝則腐肉,肉腐則為膿,膿不瀉則爛筋,筋爛則傷骨,骨傷則髓消,不當骨空,不得洩瀉,血枯空虛,則筋骨肌肉不相榮,經脈敗漏,薰於五藏,藏傷故死矣。

黃帝曰:願盡聞癰疽之形,與忌日名。岐伯曰:癰發於嗌中,名曰猛疽。猛疽不治,化為膿,膿不瀉,塞咽,半日死。其化為膿者,瀉則合豕膏,冷食,三日而已。發於頸,名曰夭疽,其癰大以赤黑,不急治,則熱氣下入淵腋,前傷任脈,內薰肝肺,薰肝肺,十餘日而死矣。陽留大發,消腦留項,名曰腦爍,其色不樂,項痛而如刺以針,煩心者,死不可治。發於肩及,名曰疵癰,其狀赤黑,急治之,此令人汗出至足,不害五藏,癰發四五日,逞之。發於腋下赤堅者,名曰米疽,治之以砭石,欲細而長,疏砭之,涂已豕膏,六日已,勿裹之。其癰堅而不潰者,為馬刀挾癭,急治之。發於胸,名曰井疽,其狀如大豆,三四日起,不早治,下入腹,不治,七日死矣。發於膺,名曰甘疽,色青,其狀如谷實,常苦寒熱,急治之,去其寒熱,十歲死,死後出膿。發於脅,名曰敗疵,敗疵者,女子之病也,炙之,其病大癰膿,治之,其中乃有生肉,大如赤小豆,銼草根各一升,以水一斗六升煮之,竭為取三升,則強飲厚衣,坐於釜上令汗出至足,已。發於股脛,名曰股脛疽,其狀不甚,變而癰膿搏骨,不急治,三十日死矣。發於尻,名曰銳疽,其狀赤堅大,急治之,不治,三十日死矣。發於股陰,名曰赤施,不急治,六十日死,在兩股之內,不治,十日而當死,發於膝,名曰疵癰,其狀大,癰色不變,寒熱,如堅石,勿石,石之者死。須其柔,乃石之者,生。諸癰疽之發於節而相應者,不可治也,發於陽者,百日死,發於陰者,三十日死。發於脛,名曰兔,其狀赤至骨,急治之,不治害人也。發於內踝,名曰走緩,其狀癰也,色不變,數石其輸,而止其寒熱,不死。發於足上下,名曰四淫,其狀大癰,急治之,百日死。發於足傍,名曰厲癰,其狀不大,初如小指發,急治之,去其黑者,不消輒益,不治,百日死。發於足指,名脫癰,其狀赤黑,死不治,不赤黑,不死,不衰,急斬之,不則死矣。

黃帝曰:夫子言癰疽,何以別之,岐伯曰:滎衛稽留於經脈之中,則血泣而不行,不行則衛氣從之而不通,壅遏而不得行,故熱。大熱不止,熱勝,則肉腐,肉腐則為膿,然不能陷骨髓,不為枯,五藏不為傷,故命曰癰。黃帝曰:何謂疽。岐伯曰:熱氣淳盛,下陷肌膚,筋髓枯,內連五藏,血氣竭,當其癰下,筋骨良肉皆無餘,故命曰疽。疽者,上之皮夭以堅,上如牛領之皮,癰者其皮上薄以澤,此其候也。
