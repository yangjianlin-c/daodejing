\documentclass[fontset=none]{utbook}

%% %% font settings
%% \setCJKfamilyfont{adobefs}{AdobeFangsongStd-Regular.otf}
%% \newcommand{\adobefs}{\CJKfamily{adobefs}}
%% \setCJKfamilyfont{adobekai}{AdobeKaitiStd-Regular.otf}
%% \newcommand{\adobekai}{\CJKfamily{adobekai}}
%% \setCJKfamilyfont{adobehei}{AdobeHeitiStd-Regular.otf}
%% \newcommand{\adobehei}{\CJKfamily{adobehei}}
%% \setCJKmainfont{Adobe Kaiti}

\DeclareFontFamily{JY2}{jpnrm}{}
\DeclareFontFamily{JY2}{schrm}{}
\DeclareFontFamily{JY2}{tchrm}{}
\DeclareFontFamily{JY2}{korrm}{}
\DeclareFontShape{JY2}{jpnrm}{m}{n}{<->s*[0.962216]upjpnrm-h}{}
\DeclareFontShape{JY2}{schrm}{m}{n}{<->s*[0.962216]upschrm-h}{}
\DeclareFontShape{JY2}{tchrm}{m}{n}{<->s*[0.962216]uptchrm-h}{}
\DeclareFontShape{JY2}{korrm}{m}{n}{<->s*[0.962216]upkorrm-h}{}
\DeclareFontShape{JY2}{jpnrm}{bx}{n}{<->ssub*jpnrm/m/n}{}
\DeclareFontShape{JY2}{schrm}{bx}{n}{<->ssub*schrm/m/n}{}
\DeclareFontShape{JY2}{tchrm}{bx}{n}{<->ssub*tchrm/m/n}{}
\DeclareFontShape{JY2}{korrm}{bx}{n}{<->ssub*korrm/m/n}{}
\DeclareRobustCommand\jpnrm{\kanjifamily{jpnrm}\selectfont}
\DeclareRobustCommand\schrm{\kanjifamily{schrm}\selectfont}
\DeclareRobustCommand\tchrm{\kanjifamily{tchrm}\selectfont}
\DeclareRobustCommand\korrm{\kanjifamily{korrm}\selectfont}


\begin{document}


使用uplatex(epTex内核)进行中文直排

http://bbs.ctex.org/forum.php?mod=viewthread\&tid=74899

uplatex test 

dvipdfmx -f charactor\_font.map test

%% \section{日本語}
%% \jpnrm
%% すべての人間は、生まれながらにして自由であり、かつ、尊厳と権利とについて平等である。
%% 人間は、理性と良心とを授けられており、互いに同胞の精神をもって行動しなければならない。

%% \section{中国語・簡体字 {\schrm 简体中文}}
%% {\schrm
%% 人人生而自由,在尊严和权利上一律平等。
%% 他们赋有理性和良心,并应以兄弟关系的精神相对待。
%% }

%% \section{中国語・繁体字 {\tchrm 繁體中文}}
%% {\tchrm
%% 人人生而自由,在尊嚴和權利上一律平等。
%% 他們賦有理性和良心,並應以兄弟關係的精神相對待。
%% }

%% \section{韓国語 {\korrm 한국어}}
%% {\korrm
%% \xkanjiskip=.1zw plus 1pt minus 1pt
%% 모든 인간은 태어날 때부터 자유로우며 그 존엄과 권리에 있어 동등하다.
%% 인간은 천부적으로 이성과 양심을 부여받았으며 서로 형제애의 정신으로 행동하여야 한다.
%% }

\section{中国語・簡体字 简体中文}
\Large

\textit{我第二次到仙岩的时候,我惊诧于梅雨潭的绿了。}

\textbf{梅雨潭是一个瀑布潭。仙瀑有三个瀑布,梅雨瀑最低。走到山边,便听见花花花花的声音;抬起头,镶在两条湿湿的黑边儿里的,一带白而发亮的水便呈现于眼前了。}

\textsc{我们先到梅雨亭。} 梅雨亭正对着那条瀑布;坐在亭边,不必仰头,便可见它的全体了。亭下深深的便是梅雨潭。这个亭踞在突出的一角的岩石上,上下都空空儿的;仿佛一只苍鹰展着翼翅浮在天宇中一般。三面都是山,像半个环儿拥着;人如在井底了。这是一个秋季的薄阴的天气。微微的云在我们顶上流着;岩面与草丛都从润湿中透出几分油油的绿意。而瀑布也似乎分外的响了。那瀑布从上面冲下,仿佛已被扯成大小的几绺;不复是一幅整齐而平滑的布。岩上有许多棱角;瀑流经过时,作急剧的撞击,便飞花碎玉般乱溅着了。那溅着的水花,晶莹而多芒;远望去,像一朵朵小小的白梅,微雨似的纷纷落着。据说,这就是梅雨潭之所以得名了。但我觉得像杨花,格外确切些。轻风起来时,点点随风飘散,那更是杨花了。——这时偶然有几点送入我们温暖的怀里,便倏的钻了进去,再也寻它不着。


\textsf{昔在黃帝生而神靈弱而能言幼而徇齊長而敦敏成而登天} 乃問於天師曰余聞上古之人春秋皆度百歲而動作不衰今時之人年半百而動作皆衰者時世異耶人將失之耶岐伯對曰上古之人其知道者法於陰陽和於術數食飲有節起居有常不妄作勞故能形與神俱而盡終其天年度百歲乃去今時之人不然也以酒為漿以妄為常醉以入房以欲竭其精以耗散其真不知持滿不時御神務快其心逆於生樂起居無節故半百而衰也

夫上古聖人之教下也.皆謂之虛邪賊風.避之有時.恬淡虛無.真氣從之.精神內守.病安從來.是以志閒而少欲.心安而不懼.形勞而不倦.氣從以順.各從其欲.皆得所願.故美其食.任其服.樂其俗.高下不相慕.其民故曰朴.是以嗜欲不能勞其目.淫邪不能惑其心.愚智賢不肖不懼於物.故合於道.所以能年皆度百歲.而動作不衰者.以其德全不危也.

帝曰。人年老而無子者。材力盡耶。將天數然也。岐伯曰。女子七歲。腎氣盛。齒更髮長。二七而天癸至。任脈通。太衝脈盛。月事以時下。故有子。三七。腎氣平均。故真牙生而長極。四七。筋骨堅。髮長極。身體盛壯。五七。陽明脈衰。面始焦。發始墮。六七。三陽脈衰於上。面皆焦。發始白。七七。任脈虛。太衝脈衰少。天癸竭。地道不通。故形壞而無子也。丈夫八歲。腎氣實。髮長齒更。二八。腎氣盛。天癸至。精氣溢寫。陰陽和。故能有子。三八。腎氣平均。筋骨勁強。故真牙生而長極。四八。筋骨隆盛。肌肉滿壯。五八。腎氣衰。發墮齒槁。六八。陽氣衰竭於上。面焦。髮鬢頒白。七八。肝氣衰。筋不能動。天癸竭。精少。腎藏衰。形體皆極。八八。則齒發去。腎者主水。受五藏六府之精而藏之。故五藏盛。乃能寫。今五藏皆衰。筋骨解墮。天癸盡矣。故髮鬢白。身體重。行步不正。而無子耳。

帝曰。有其年已老而有子者何也。岐伯曰。此其天壽過度。氣脈常通。而腎氣有餘也。此雖有子。男不過盡八八。女不過盡七七。而天地之精氣皆竭矣。

帝曰。夫道者年皆百數。能有子乎。岐伯曰。夫道者能卻老而全形。身年雖壽。能生子也。

黃帝曰。余聞上古有真人者。提挈天地。把握陰陽。呼吸精氣。獨立守神。肌肉若一。故能壽敝天地。無有終時。此其道生。中古之時。有至人者。淳德全道。和於陰陽。調於四時。去世離俗。積精全神。遊行天地之間。視聽八達之外。此蓋益其壽命而強者也。亦歸於真人。其次有聖人者。處天地之和。從八風之理。適嗜欲於世俗之間。無恚嗔之心。行不欲離於世。被服章。舉不欲觀於俗。外不勞形於事。內無思想之患。以恬愉為務。以自得為功。形體不敝。精神不散。亦可以百數。其次有賢人者。法則天地。像似日月。辨列星辰。逆從陰陽。分別四時。將從上古合同於道。亦可使益壽而有極時。

  春三月,此謂發陳,天地俱生,萬物以榮,夜臥早起,廣步於庭,被發緩形,以使志生,生而勿殺,予而勿奪,賞而勿罰,此春氣之應,養生之道也。逆之則傷肝,夏為寒變,奉長者少。
  夏三月,此謂蕃秀,天地氣交,萬物華實,夜臥早起,無厭於日,使志無怒,使華英成秀,使氣得洩,若所愛在外,此夏氣之應,養長之道也。逆之則傷心,秋為痎瘧,奉收者少,冬至重病。
  秋三月,此謂容平,天氣以急,地氣以明,早臥早起,與雞俱興,使志安寧,以緩秋刑,收斂神氣,使秋氣平,無外其志,使肺氣清,此秋氣之應,養收之道也。逆之則傷肺,冬為飧洩,奉藏者少。
  冬三月,此謂閉藏,水冰地坼,無擾乎陽,早臥晚起,必待日光,使志若伏若匿,若有私意,若已有得,去寒就溫,無洩皮膚,使氣亟奪,此冬氣之應,養藏之道也。逆之則傷腎,春為痿厥,奉生者少。
  天氣,清淨光明者也,藏德不止,故不下也。天明則日月不明,邪害空竅,陽氣者閉塞,地氣者冒明,雲霧不精,則上應白露不下。交通不表,萬物命故不施,不施則名木多死。惡氣不發,風雨不節,白露不下,則菀槁不榮。賊風數至,暴雨數起,天地四時不相保,與道相失,則未央絕滅。唯聖人從之,故身無奇病,萬物不失,生氣不竭。逆春氣,則少陽不生,肝氣內變。逆夏氣,則太陽不長,心氣內洞。逆秋氣,則太陰不收,肺氣焦滿。逆冬氣,則少陰不藏,腎氣獨沉。夫四時陰陽者,萬物之根本也。所以聖人春夏養陽,秋冬養陰,以從其根,故與萬物沉浮於生長之門。逆其根,則伐其本,壞其真矣。
  故陰陽四時者,萬物之終始也,死生之本也,逆之則災害生,從之則苛疾不起,是謂得道。道者,聖人行之,愚者佩之。從陰陽則生。逆之則死,從之則治,逆之則亂。反順為逆,是謂內格。
  是故聖人不治已病,治未病,不治已亂,治未亂,此之謂也。夫病已成而後藥之,亂已成而後治之,譬猶渴而穿井,而鑄錐,不亦晚乎。


\section{生氣通天論篇第三}

  黃帝曰:夫自古通天者生之本,本於陰陽。天地之間,六合之內,其氣九州、九竅、五藏、十二節,皆通乎天氣。其生五,其氣三,數犯此者,則邪氣傷人,此壽命之本也。
  蒼天之氣清淨,則志意治,順之則陽氣固,雖有賊邪,弗能害也,此因時之序。故聖人傳精神,服天氣,而通神明。失之則內閉九竅,外壅肌肉,衛氣散解,此謂自傷,氣之削也。
  陽氣者若天與日,失其所,則折壽而不彰,故天運當以日光明。是故陽因而上,衛外者也。因於寒,欲如運樞,起居如驚,神氣乃浮。因於暑,汗煩則喘喝,靜則多言,體若燔炭,汗出而散。因於濕,首如裹,濕熱不攘,大筋短,小筋弛長,短為拘,弛長為痿。因於氣,為腫,四維相代,陽氣乃竭。
  陽氣者,煩勞則張,精絕,辟積於夏,使人煎厥。目盲不可以視,耳閉不可以聽,潰潰乎若壞都,汨汨乎不可止。陽氣者,大怒則形氣絕,而血菀於上,使人薄厥。有傷於筋,縱,其若不容,汗出偏沮,使人偏枯。汗出見濕,乃生痤。高粱之變,足生大丁,受如持虛。勞汗當風,寒薄為,郁乃痤。
  陽氣者,精則養神,柔則養筋。開闔不得,寒氣從之,乃生大僂。陷脈為瘻。留連肉腠,俞氣化薄,傳為善畏,及為驚駭。營氣不從,逆於肉理,乃生癰腫。魄汗未盡,形弱而氣爍,穴俞以閉,發為風瘧。
  故風者,百病之始也,清靜則肉腠閉拒,雖有大風苛毒,弗之能害,此因時之序也。
  故病久則傳化,上下不併,良醫弗為。故陽畜積病死,而陽氣當隔,隔者當寫,不亟正治,粗乃敗之。
  故陽氣者,一日而主外,平旦人氣生,日中而陽氣隆,日西而陽氣已虛,氣門乃閉。是故暮而收拒,無擾筋骨,無見霧露,反此三時,形乃困薄。
  岐伯曰:陰者,藏精而起亟也,陽者,衛外而為固也。陰不勝其陽,則脈流薄疾,並乃狂。陽不勝其陰,則五藏氣爭,九竅不通。是以聖人陳陰陽,筋脈和同,骨髓堅固,氣血皆從。如是則內外調和,邪不能害,耳目聰明,氣立如故。
  風客淫氣,精乃亡,邪傷肝也。因而飽食,筋脈橫解,腸澼為痔。因而大飲,則氣逆。因而強力,腎氣乃傷,高骨乃壞。
  凡陰陽之要,陽密乃固,兩者不和,若春無秋,若冬無夏,因而和之,是謂聖度。故陽強不能密,陰氣乃絕,陰平陽秘,精神乃治,陰陽離決,精氣乃絕。
  因於露風,乃生寒熱。是以春傷於風,邪氣留連,乃為洞洩,夏傷於暑,秋為瘧。秋傷於濕,上逆而咳,發為痿厥。冬傷於寒,春必溫病。四時之氣,更傷五藏。
  陰之所生,本在五味,陰之五宮,傷在五味。是故味過於酸,肝氣以津,脾氣乃絕。味過於咸,大骨氣勞,短肌,心氣抑。味過於甘,心氣喘滿,色黑,腎氣不衡。味過於苦,脾氣不濡,胃氣乃厚。味過於辛,筋脈沮弛,精神乃央。是故謹和五味,骨正筋柔,氣血以流,腠理以密,如是,則骨氣以精,謹道如法,長有天命。


\section{金匱真言論篇第四}

  黃帝問曰:天有八風,經有五風,何謂?岐伯對曰:八風發邪,以為經風,觸五藏,邪氣發病。所謂得四時之勝者,春勝長夏,長夏勝冬,冬勝夏,夏勝秋,秋勝春,所謂四時之勝也。
  東風生於春,病在肝,俞在頸項;南風生於夏,病在心,俞在胸脅;西風生於秋,病在肺,俞在肩背;北風生於冬,病在腎,俞在腰股;中央為土,病在脾,俞在脊。故春氣者病在頭,夏氣者病在藏,秋氣者病在肩背,冬氣者病在四支。
  故春善病鼽衄,仲夏善病胸脅,長夏善病洞洩寒中,秋善病風瘧,冬善病痹厥。故冬不按蹻,春不鼽衄,春不病頸項,仲夏不病胸脅,長夏不病洞洩寒中,秋不病風瘧,冬不病痹厥,飧洩而汗出也。
  夫精者身之本也。故藏於精者春不病溫。夏暑汗不出者,秋成風瘧。此平人脈法也。
  故曰:陰中有陰,陽中有陽。平旦至日中,天之陽,陽中之陽也;日中至黃昏,天之陽,陽中之陰也;合夜至雞鳴,天之陰,陰中之陰也;雞鳴至平旦,天之陰,陰中之陽也。
  故人亦應之。夫言人之陰陽,則外為陽,內為陰。言人身之陰陽,則背為陽,腹為陰。言人身之藏府中陰陽。則藏者為陰,府者為陽。肝心脾肺腎五藏,皆為陰。膽胃大腸小腸膀胱三焦六府,皆為陽。所以欲知陰中之陰陽中之陽者何也,為冬病在陰,夏病在陽,春病在陰,秋病在陽,皆視其所在,為施針石也。故背為陽,陽中之陽,心也;背為陽,陽中之陰,肺也;腹為陰,陰中之陰,腎也;腹為陰,陰中之陽,肝也;腹為陰,陰中之至陰,脾也。此皆陰陽表裡內外雌雄相俞應也,故以應天之陰陽也。
  帝曰:五藏應四時,各有收受乎?岐伯曰:有。東方青色,入通於肝,開竅於目,藏精於肝,其病發驚駭。其味酸,其類草木,其畜雞,其穀麥,其應四時,上為歲星,是以春氣在頭也,其音角,其數八,是以知病之在筋也,其臭臊。
  南方赤色,入通於心,開竅於耳,藏精於心,故病在五藏,其味苦,其類火,其畜羊,其谷黍,其應四時,上為熒惑星,是以知病之在脈也,其音徵,其數七,其臭焦。
  中央黃色,入通於脾,開竅於口,藏精於脾,故病在舌本,其味甘,其類土,其畜牛,其谷稷,其應四時,上為鎮星,是以知病之在肉也,其音宮,其數五,其臭香。
  西方白色,入通於肺,開竅於鼻,藏精於肺,故病在背,其味辛,其類金,其畜馬,其穀稻,其應四時,上為太白星,是以知病之在皮毛也,其音商,其數九,其臭腥。
  北方黑色,入通於腎,開竅於二陰,藏精於腎,故病在谿,其味咸,其類水,其畜彘,其谷豆,其應四時,上為辰星,是以知病之在骨也,其音羽,其數六,其臭腐。故善為脈者,謹察五藏六府,一逆一從,陰陽表裡雌雄之紀,藏之心意,合心於精,非其人勿教,非其真勿授,是謂得道。


\section{陰陽應像大論篇第五}

黃帝曰:陰陽者,天地之道也,萬物之綱紀,變化之父母,生殺之本始,神明之府也。治病必求於本。故積陽為天,積陰為地。陰靜陽躁,陽生陰長,陽殺陰藏。陽化氣,陰成形。寒極生熱,熱極生寒。寒氣生濁,熱氣生清。清氣在下,則生飧洩;濁氣在上,則生(月真)脹。此陰陽反作,病之逆從也。
  故清陽為天,濁陰為地;地氣上為雲,天氣下為雨;雨出地氣,雲出天氣。故清陽出上竅,濁陰出下竅;清陽發腠理,濁陰走五藏;清陽實四支,濁陰歸六府。
  水為陰,火為陽,陽為氣,陰為味。味歸形,形歸氣,氣歸精,精歸化,精食氣,形食味,化生精,氣生形。味傷形,氣傷精,精化為氣,氣傷於味。
  陰味出下竅,陽氣出上竅。味厚者為陰,薄為陰之陽。氣厚者為陽,薄為陽之陰。味厚則洩,薄則通。氣薄則發洩,厚則發熱。壯火之氣衰,少火之氣壯。壯火食氣,氣食少火。壯火散氣,少火生氣。
  氣味辛甘發散為陽,酸苦湧洩為陰。陰勝則陽病,陽勝則陰病。陽勝則熱,陰勝則寒。重寒則熱,重熱則寒。寒傷形,熱傷氣。氣傷痛,形傷腫。故先痛而後腫者,氣傷形也;先腫而後痛者,形傷氣也。
  風勝則動,熱勝則腫,燥勝則干,寒勝則浮,濕勝則濡寫。
  天有四時五行,以生長收藏,以生寒暑燥濕風。人有五藏,化五氣,以生喜怒悲憂恐。故喜怒傷氣,寒暑傷形。暴怒傷陰,暴喜傷陽。厥氣上行,滿脈去形。喜怒不節,寒暑過度,生乃不固。故重陰必陽,重陽必陰。
  故曰:冬傷於寒,春必溫病;春傷於風,夏生飧洩;夏傷於暑,秋必痎瘧;秋傷於濕,冬生咳嗽。
  帝曰:余聞上古聖人,論理人形,列別藏府,端絡經脈,會通六合,各從其經,氣穴所發各有處名,谿谷屬骨皆有所起,分部逆從,各有條理,四時陰陽,盡有經紀,外內之應,皆有表裡,其信然乎?
  岐伯對曰:東方生風,風生木,木生酸,酸生肝,肝生筋,筋生心,肝主目。其在天為玄,在人為道,在地為化。化生五味,道生智,玄生神,神在天為風,在地為木,在體為筋,在藏為肝,在色為蒼,在音為角,在聲為呼,在變動為握,在竅為目,在味為酸,在志為怒。怒傷肝,悲勝怒;風傷筋,燥勝風;酸傷筋,辛勝酸。
  南方生熱,熱生火,火生苦,苦生心,心生血,血生脾,心主舌。其在天為熱,在地為火,在體為脈,在藏為心,在色為赤,在音為徵,在聲為笑,在變動為憂,在竅為舌,在味為苦,在志為喜。喜傷心,恐勝喜;熱傷氣,寒勝熱,苦傷氣,咸勝苦。
  中央生濕,濕生土,土生甘,甘生脾,脾生肉,肉生肺,脾主口。其在天為濕,在地為土,在體為肉,在藏為脾,在色為黃,在音為宮,在聲為歌,在變動為噦,在竅為口,在味為甘,在志為思。思傷脾,怒勝思;濕傷肉,風勝濕;甘傷肉,酸勝甘。
  西方生燥,燥生金,金生辛,辛生肺,肺生皮毛,皮毛生腎,肺主鼻。其在天為燥,在地為金,在體為皮毛,在藏為肺,在色為白,在音為商,在聲為哭,在變動為咳,在竅為鼻,在味為辛,在志為憂。憂傷肺,喜勝憂;熱傷皮毛,寒勝熱;辛傷皮毛,苦勝辛。
  北方生寒,寒生水,水生咸,咸生腎,腎生骨髓,髓生肝,腎主耳。其在天為寒,在地為水,在體為骨,在藏為腎,在色為黑,在音為羽,在聲為呻,在變動為栗,在竅為耳,在味為咸,在志為恐。恐傷腎,思勝恐;寒傷血,燥勝寒;咸傷血,甘勝咸。
  故曰:天地者,萬物之上下也;陰陽者,血氣之男女也;左右者,陰陽之道路也;水火者,陰陽之徵兆也;陰陽者,萬物之能始也。故曰:陰在內,陽之守也;陽在外,陰之使也。
  帝曰:法陰陽奈何?岐伯曰:陽勝則身熱,腠理閉,喘粗為之仰,汗不出而熱,齒干以煩冤,腹滿,死,能冬不能夏。陰勝則身寒,汗出,身常清,數栗而寒,寒則厥,厥則腹滿,死,能夏不能冬。此陰陽更勝之變,病之形能也。
  帝曰:調此二者奈何?岐伯曰:能知七損八益,則二者可調,不知用此,則早衰之節也。年四十,而陰氣自半也,起居衰矣。年五十,體重,耳目不聰明矣。年六十,陰痿,氣大衰,九竅不利,下虛上實,涕泣俱出矣。故曰:知之則強,不知則老,故同出而名異耳。智者察同,愚者察異,愚者不足,智者有餘,有餘則耳目聰明,身體輕強,老者復壯,壯者益治。是以聖人為無為之事,樂恬憺之能,從欲快志於虛無之守,故壽命無窮,與天地終,此聖人之治身也。
  天不足西北,故西北方陰也,而人右耳目不如左明也。地不滿東南,故東南方陽也,而人左手足不如右強也。帝曰:何以然?岐伯曰:東方陽也,陽者其精並於上,並於上,則上明而下虛,故使耳目聰明,而手足不便也。西方陰也,陰者其精並於下,並於下,則下盛而上虛,故其耳目不聰明,而手足便也。故俱感於邪,其在上則右甚,在下則左甚,此天地陰陽所不能全也,故邪居之。
  故天有精,地有形,天有八紀,地有五里,故能為萬物之父母。清陽上天,濁陰歸地,是故天地之動靜,神明為之綱紀,故能以生長收藏,終而復始。惟賢人上配天以養頭,下象地以養足,中傍人事以養五藏。天氣通於肺,地氣通於嗌,風氣通於肝,雷氣通於心,谷氣通於脾,雨氣通於腎。六經為川,腸胃為海,九竅為水注之氣。以天地為之陰陽,陽之汗,以天地之雨名之;陽之氣,以天地之疾風名之。暴氣象雷,逆氣象陽。故治不法天之紀,不用地之理,則災害至矣。
  故邪風之至,疾如風雨,故善治者治皮毛,其次治肌膚,其次治筋脈,其次治六府,其次治五藏。治五藏者,半死半生也。故天之邪氣,感則害人五藏;水谷之寒熱,感則害於六府;地之濕氣,感則害皮肉筋脈。
  故善用針者,從陰引陽,從陽引陰,以右治左,以左治右,以我知彼,以表知裡,以觀過與不及之理,見微得過,用之不殆。善診者,察色按脈,先別陰陽;審清濁,而知部分;視喘息,聽音聲,而知所苦;觀權衡規矩,而知病所主。按尺寸,觀浮沉滑澀,而知病所生;以治無過,以診則不失矣。
  故曰:病之始起也,可刺而已;其盛,可待衰而已。故因其輕而揚之,因其重而減之,因其衰而彰之。形不足者,溫之以氣;精不足者,補之以味。其高者,因而越之;其下者,引而竭之;中滿者,寫之於內;其有邪者,漬形以為汗;其在皮者,汗而發之;其慓悍者,按而收之;其實者,散而寫之。審其陰陽,以別柔剛,陽病治陰,陰病治陽,定其血氣,各守其鄉,血實宜決之,氣虛宜掣引之。



%% \section{异体字注音}

%% 焫:ruo4
%% 鑱:chan2
%% 熇:he4
%% 虙:fu2
%% 皏:peng3
%% 炲:tai2
%% 脽:shui2
%% 齗:yin2
%% 鬄:ti4
%% 稸:xu4
%% 黅:jin1
%% 憹:nao2
%% 爇:ruo4
%% 吤:jie4
%% 蛕:hui2,蛔
%% 腄:chui2
%% 晬:zui4
%% 肬:you2
%% 髃:yu2
%% 覩:du3,睹
%% 昬:hun1,昏
%% 顀:chui2,椎
%% 頄:qiu2,颧骨

%% (月真):chen1,胸膈或上腹部脹滿不適
%% (雩重):zhong1,氣之往來不息
%% (疒肙):juan1
%% (亻亦):yi4
%% (月囷):jun4,肌肉的突起部位
%% (火矣):ai1
%% (目巟):huang1,目昏暗,視物不清
%% (骨行):heng2,脛腓骨的統稱,小腿部,腳脛部
%% (疒龍):
%% (月少):miao3,季脅下挾脊兩旁空軟處
%% (火矣):ai1,火燒,火熨、灸焫等治法
%% (骨盾):tu2:皮肉肥厚之處
%% (月呂):lv2
%% (蕓去草頭令)
%% (疒帬):wan2,痹,麻木
%% (出頁):zhuo1,眼眶下面的骨
%% (疒峻-山):
%% (疒貴):tui2
%% (亻聶):che4,懾
%% (口父):fu3,用嘴咀嚼
%% (去欠):qu4,呿,张口
%% (骨曷):he2,肩骨
%% (骨亏)
%% (疒水):shui4

\end{document}
