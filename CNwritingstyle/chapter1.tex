\chapter{中文直書}
使用方式
\begin{itemize}
\item 主檔名為{\tt test.tex}。其檔案結構如第三章所示,基本上是根據\LaTeX{}的檔案結構,指令亦是\LaTeX{}指令。
{\tt \textbackslash chapter}\{章名\},
{\tt \textbackslash section}\{節名\},
{\tt \textbackslash subsection}\{小節名\}。
\item
預設是中文直向直書,但書面旋轉九十度。故完稿時要反轉九十度。編排時是水平排列以便閱讀,完稿時請除去{\tt CJKhorz}指令,然後編譯數次後,再用{\tt PDF}旋轉功能,轉成垂直排列。
\item
編排時會產生文字框,完稿時請在最上一行加入{\tt [noframe]}以消除文字框。
\end{itemize}
\begin{table}
\caption{中文圖表}
\rotatebox[origin=cc]{90}{
\begin{tabular}{ccc}\\ \hline
{\CJKhorz 實驗} & {\CJKhorz方法一} & {\CJKhorz方法二} \\
1      &   123 & 456 \\
2      &  abc  & def \\ \hline
\end{tabular}
}
\end{table}
如{\tt section 1.1, subsection 1.1.2}。本書將以幾之幾表達。例如第一章第一節不以第{\tt 1.1}節表示,而以第一之一節表示。\cite{test}\cite{test2}
\section{唐詩}
第一章將以詩詞測試。
\subsection{樂遊原 唐 李商隱}

向晚意不適 驅車登古原夕陽無限好 只是近黃昏
\subsection{無題 唐 李商隱}

相見時難別亦難 東風無力百花殘春蠶到死絲方盡 蠟炬成灰淚始乾

曉鏡但愁雲鬢改 夜吟應覺月光寒蓬萊此去無多路 青鳥殷勤為探看
\subsection{獨坐敬亭山 唐 李白}

眾鳥高飛盡  孤雲獨去閒  相看兩不厭 只有敬亭山
\section{勵志} \index{勵志}
第二節則以詩詞勵志。
\subsection{無題}

鳥入青雲倦亦飛\\
\fbox{\small
\begin{tabular}{l}
小鳥飛入高空,疲倦了還是要飛,不可休息。青雲指青雲之志,鳥當指麻雀等小鳥而非鷹騭。\\
故該句引伸為即使一般人達到目標後,仍應當繼續努力,不可休息。
\end{tabular}
}\\
這裡的翻譯是採立即出現在本文左邊。有些文言文書是一段一段地翻譯。

\subsection{卜算子 詠梅}

陸游\\
驛外斷橋邊 寂寞開無主 已是黃昏獨自愁 更著風和雨\\
無意苦爭春 一任群芳妒 零落成泥碾作塵 只有香如故\\
\fbox{\small
\begin{minipage}{0.49\textwidth}
\begin{enumerate}
\item[\rotatebox{90}{(1)}] 驛外,斷橋邊 指所居非地。
\item[\rotatebox{90}{(2)}] 黃昏,風和雨 指所遇非時。
\end{enumerate}
\end{minipage}
\vrule
\begin{minipage}{0.49\textwidth}
\begin{enumerate}
\item[\rotatebox{90}{(3)}] 苦爭春 苦苦地與他人爭鋒。
\item[\rotatebox{90}{(4)}] 碾作塵 碾成泥土,依然綻放著清香。
\end{enumerate}
\end{minipage}
}
這裡分成上下兩欄的註釋欄,其實可依第二章的技巧寫成含變數的{\tt newtheorem};如果更多則可寫入{\tt cnwritingCJK}成{\tt Class}檔。同樣地有些書的註解是放在本文後。本書只是想說明中文書籍亦可用\LaTeX{}做出,甚至更具特色。不一定要用{\tt Office Words}。

\subsection{中文古籍}

此節將比擬於中文古籍,不分節、小節,有如章回小說。

\section{將進酒}
\begin{figure}
\includegraphics[angle=90]{NCU_logo.jpg}
\caption{中文圖表}
\end{figure}

唐 李白\\
君不見 黃河之水天上來 奔流到海不復回\\
君不見 高堂明鏡悲白髮 朝如青絲暮成雪\\
人生得意須盡歡 莫使金樽空對月\\
天生我材必有用 千金散盡還復來\\
烹羊宰牛且為樂 會須一飲三百杯\\
岑夫子 丹丘生\\
將進酒  杯莫停\\
與君歌一曲 請君為我側耳聽\\
鐘鼓饌玉不足貴 但願長醉不願醒\\
古來聖賢皆寂寞 惟有飲者留其名\\
陳王昔時宴平樂 斗酒十千恣讙謔\\
主人何為言少錢 徑須沽取對君酌\\
五花馬 千金裘\\
呼兒將出換美酒 與爾同消萬古愁
