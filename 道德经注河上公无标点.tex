\documentclass[a4paper,zihao=-4,oneside,landscape,UTF8]{ctexart}

%页面旋转
\usepackage{everypage}
\AddEverypageHook{\special{pdf: put @thispage <</Rotate 90>>}}

%% font文字旋转
\defaultCJKfontfeatures{RawFeature={vertical:+vert}}

%% 默认字体为源雲明體
\setCJKmainfont[BoldFont=源雲明體 TTF SemiBold,ItalicFont=思源宋体 Light]{源雲明體 TTF Medium}

% 页码编号为汉字
\renewcommand{\thepage}{\Chinese{page}}

\usepackage{geometry}
\newgeometry{top=50pt, bottom=100pt, left=100pt, right=80pt,headsep=0pt}

\usepackage{titlesec}
\titleformat{\section}[hang]{\large\bfseries}{}{0pt}{}
\titlespacing{\section}{4em}{0pt}{0pt}


\ctexset{
	today = big,
	punct = quanjiao, 
	autoindent = 0pt,	
}

%去掉页眉
\usepackage{fancyhdr}
\renewcommand{\headrulewidth}{0.0pt}%
\fancypagestyle{plain}{% Redefine plain pages tyle  
  % Clear header/footer
  \fancyhf{}
  \renewcommand{\headrulewidth}{0.0pt}%
  \fancyfoot[L]{\thepage}
}
\pagestyle{plain}

%%% 命令设置
\usepackage{xcolor}
\definecolor{gray}{gray}{0.3}
\newcommand{\zhushi}[1]{\scriptsize{\textit{\textcolor{gray}{#1}}}\normalsize}

\definecolor{gray1}{gray}{0.2}
\color{gray1}


\title{\textbf{老子}\textit{ 注 \hspace{5em}河上公}\hfil}
\author{Colin Yang \\Email: yangjianlin@gmail.com}
\date{\normalsize\today 版}




\begin{document}

\ziju{0.1}
\large

\maketitle


\section{體道第一}


道可道
\zhushi{謂經術政教之道也}
非常道
\zhushi{非自然生長之道也常道當以無為養神無事安民含光藏暉滅跡匿端不可稱道}
名可名
\zhushi{謂富貴尊榮高世之名也}
非常名
\zhushi{非自然常在之名也常名當如嬰兒之未言雞子之未分明珠在蚌中美玉處石間內雖昭昭外如愚頑}
無名天地之始
\zhushi{無名者謂道道無形故不可名也始者道本也吐氣布化出於虛無為天地本始也}
有名萬物之母
\zhushi{有名謂天地天地有形位有陰陽有柔剛是其有名也萬物母者天地含氣生萬物長大成熟如母之養子也}
故常無欲以觀其妙
\zhushi{妙要也人常能無欲則可以觀道之要要謂一也一出布名道贊敘明是非}
常有欲以觀其僥
\zhushi{僥歸也常有欲之人可以觀世俗之所歸趣也}
此兩者同出而異名
\zhushi{兩者謂有欲無欲也同出者同出人心也而異名者所名各異也名無欲者長存名有欲者亡身也}
同謂之玄
\zhushi{玄天也言有欲之人與無欲之人同受氣於天}
玄之又玄
\zhushi{天中復有天也稟氣有厚薄得中和滋液則生賢聖得錯亂汙辱則生貪淫也}
眾妙之門
\zhushi{能知天中復有天稟氣有厚薄除情去欲守中和是謂知道要之門戶也}


\section{養身第二}

天下皆知美之為美
\zhushi{自揚己美使彰顯也}
斯惡已
\zhushi{有危亡也}
皆知善之為善
\zhushi{有功名也}
斯不善已
\zhushi{人所爭也}
故有無相生
\zhushi{見有而為無也}
難易相成
\zhushi{見難而為易也}
長短相較
\zhushi{見短而為長也}
高下相傾
\zhushi{見高而為下也}
音聲相和
\zhushi{上唱下必和也}
前後相隨
\zhushi{上行下必隨也}
是以聖人處無為之事
\zhushi{以道治也}
行不言之教
\zhushi{以身師導之也}
萬物作焉
\zhushi{各自動也}
而不辭
\zhushi{不辭謝而逆止}
生而不有
\zhushi{元氣生萬物而不有}
為而不恃
\zhushi{道所施為不恃望其報也}
功成而弗居
\zhushi{功成事就退避不居其位}
夫惟弗居
\zhushi{夫惟功成不居其位}
是以不去
\zhushi{福德常在不去其身也此言不行不可隨不言不可知疾上六句有高下長短君開一源下生百端百端之變無不動亂}


\section{安民第三}

不尚賢
\zhushi{賢謂世俗之賢辯口明文離道行權去質為文也不尚者不貴之以祿不貴之以官}
使民不爭
\zhushi{不爭功名返自然也}
不貴難得之貨
\zhushi{言人君不禦好珍寶黃金棄於山珠玉捐於淵}
使民不為盜
\zhushi{上化清靜下無貪人}
不見可欲
\zhushi{放鄭聲遠美人}
使心不亂
\zhushi{不邪淫不惑亂也}
是以聖人之治
\zhushi{說聖人治國與治身同也}
虛其心
\zhushi{除嗜欲去亂煩}
實其腹
\zhushi{懷道抱一守五神也}
弱其誌
\zhushi{和柔謙讓不處權也}
強其骨
\zhushi{愛精重施髓滿骨堅強其良反}
常使民無知無欲
\zhushi{返樸守淳}
使夫智者不敢為也
\zhushi{思慮深不輕言夫音符知音智}
為無為
\zhushi{不造作動因循}
則無不治
\zhushi{德化厚百姓安}


\section{無源第四}

道沖而用之
\zhushi{沖中也道匿名藏譽其用在中沖直隆反}
或不盈
\zhushi{或常也道常謙虛不盈滿}
淵乎似萬物之宗
\zhushi{道淵深不可知似為萬物之宗祖}
挫其銳
\zhushi{銳進也人欲銳精進取功名當挫止之法道不自見也}
解其紛
\zhushi{紛結恨也當念道無為以解釋}
和其光
\zhushi{言雖有獨見之明當知暗昧不當以擢亂人也}
同其塵
\zhushi{當與眾庶同垢塵不當自別殊}
湛兮似若存
\zhushi{言當湛然安靜故能長存不亡}
吾不知誰之子
\zhushi{老子言:我不知道所從生}
象帝之先
\zhushi{道自在天帝之前此言道乃先天地之生也至今在者以能安靜湛然不勞煩欲使人修身法道}


\section{虛用第五}

天地不仁
\zhushi{天施地化不以仁恩任自然也}
以萬物為芻狗
\zhushi{天地生萬物人最為貴天地視之如芻草狗畜不貴望其報也}
聖人不仁
\zhushi{聖人愛養萬民不以仁恩法天地行自然}
以百姓為芻狗
\zhushi{聖人視百姓如芻草狗畜不貴望其禮意}
天地之間
\zhushi{天地之間空虛和氣流行故萬物自生人能除情欲節滋味清五臟則神明居之也}
其猶橐龠乎
\zhushi{橐龠中空虛人能有聲氣}
虛而不屈動而愈出
\zhushi{言空虛無有屈竭時動搖之益出聲氣也}
多言數窮
\zhushi{多事害神多言害身口開舌舉必有禍患}
不如守中
\zhushi{不如守德於中育養精神愛氣希言}


\section{成象第六}

谷神不死
\zhushi{谷養也人能養神則不死也神謂五臟之神也肝藏魂肺藏魄心藏神腎藏精脾藏誌五藏盡傷則五神去矣}
是謂玄牝
\zhushi{言不死之有在於玄牝玄天也於人為鼻牝地也於人為口天食人以五氣從鼻入藏於心五氣輕微為精神聰明音聲五性其鬼曰魂魂者雄也主出入於人鼻與天通故鼻為玄也地食人以五味從口入藏於胃五味濁辱為形骸骨肉血脈六情其鬼曰魄魄者雌也主出入於人口與地通故口為牝也}
玄牝之門是謂天地根
\zhushi{根元也言鼻口之門是乃通天地之元氣所從往來也}
綿綿若存
\zhushi{鼻口呼噏喘息當綿綿微妙若可存復若無有}
用之不勤
\zhushi{用氣當寬舒不當急疾勤勞也}


\section{韜光第七}

天長地久
\zhushi{說天地長生久壽以喻教人也 }
天地所以能長且久者以其不自生
\zhushi{天地所以獨長且久者以其安靜施不求報不如人居處汲汲求自饒之利奪人以自與也}
故能長生
\zhushi{以其不求生故能長生不終也}
是以聖人後其身
\zhushi{先人而後己也}
而身先
\zhushi{天下敬之先以為長}
外其身
\zhushi{薄己而厚人也}
而身存
\zhushi{百姓愛之如父母神明佑之若赤子故身常存}
非以其無私邪
\zhushi{聖人為人所愛神明所佑非以其公正無私所致乎}
故能成其私
\zhushi{人以為私者欲以厚己也聖人無私而己自厚故能成其私也}


\section{易性第八}

上善若水
\zhushi{上善之人如水之性}
水善利萬物而不爭
\zhushi{水在天為霧露在地為源泉也}
處眾人之所惡
\zhushi{眾人惡卑濕垢濁水獨靜流居之也}
故幾於道
\zhushi{水性幾於道同}
居善地
\zhushi{水性善喜於地草木之上即流而下有似於牝動而下人也}
心善淵
\zhushi{水深空虛淵深清明}
與善仁
\zhushi{萬物得水以生與虛不與盈也}
言善信
\zhushi{水內影照形不失其情也}
正善治
\zhushi{無有不洗清且平也}
事善能
\zhushi{能方能圓曲直隨形}
動善時
\zhushi{夏散冬凝應期而動不失天時}
夫唯不爭
\zhushi{壅之則止決之則流聽從人也}
故無尤
\zhushi{水性如是故天下無有怨尤水者也}


\section{運夷第九}

持而盈之不如其已
\zhushi{盈滿也已止也持滿必傾不如止也}
揣而梲之不可長保
\zhushi{揣治也先揣之後必棄捐}
金玉滿堂莫之能守
\zhushi{嗜欲傷神財多累身}
富貴而驕自遺其咎
\zhushi{夫富當賑貧貴當憐賤而反驕恣必被禍患也}
功成名遂身退天之道
\zhushi{言人所為功成事立名跡稱遂不退身避位則遇於害此乃天之常道也譬如日中則移月滿則虧物盛則衰樂極則哀}


\section{能為第十}

載營魄
\zhushi{營魄魂魄也人載魂魄之上得以生當愛養之喜怒亡魂卒驚傷魄魂在肝魄在肺美酒甘肴腐人肝肺故魂靜誌道不亂魄安得壽延年也}
抱一能無離乎
\zhushi{言人能抱一使不離於身則長存一者道始所生太言道行德玄冥不可得見欲使人如道也}


\section{無用第十一}

三十輻共一轂
\zhushi{古者車三十輻法月數也共一轂者轂中有孔故眾輻共湊之治身者當除情去欲使五藏空虛神乃歸之治國者寡能摠眾弱共使強也}
當其無有車之用
\zhushi{無謂空虛轂中空虛輪得轉行輿中空虛人得載其上也}
埏埴以為器
\zhushi{埏和也埴土也和土以為飲食之器}
當其無有器之用
\zhushi{器中空虛故得有所盛受}
鑿戶牖以為室
\zhushi{謂作屋室}
當其無有室之用
\zhushi{言戶牖空虛人得以出入觀視室中空虛人得以居處是其用}
故有之以為利
\zhushi{利物也利於形用器中有物室中有人恐其屋破壞腹中有神畏其形亡也}
無之以為用
\zhushi{言虛空者乃可用盛受萬物故曰虛無能制有形道者空也}


\section{檢欲第十二}

五色令人目盲
\zhushi{貪淫好色則傷精失明也}
五音令人耳聾
\zhushi{好聽五音則和氣去心不能聽無聲之聲}
五味令人口爽
\zhushi{爽亡也人嗜於五味於口則口亡言失於道也}
馳騁畋獵令人心發狂
\zhushi{人精神好安靜馳騁呼吸精神散亡故發狂也}
難得之貨令人行妨
\zhushi{妨傷也難得之貨謂金銀珠玉心貪意欲不知饜足則行傷身辱也}
是以聖人為腹
\zhushi{守五性去六情節誌氣養神明}
不為目
\zhushi{目不妄視妄視泄精於外}
故去彼取此
\zhushi{去彼目之妄視取此腹之養性}


\section{厭恥第十三}

寵辱若驚
\zhushi{身寵亦驚身辱亦驚}
貴大患若身
\zhushi{貴畏也若至也謂大患至身故皆驚}
何謂寵辱
\zhushi{問何謂寵何謂辱寵者尊榮辱者恥辱及身還自問者以曉人也}
辱為下
\zhushi{辱為下賤}
得之若驚
\zhushi{得寵榮驚者處高位如臨深危也貴不敢驕富不敢奢}
失之若驚
\zhushi{失者失寵處辱也驚者恐禍重來也}
是謂寵辱若驚
\zhushi{解上得之若驚失之若驚}
何謂貴大患若身
\zhushi{復還自問:何故畏大患至身}
吾所以有大患者為吾有身
\zhushi{吾所以有大患者為吾有身有身憂者勤勞念其饑寒觸情從欲則遇禍患也}
及吾無身吾何有患
\zhushi{使吾無有身體得道自然輕舉升雲出入無間與道通神當有何患}
故貴以身為天下者則可寄天下
\zhushi{言人君貴其身而賤人欲為天下主者則可寄立不可以久也}
愛以身為天下若可托天下
\zhushi{言人君能愛其身非為己也乃欲為萬民之父母以此得為天下主者乃可以托其身於萬民之上長無咎也}


\section{贊玄第十四}


視之不見名曰夷
\zhushi{無色曰夷言一無采色不可得視而見之}
聽之不見名曰希
\zhushi{無聲曰希言一無音聲不可得聽而聞之}
搏之不得名曰微
\zhushi{無形曰微言一無形體不可摶持而得之}
此三者不可致詰
\zhushi{三者謂夷希微也不可致詰者夫無色無聲無形口不能言書不能傳當受之以靜求之以神不可問詰而得之也}
故混而為一
\zhushi{混合也故合於三名之為一}
其上不皦
\zhushi{言一在天上不皦皦光明}
其下不昧
\zhushi{言一在天下不昧昧有所暗冥}
繩繩不可名
\zhushi{繩繩者動行無窮級也不可名者非一色也不可以青黃白黑別非一聲也不可以宮商角徵羽聽非一形也不可以長短大小度之也}
復歸於無物
\zhushi{物質也復當歸之於無質}
是謂無狀之狀
\zhushi{言一無形狀而能為萬物作形狀也}
無物之象
\zhushi{一無物質而為萬物設形象也}
是謂惚恍
\zhushi{一忽忽恍恍者若存若亡不可見之也}
迎之不見其首
\zhushi{一無端末不可預待也除情去欲一自歸之也}
隨之不見其後
\zhushi{言一無影跡不可得而看}
執古之道以禦今之有
\zhushi{聖人執守古道生一以禦物知今當有一也}
能知古始是謂道紀
\zhushi{人能知上古本始有一是謂知道綱紀也}


\section{顯德第十五}

古之善為士者
\zhushi{謂得道之君也}
微妙玄通
\zhushi{玄天也言其誌節玄妙精與天通也}
深不可識
\zhushi{道德深遠不可識知內視若盲反聽若聾莫知所長}
夫唯不可識故強為之容
\zhushi{謂下句也}
與兮若冬涉川
\zhushi{舉事輒加重慎與與兮若冬涉川心難之也}
猶兮若畏四鄰
\zhushi{其進退猶猶如拘制若人犯法畏四鄰知之也}
儼兮其若容
\zhushi{如客畏主人儼然無所造作也}
渙兮若冰之將釋
\zhushi{渙者解散釋者消亡除情去欲日以空虛}
敦兮其若樸
\zhushi{敦者質厚樸者形未分內守精神外無文采也}
曠兮其若谷
\zhushi{曠者寬大谷者空虛不有德功名無所不包也}
渾兮其若濁
\zhushi{渾者守本真濁者不照然與眾合同不自專也}
孰能濁以靜之徐清
\zhushi{孰誰也誰能知水之濁止而靜之徐徐自清也}
孰能安以久動之徐生
\zhushi{誰能安靜以久徐徐以長生也}
保此道者不欲盈
\zhushi{保此徐生之道不欲奢泰盈溢}
夫惟不盈故能蔽不新成
\zhushi{夫為不盈滿之人能守蔽不為新成蔽者匿光榮也新成者貴功名}


\section{歸根第十六}

致虛極
\zhushi{得道之人捐情去欲五內清靜至於虛極}
守靜篤
\zhushi{守清靜行篤厚}
萬物並作
\zhushi{作生也萬物並生也}
吾以觀復
\zhushi{言吾以觀見萬物無不皆歸其本也人當念重其本也}
夫物蕓蕓
\zhushi{蕓蕓者華葉盛也}
各復歸其根
\zhushi{言萬物無不枯落各復反其根而更生也}
歸根曰靜
\zhushi{靜謂根也根安靜柔弱謙卑處下故不復死也}
是謂復命
\zhushi{言安靜者是為復還性命使不死也}
復命曰常
\zhushi{復命使不死乃道之所常行也}
知常曰明
\zhushi{能知道之所常行則為明}
不知常妄作兇
\zhushi{不知道之所常行妄作巧詐則失神明故兇也}
知常容
\zhushi{能知道之所常行去情忘欲無所不包容也}
容乃公
\zhushi{無所不包容則公正無私眾邪莫當}
公乃王
\zhushi{公正無私可以為天下王治身正則形一神明千萬共湊其躬也}
王乃天
\zhushi{能王德合神明乃與天通}
天乃道
\zhushi{德與天通則與道合同也}
道乃久
\zhushi{與道合同乃能長久}
沒身不殆
\zhushi{能公能王通天合道四者純備道德弘遠無殃無咎乃與天地俱沒不危殆也}


\section{淳風第十七}

太上下知有之
\zhushi{太上謂太古無名之君下知有之者下知上有君而不臣事質樸也}
其次親之譽之
\zhushi{其德可見恩惠可稱故親愛而譽之}
其次畏之
\zhushi{設刑法以治之}
其次侮之
\zhushi{禁多令煩不可歸誠故欺侮之}
信不足焉﹝有不信焉﹞
\zhushi{君信不足於下下則應之以不信而欺其君也}
猶兮其貴言
\zhushi{說太上之君舉事猶貴重於言恐離道失自然也}
功成事遂
\zhushi{謂天下太平也}
百姓皆謂我自然
\zhushi{百姓不知君上之德淳厚反以為己自當然也}


\section{俗薄第十八}

大道廢有仁義
\zhushi{大道之時家有孝子戶有忠信仁義不見也大道廢不用惡逆生乃有仁義可傳道}
智慧出有大偽
\zhushi{智慧之君賤德而貴言賤質而貴文下則應之以為大偽奸詐}
六親不和有孝慈
\zhushi{六紀絕親戚不合乃有孝慈相牧養也}
國家昏亂有忠臣
\zhushi{政令不明上下相怨邪僻爭權乃有忠臣匡正其君也此言天下太平不知仁人盡無欲不知廉各自潔己不知貞大道之世仁義沒孝慈滅猶日中盛明眾星失光}


\section{還淳第十九}

絕聖
\zhushi{絕聖制作反初守元五帝垂象倉頡作書不如三皇結繩無文}
棄智
\zhushi{棄智慧反無為}
民利百倍
\zhushi{農事修公無私}
絕仁棄義
\zhushi{絕仁之見恩惠棄義之尚華言}
民復孝慈
\zhushi{德化淳也}
絕巧棄利
\zhushi{絕巧者詐偽亂真也棄利者塞貪路閉權門也}
盜賊無有
\zhushi{上化公正下無邪私}
此三者
\zhushi{謂上三事所棄絕也}
以為文不足
\zhushi{以為文不足者文不足以教民}
故令有所屬
\zhushi{當如下句}
見素抱樸
\zhushi{見素者當抱素守真不尚文飾也抱樸者當抱其質樸以示下故可法則}
少私寡欲
\zhushi{少私者正無私也寡欲者當知足也}


\section{異俗第二十}

絕學
\zhushi{絕學不真不合道文}
無憂
\zhushi{除浮華則無憂患也}
唯之與阿相去幾何
\zhushi{同為應對而相去幾何疾時賤質而貴文}
善之與惡相去若何
\zhushi{善者稱知其所窮極也}
漂兮若無所止
\zhushi{我獨漂漂若飛若揚無所止也誌意在神域也}
眾人皆有以
\zhushi{以有為也}
而我獨頑
\zhushi{我獨無為}
似鄙
\zhushi{鄙似若不逮也}
我獨異於人
\zhushi{我獨與人異也}
而貴食母
\zhushi{食用也母道也我獨貴用道也}


\section{虛心第二十一}

孔德之容
\zhushi{孔大也有大德之人無所不容能受垢濁處謙卑也}
唯道是從
\zhushi{唯獨也大德之人不隨世俗所行獨從於道也}
道之為物唯恍唯忽
\zhushi{道之於萬物獨恍忽往來於其無所定也}
忽兮恍兮其中有象
\zhushi{道唯忽恍無形之中獨有萬物法象}
恍兮忽兮其中有物
\zhushi{道唯恍忽其中有一經營生化因氣立質}
窈兮冥兮其中有精
\zhushi{道唯窈冥無形其中有精實神明相薄陰陽交會也}
其精甚真
\zhushi{言存精氣其妙甚真非有飾也}
其中有信
\zhushi{道匿功藏名其信在中也}
自古及今其名不去
\zhushi{自從也自古至今道常在不去}
以閱眾甫
\zhushi{閱稟也甫始也言道稟與萬物始生從道受氣}
吾何以知眾甫之然哉
\zhushi{吾何以知萬物從道受氣}
以此
\zhushi{此今也以今萬物皆得道精氣而生動作起居非道不然}


\section{益謙第二十二}

曲則全
\zhushi{曲己從眾不自專則全其身也}
枉則直
\zhushi{枉屈己而伸人久久自得直也}
窪則盈
\zhushi{地窪下水流之人謙下德歸之}
弊則新
\zhushi{自受弊薄後己先人天下敬之久久自新也}
少則得
\zhushi{自受取少則得多也天道佑謙神明托虛}
多則惑
\zhushi{財多者惑於所守學多者惑於所聞}
是以聖人抱一為天下式
\zhushi{抱守也式法也聖人守一乃知萬事故能為天下法式也}
不自見故明
\zhushi{聖人不以其目視千裏之外也乃因天下之目以視故能明達也}
不自是故彰
\zhushi{聖人不自以為是而非人故能彰顯於世}
不自伐故有功
\zhushi{伐取也聖人德化流行不自取其美故有功於天下}
不自矜故長
\zhushi{矜大也聖人不自貴大故能久不危}
夫唯不爭故天下莫能與之爭
\zhushi{此言天下賢與不肖無能與不爭者爭也}
古之所謂曲則全者豈虛言哉
\zhushi{傳古言曲從則全身此言非虛妄也}
誠全而歸之
\zhushi{誠實也能行曲從者實其肌體歸之於父母無有傷害也}


\section{虛無第二十三}

希言自然
\zhushi{希言者謂愛言也愛言者自然之道}
故飄風不終朝驟雨不終日
\zhushi{飄風疾風也驟雨暴雨也言疾不能長暴不能久也}
孰為此者?天地
\zhushi{孰誰也誰為此飄風暴雨者乎?天地所為}
天地尚不能久
\zhushi{不能終於朝暮也}
而況於人乎?
\zhushi{天地至神合為飄風暴雨尚不能使終朝至暮何況人欲為暴卒乎}
故從事於道者
\zhushi{從為也人為事當如道安靜不當如飄風驟雨也}
道者同於道
\zhushi{道者謂好道人也同於道者所謂與道同也}
德者同於德
\zhushi{德者謂好德之人也同於德者所謂與德同也}
失者同於失
\zhushi{失謂任己而失人也同於失者所謂與失同也}
同於道者道亦樂得之
\zhushi{與道同者道亦樂得之也}
同於德者德亦樂得之
\zhushi{與德同者德亦樂得之也}
同於失者失亦樂失之
\zhushi{與失同者失亦樂失之也}
信不足焉
\zhushi{君信不足於下下則應君以不信也}
有不信焉
\zhushi{此言物類相歸同聲相應同氣相求雲從龍風從虎水流濕火就燥自然之類也}


\section{苦恩第二十四}

跂者不立
\zhushi{跂進也謂貪權慕名進取功榮則不可久立身行道也}
跨者不行
\zhushi{自以為貴而跨於人眾共蔽之使不得行}
自見者不明
\zhushi{人自見其形容以為好自見其所行以為應道殊不知其形醜操行之鄙}
自是者不彰
\zhushi{自以為是而非人眾共蔽之使不得彰明}
自伐者無功
\zhushi{所謂輒自伐取其功美即失有功於人也}
自矜者不長
\zhushi{好自矜大者不可以長久}
其其於道也曰:餘食贅行
\zhushi{贅貪也使此自矜伐之人在治國之道日賦斂餘祿食以為貪行}
物或惡之
\zhushi{此人在位動欲傷害故物無有不畏惡之者}
故有道者不處也
\zhushi{言有道之人不居其國也}


\section{象元第二十五}

有物混成先天地生
\zhushi{謂道無形混沌而成萬物乃在天地之前}
寂兮寥兮獨立而不改
\zhushi{寂者無音聲寥者空無形獨立者無匹雙不改者化有常}
周行而不殆
\zhushi{道通行天地無所不入在陽不焦托蔭不腐無不貫穿而不危怠也}
可以為天下母
\zhushi{道育養萬物精氣如母之養子}
吾不知其名字之曰道
\zhushi{我不見道之形容不知當何以名之見萬物皆從道所生故字之曰道}
強為之名曰大
\zhushi{不知其名強曰大者高而無上羅而無外無不包容故曰大也}
大曰逝
\zhushi{其為大非若天常在上非若地常在下乃復逝去無常處所也}
逝曰遠
\zhushi{言遠者窮乎無窮布氣天地無所不通也}
遠曰反
\zhushi{言其遠不越絕乃復反在人身也}
故道大天大地大王亦大
\zhushi{道大者包羅天地無所不容也天大者無所不蓋也地大者無所不載也王大者無所不制也}
域中有四大
\zhushi{四大道天地王也凡有稱有名則非其極也言道則有所由有所由然後謂之為道然則是道稱中之大也不若無稱之大也無稱不可而得為名曰域也天地王皆在乎無稱之內也故曰域中有四大者也}
而王居其一焉
\zhushi{八極之內有四大王居其一也}
人法地
\zhushi{人當法地安靜和柔也種之得五谷掘之得甘泉勞而不怨也有功而不制也}
地法天
\zhushi{天湛泊不動施而不求報生長萬物無所收取}
天法道
\zhushi{道清靜不言陰行精氣萬物自成也}
道法自然
\zhushi{道性自然無所法也}


\section{重德第二十六}

重為輕根
\zhushi{人君不重則不尊治身不重則失神草木之花葉輕故零落根重故長存也}
靜為躁君
\zhushi{人君不靜則失威治身不靜則身危龍靜故能變化虎躁故夭虧也躁早報反}
是以聖人終日行不離輜重
\zhushi{輜靜也聖人終日行道不離其靜與重也離音利輜側基反重直用反}
雖有榮觀燕處超然
\zhushi{榮觀謂宮𨵗燕處後妃所居也超然逺避而不處也觀古亂反}
奈何萬乘之主
\zhushi{奈何者疾時主傷痛之辭萬乗之主謂王乗繩證反}
而以身輕天下?
\zhushi{王者至尊而以其身行輕躁乎疾時王奢恣輕淫也}
輕則失臣
\zhushi{王者輕滔則失其臣治身輕淫則失其精}
躁則失君
\zhushi{王者行躁疾則失其君位治身躁疾則失其精神也}


\section{巧用第二十七}

善行無轍跡
\zhushi{善行道者求之於身不下堂不出門故無轍跡}
善言無瑕謫
\zhushi{善言謂擇言而出之則無瑕疵謫過於天下}
善計不用籌策
\zhushi{善以道計事者則守一不移所計不多則不用籌策而可知也}
善閉無關楗而不可開
\zhushi{善以道閉情欲守精神者不如門戶有關楗可得開楗其偃反}
善結無繩約而不可解
\zhushi{善以道結事者乃可結其心不如繩索可得解也}
是以聖人常善救人
\zhushi{聖人所以常教人忠孝者欲以救人性命}
故無棄人
\zhushi{使貴賤各得其所也}
常善救物
\zhushi{聖人所以常教民順四時者欲以救萬物之殘傷}
故無棄物
\zhushi{聖人不賤名而貴玉視之如一}
是謂襲明
\zhushi{聖人善救人物是謂襲明大道}
故善人者不善人之師
\zhushi{人之行善者聖人即以為人師}
不善人者善人之資
\zhushi{資用也人行不善者聖人猶教導使為善得以給用也}
不貴其師
\zhushi{獨無輔也不愛其資無所使也}
雖智大迷
\zhushi{雖自以為智言此人乃大迷惑}
是謂要妙
\zhushi{能通此意是謂知微妙要道也}


\section{反樸第二十八}

知其雄守其雌為天下溪
\zhushi{雄以喻尊神也}


\section{無為第二十九}

將欲取天下
\zhushi{欲為天下主也}
而為之
\zhushi{欲以有為治民}
吾見其不得已
\zhushi{我見其不得天道人心已明矣天道惡煩濁人心惡多欲}
天下神器不可為也
\zhushi{器物也人乃天下之神物也神物好安靜不可以有為治}
為者敗之
\zhushi{以有為治之則敗其質性}
執者失之
\zhushi{強執教之則失其情實生於詐偽也}
故物或行或隨
\zhushi{上所行下必隨之也}
或呴或吹
\zhushi{歔溫也吹寒也有所溫必有所寒也}
或強或羸
\zhushi{有所強大必有所羸弱也}
或載或隳
\zhushi{載安也隳危也有所安必有所危明人君不可以有為治國與治身也}
是以聖人去甚去奢去泰
\zhushi{甚謂貪淫聲色奢謂服飾飲食泰謂宮室臺榭去此三者處中和行無為則天下自化}


\section{儉武第三十}

以道佐人主者
\zhushi{謂人主能以道自輔佐也}
不以兵強天下
\zhushi{以道自佐之主不以兵革順天任德敵人自服}
其事好還
\zhushi{其舉事好還自責不怨於人也}
師之所處荊棘生焉
\zhushi{農事廢田不修}
大軍之後必有兇年
\zhushi{天應之以惡氣即害五谷盡傷人也}
善有果而已
\zhushi{善用兵者當果敢而已不美之}
不敢以取強
\zhushi{不以果敢取強大之名也}
果而勿矜
\zhushi{當果敢謙卑勿自矜大也}
果而勿伐
\zhushi{當果敢推讓勿自伐取其美也}
果而勿驕
\zhushi{驕欺也果敢勿以驕欺人}
果而不得已
\zhushi{當過果敢至誠不當逼迫不得已也}
果而勿強
\zhushi{果敢勿以為強兵堅甲以欺淩人也}
物壯則老
\zhushi{草木壯極則枯落人壯極則衰老也言強者不可以久}
是謂不道
\zhushi{枯老者坐不行道也}
不道早已
\zhushi{不行道者早死}


\section{偃武第三十一}

處之
\zhushi{上將軍居右喪禮尚右死人貴陰也}
殺人之眾以哀悲泣之
\zhushi{傷己德薄不能以道化人而害無辜之民}
戰勝以喪禮處之
\zhushi{古者戰勝將軍居喪主禮之位素服而哭之明君子貴德而賤兵不得以而誅不祥心不樂之比於喪也知後世用兵不已故悲痛之}


\section{聖德第三十二}

道常無名
\zhushi{道能陰能陽能弛能張能存能亡故無常名也}
樸雖小天下莫敢臣
\zhushi{道樸雖小微妙無形天下不敢有臣使道者也}
侯王若能守之萬物將自賔
\zhushi{侯王若能守道無為萬物將自賓服從於德也}
天地相合以降甘露
\zhushi{侯王動作能與天相應和天即降下甘露善瑞也}
民莫之令而自均
\zhushi{天降甘露善瑞則萬物莫有教令之者皆自均調若一也}
始制有名
\zhushi{始道也有名萬物也道無名能制於有名無形能制於有形也}
名亦既有
\zhushi{既盡也有名之物盡有情欲叛道離德故身毀辱也}
夫亦將知之
\zhushi{人能法道行德天亦將自知之}
知之可以不殆
\zhushi{天知之則神靈佑助不復危怠}
譬道之在天下猶川谷之與江海
\zhushi{譬言道之在天下與人相應和如川谷與江海相流通也}


\section{辯德第三十三}

知人者智
\zhushi{能知人好惡是為智}
自知者明
\zhushi{人能自知賢與不肖是為反聽無聲內視無形故為明也}
勝人者有力
\zhushi{能勝人者不過以威力也}
自勝者強
\zhushi{人能自勝己情欲則天下無有能與己爭者故為強也}
知足者富
\zhushi{人能知足則長保福祿故為富也}
強行者有誌
\zhushi{人能強力行善則為有意於道道亦有意於人}
不失其所者乆
\zhushi{人能自節養不失其所受天之精氣則可以長久}
死而不亡者壽
\zhushi{目不妄視耳不妄聽口不妄言則無怨惡於天下故長壽}


\section{任成第三十四}

大道泛兮
\zhushi{言道泛泛若浮若沈若有若無視之不見說之難殊泛音泛}
其可左右
\zhushi{道可左右無所不宜}
萬物恃之而生
\zhushi{恃待也萬物皆恃道而生}
而不辭
\zhushi{道不辭謝而逆止也}
功成不名有
\zhushi{有道不名其有功也}
愛養萬物而不為主
\zhushi{道雖愛養萬物不如人主有所收取}
常無欲可名於小
\zhushi{道匿德藏名怕然無為似若微小也}
萬物歸焉而不為主
\zhushi{萬物皆歸道受氣道非如人主有所禁止也}
可名為大
\zhushi{萬物橫來橫去使名自在故不若於大也}
是以聖人終不為大
\zhushi{聖人法道匿德藏名不為滿大}
故能成其大
\zhushi{聖人以身師導不言而化萬事修治故能成其大}


\section{仁德第三十五}

執大象天下往
\zhushi{執守也象道也聖人守大道則天下萬民移心歸往之也治身則天降神明往來於己也}
往而不害安平太
\zhushi{萬物歸往而不傷害則國家安寕而致太平矣治身不害神明則身安而大壽也}
樂與餌過客止
\zhushi{餌美也過客一也人能樂美於道則一留止也一者去盈而處虛忽忽如過客}
道之出口淡乎其無味
\zhushi{道出入於口淡淡非如五味有酸鹹苦甘辛也}
視之不足見
\zhushi{足德也道無形非若五色有青黃赤白黑可得見也}
聽之不足聞
\zhushi{道非若五音有宮商角徵羽可得聽聞也}
用之不足既
\zhushi{用道治國則國安民昌治身則壽命延長無有既盡時也}


\section{微明第三十六}

將欲歙之必固張之
\zhushi{先開張之者欲極其奢淫}
將欲弱之必固強之
\zhushi{先強大之者欲使遇禍患}
將欲廢之必固興之
\zhushi{先興之者欲使其驕危}
將欲奪之必固與之
\zhushi{先與之者欲極其貪心}
是謂微明
\zhushi{此四事其道微其效明也}
柔弱勝剛強
\zhushi{柔弱者久長剛強者先亡也}
魚不可脫於淵
\zhushi{魚脫於淵謂去剛得柔不可復制焉}
國之利器不可以示人
\zhushi{利器者謂權道也治國權者不可以示執事之臣也治身道者不可以示非其人也}


\section{為政第三十七}

道常無為而無不為
\zhushi{道以無為為常也}
侯王若能守之萬物將自化
\zhushi{言侯王若能守道萬物將自化效於己也}
化而欲作吾將鎮之以無名之樸
\zhushi{吾身也無明之樸道德也萬物已化效於己也復欲作巧偽者侯王當身鎮撫以道德也}
無名之樸夫亦將無欲不欲以靜
\zhushi{言侯王鎮撫以道德民亦將不欲故當以清靜導化之也}
天下將自定
\zhushi{能如是者天下將自正定也}


\section{論德第三十八}

上德不德
\zhushi{上德謂太古無名號之君德大無上故言上德也不德者言其不以德教民因循自然養人性命其德不見故言不德也}
是以有德
\zhushi{言其德合於天地和氣流行民德以全也}
下德不失德
\zhushi{下德謂號謚之君德不及上德故言下德也不失德者其德可見其功可稱也}
是以無德
\zhushi{以有名號及其身故}
上德無為
\zhushi{謂法道安靜無所施為也}
而無以為
\zhushi{言無以名號為也}
下德為之
\zhushi{言為教令施政事也}
而有以為
\zhushi{言以為己取名號也}
上仁為之
\zhushi{上仁謂行仁之君其仁無上故言上仁為之者為人恩也}
而無以為
\zhushi{功成事立無以執為}
上義為之
\zhushi{為義以斷割也}
而有以為
\zhushi{動作以為己殺人以成威賊下以自奉也}
上禮為之
\zhushi{謂上禮之君其禮無上故言上禮為之者言為禮制度序威儀也}
而莫之應
\zhushi{言禮華盛實衰飾偽煩多動則離道不可應也}
則攘臂而扔之
\zhushi{言禮煩多不可應上下忿爭故攘臂相仍引}
故失道而後德
\zhushi{言道衰而德化生也}
失德而後仁
\zhushi{言德衰而仁愛見也}
失仁而後義
\zhushi{言仁衰而分義明也}
失義而後禮
\zhushi{言義衰則失禮聘行玉帛也}
夫禮者忠信之薄
\zhushi{言禮廢本治末忠信日以衰薄}
而亂之首
\zhushi{禮者賤質而貴文故正直日以少邪亂日以生}
前識者道之華
\zhushi{不知而言知為前識此人失道之實得道之華}
而愚之始
\zhushi{言前識之人愚暗之倡始也}
是以大丈夫處其厚
\zhushi{大丈夫謂得道之君也處其厚者謂處身於敦樸}
不居其薄
\zhushi{不處身違道為世煩亂也}
處其實
\zhushi{處忠信也}
不居其華
\zhushi{不尚華言也}
故去彼取此
\zhushi{去彼華薄取此厚實}


\section{法本第三十九}


昔之得一者:
\zhushi{昔往也一無為道之子也}
天得一以清
\zhushi{言天得一故能垂象清明}
地得一以寧
\zhushi{言地得一故能安靜不動搖}
神得一以靈
\zhushi{言神得一故能變化無形}
谷得一以盈
\zhushi{言谷得一故能盈滿而不絕也}
萬物得一以生
\zhushi{言萬物皆須道以生成也}
侯王得一以為天下貞
\zhushi{言侯王得一故能為天下平正}
其致之
\zhushi{致誡也謂下六事也}
天無以清將恐裂
\zhushi{言天當有陰陽弛張晝夜更用不可但欲清明無已時將恐分裂不為天}
地無以寧將恐發
\zhushi{言地當有高下剛柔節氣五行不可但欲安靜無已時將恐發泄不為地}
神無以靈將恐歇
\zhushi{言神當有王相囚死休廢不可但欲靈變無已時將恐虛歇不為神}
谷無以盈將恐竭
\zhushi{言谷當有盈縮虛實不可但欲盈滿無已時將恐枯竭不為谷}
萬物無以生將恐滅
\zhushi{言萬物當隨時生死不可但欲長生無已時將恐滅亡不為物}
侯王無以貴高將恐蹶
\zhushi{言侯王當屈己以下人汲汲求賢不可但欲貴高於人無已時將恐顛蹶失其位}
故貴以賤為本
\zhushi{言必欲尊貴當以薄賤為本若禹稷躬稼舜陶河濱周公下白屋也}
高以下為基
\zhushi{言必欲尊貴當以下為本基猶築墻造功因卑成高下不堅固後必傾危}
是以侯王自謂孤寡不轂
\zhushi{孤寡喻孤獨不轂喻不能如車轂為眾輻所湊}
此非以賤為本邪?
\zhushi{言侯王至尊貴能以孤寡自稱此非以賤為本乎以曉人?}
非乎!
\zhushi{嗟嘆之辭}
故致數輿無輿
\zhushi{致就也言人就車數之為輻為輪為轂為衡為輿無有名為車者故成為車以喻侯王不以尊號自名故能成其貴}
不欲琭琭如玉珞珞如石
\zhushi{琭琭喻少落落喻多玉少故見貴石多故見賤言不欲如玉為人所貴如石為人所賤當處其中也}

\section{去用第四十}

反者道之動
\zhushi{反本也本者道之所以動動生萬物背之則亡也}
弱者道之用
\zhushi{柔弱者道之所常用故能常久}
天下萬物生於有
\zhushi{天下萬物皆從天地生天地有形位故言生於有也}
有生於無
\zhushi{天地神明蜎飛蠕動皆從道生道無形故言生於無也此言本勝於華弱勝於強謙虛勝盈滿也}


\section{同異第四十一}

章上士聞道勤而行之
\zhushi{上士聞道自勤苦竭力而行之}
中士聞道若存若亡
\zhushi{中士聞道治身以長存治國以太平欣然而存之退見財色榮譽惑於情欲而復亡之也}
下士聞道大笑之
\zhushi{下士貪狠多欲見道柔弱謂之恐懼見道質樸謂之鄙陋故大笑之}
不笑不足以為道
\zhushi{不為下士所笑不足以名為道}
故建言有之:
\zhushi{建設也設言以有道當如下句}
明道若昧
\zhushi{明道之人若暗昧無所見}
進道若退
\zhushi{進取道者若退不及}
夷道若纇
\zhushi{夷平也大道之人不自別殊若多比類也}
上德若谷
\zhushi{上德之人若深谷不恥垢濁也}
大白若辱
\zhushi{大潔白之人若汙辱不自彰顯}
廣德若不足
\zhushi{德行廣大之人若愚頑不足也}
建德若偷
\zhushi{建設道德之人若可偷引使空虛也}
質真若渝
\zhushi{質樸之人若五色有渝淺不明也}
大方無隅
\zhushi{大方正之人無委屈廉隅}
大器晚成
\zhushi{大器之人若九鼎瑚璉不可卒成也}
大音希聲
\zhushi{大音猶雷霆待時而動喻當愛氣希言也}
大象無形
\zhushi{大法象之人質樸無形容}
道隱無名
\zhushi{道潛隱使人無能指名也}
夫惟道善貸且成
\zhushi{成就也言道善稟貸人精氣且成就之也}


\section{道化第四十二}

道生一
\zhushi{道使所生者一也}
一生二
\zhushi{一生陰與陽也}
二生三
\zhushi{陰陽生和清濁三氣分為天地人也}
三生萬物
\zhushi{天地人共生萬物也天施地化人長養之也}
萬物負陰而抱陽
\zhushi{萬物無不負陰而向陽回心而就日}
沖氣以為和
\zhushi{萬物中皆有元氣得以和柔若胸中有藏骨中有髓草木中有空虛與氣通故得久生也}
人之所惡惟孤寡不谷而王公以為稱
\zhushi{孤寡不轂者不祥之名而王公以為稱者處謙卑法空虛和柔}
故物或損之而益
\zhushi{引之不得推之必還}
或益之而損
\zhushi{夫增高者誌崩貪富者致患}
人之所教
\zhushi{謂眾人所教去弱為強去柔為剛}
我亦教之
\zhushi{言我教眾人使去強為弱去剛為柔}
強梁者不得其死
\zhushi{強粱者謂不信玄妙背叛道德不從經教尚勢任力也不得其死者為天命所絕兵刃所伐王法所殺不得以壽命死}
吾將以為教父
\zhushi{父始也老子以強梁之人為教誡之始也}


\section{偏用第四十三}

天下之至柔馳騁天下之至堅
\zhushi{至柔者水也至堅者金石也水能貫堅入剛無所不通}
無有入無間
\zhushi{無有謂道也道無形質故能出入無間通神明濟群生也}
吾是以知無為之有益
\zhushi{吾見道無為而萬物自化成是以知無為之有益於人也}
不言之教
\zhushi{法道不言師之以身}
無為之益
\zhushi{法道無為治身則有益於精神治國則有益於萬民不勞煩也}
天下希及之
\zhushi{天下人主也希能有及道無為之治身治國也}


\section{立戒第四十四}

名與身孰親
\zhushi{名遂則身退也}
身與貨孰多
\zhushi{財多則害身也}
得與亡孰病
\zhushi{好得利則病於行也}
甚愛必大費
\zhushi{甚愛色費精神甚愛財遇禍患所愛者少所亡者多故言大費}
多藏必厚亡
\zhushi{生多藏於府庫死多藏於丘墓生有攻劫之憂死有掘冢探柩之患}
知足不辱
\zhushi{知足之人絕利去欲不辱於身}
知止不殆
\zhushi{知可止則財利不累於身聲色不亂於耳目則身不危殆也}
可以長久
\zhushi{人能知止足則福祿在己治身者神不勞治國者民不擾故可長久}


\section{洪德第四十五}

大成若缺
\zhushi{謂道德大成之君也若缺者滅名藏譽如毀缺不備也}
其用不弊
\zhushi{其用心如是則無敝盡時也}
大盈若沖
\zhushi{謂道德大盈滿之君也若沖者貴不敢驕也富不敢奢也}
其用不窮
\zhushi{其用心如是則無窮盡時也}
大直若屈
\zhushi{大直謂修道法度正直如一也若屈者不與俗人爭若可屈折}
大巧若拙
\zhushi{大巧謂多才術也若拙者亦不敢見其能}
大辯若訥
\zhushi{大辯者智無疑若訥者口無辭}
躁勝寒
\zhushi{勝極也春夏陽氣躁疾於上萬物盛大極則寒寒則零落死亡也言人不當剛躁也}
靜勝熱
\zhushi{秋冬萬物靜於黃泉之下極則熱熱者生之源}
清靜能為天下正
\zhushi{能清靜則為天下之長持身正則無終已時也}


\section{儉欲第四十六}

天下有道
\zhushi{謂人主有道也}
卻走馬以糞
\zhushi{糞者糞田也兵甲不用卻走馬治農田治身者卻陽精以糞其身}
天下無道
\zhushi{謂人主無道也}
戎馬生於郊
\zhushi{戰伐不止戎馬生於郊境之上久不還也}
罪莫大於可欲
\zhushi{好淫色也}
禍莫大於不知足
\zhushi{富貴不能自禁止也}
咎莫大於欲得
\zhushi{欲得人物利且貪也}
故知足之足
\zhushi{守真根也}
常足
\zhushi{無欲心也}


\section{鑒遠第四十七}

不出戶知天下
\zhushi{聖人不出戶以知天下者以己身知人身以己家知人家所以見天下也}
不窺牖見天道
\zhushi{天道與人道同天人相通精氣相貫人君清凈天氣自正人君多欲天氣煩濁吉兇利害皆由於己}
其出彌遠其知彌少
\zhushi{謂去其家觀人家去其身觀人身所觀益遠所見益少也}
是以聖人不行而知
\zhushi{聖人不上天不入淵能知天下者以心知之也}
不見而名
\zhushi{上好道下好德上好武下好力聖人原小知大察內知外}
不為而成
\zhushi{上無所為則下無事家給人足萬物自化就也}


\section{忘知第四十八}

為學日益
\zhushi{學謂政教禮樂之學也日益者情欲文飾日以益多}
為道日損
\zhushi{道謂之自然之道也日損者情欲文飾日以消損}
損之又損
\zhushi{損情欲也又損之所以漸去}
以至於無為
\zhushi{當恬淡如嬰兒無所造為也}
無為而無不為
\zhushi{情欲斷絕德於道合則無所不施無所不為也}
取天下常以無事
\zhushi{取治也治天下當以無事不當以勞煩也}
及其有事不足以取天下
\zhushi{及其好有事則政教煩民不安故不足以治天下也}


\section{任德四十九}

聖人無常心
\zhushi{聖人重改更貴因循若自無心}
以百姓心為心
\zhushi{百姓心之所便聖人因而從之}
善者吾善之
\zhushi{百姓為善聖人因而善之}
不善者吾亦善之
\zhushi{百姓雖有不善者聖人化之使善也}
德善
\zhushi{百姓德化聖人為善}
信者吾信之
\zhushi{百姓為信聖人因而信之}
不信者吾亦信之
\zhushi{百姓為不信聖人化之為信者也}
德信
\zhushi{百姓德化聖人以為信}
聖人在天下怵怵
\zhushi{聖人在天下怵怵常恐怖富貴不敢驕奢}
為天下渾其心
\zhushi{言聖人為天下百姓混濁其心若愚暗不通也}
百姓皆註其耳目
\zhushi{註用也百姓皆用其耳目為聖人視聽也}
聖人皆孩之
\zhushi{聖人愛念百姓如嬰孩赤子長養之而不責望其報}


\section{貴生第五十}

出生入死
\zhushi{出生謂情欲出五內魂靜魄定故生入死謂情欲入於胸臆精勞神惑故死}
生之徒十有三死之徒死十有三
\zhushi{言生死之類各有十三謂九竅四關也其生也目不妄視耳不妄聽鼻不妄嗅口不妄言味手不妄持足不妄行精神不妄施其死也反是也}
人之生動之死地十有三
\zhushi{人知求生動作反之十三死也}
夫何故
\zhushi{問何故動之死地也}
以其求生之厚
\zhushi{所以動之死地者以其求生活之事太厚違道忤天妄行失紀}
蓋以聞善攝生者
\zhushi{攝養也}
路行不遇兕虎
\zhushi{自然遠離害不幹也}
入軍不披甲兵
\zhushi{不好戰以殺人}
兕無投其角虎無所措爪兵無所容其刃
\zhushi{養生之人兕虎無由傷兵刃無從加之也}
夫何故
\zhushi{問兕虎兵甲何故不加害之}
以其無死地
\zhushi{以其不犯十三之死地也言神明營護之此物不敢害}


\section{養德第五十一}

道生之
\zhushi{道生萬物}
德畜之
\zhushi{德一也一主布氣而蓄養}
物形之
\zhushi{一為萬物設形像也}
勢成之
\zhushi{一為萬物作寒暑之勢以成之}
是以萬物莫不尊道而貴德
\zhushi{道德所為無不盡驚動而尊敬之}
道之尊德之貴夫莫之命而常自然
\zhushi{道一不命召萬物而常自然應之如影響}
故道生之德畜之長之育之成之孰之養之覆之
\zhushi{道之於萬物非但生而已乃復長養成孰覆育全其性命人君治國治身亦當如是也}
生而不有
\zhushi{道生萬物不有所取以為利也}
為而不恃
\zhushi{道所施為不恃望其報也}
長而不宰
\zhushi{道長養萬物不宰割以為利也}
是謂玄德
\zhushi{道之所行恩德玄暗不可得見}


\section{歸元第五十二}

天下有始
\zhushi{始有道也}
以為天下母
\zhushi{道為天下萬物之母}
既知其母復知其子
\zhushi{子一也既知道己當復知一也}
既知其子復守其母
\zhushi{己知一當復守道反無為也}
沒身不殆
\zhushi{不危殆也}
塞其兌
\zhushi{兌目也目不妄視也}
閉其門
\zhushi{門口也使口不妄言}
終身不勤
\zhushi{人當塞目不妄視閉口不妄言則終生不勤苦}
開其兌
\zhushi{開目視情欲也}
濟其事
\zhushi{濟益也益情欲之事}
終身不救
\zhushi{禍亂成也}
見小曰明
\zhushi{萌芽未動禍亂未見為小昭然獨見為明}
守柔日強
\zhushi{守柔弱日以強大也}
用其光
\zhushi{用其目光於外視時世之利害}
復歸其明
\zhushi{復當返其光明於內無使精神泄也}
無遺身殃
\zhushi{內視存神不為漏失}
是謂習常
\zhushi{人能行此是謂修習常道}


\section{益證第五十三}

使我介然有知行於大道
\zhushi{介大也老子疾時王不行大道故設此言使我介然有知於政事我則行於大道躬行無為之化}
唯施是畏
\zhushi{唯獨也獨畏有所施為恐失道意欲賞善恐偽善生欲信忠恐詐忠起}
大道甚夷而民好徑
\zhushi{夷平易也徑邪不平正也大道甚平易而民好從邪徑也}
朝甚除
\zhushi{高臺榭宮室修}
田甚蕪
\zhushi{農事廢不耕治}
倉甚虛
\zhushi{五谷傷害國無儲也}
服文彩
\zhushi{好飾偽貴外華}
帶利劍
\zhushi{尚剛強武且奢}
厭飲食財貨有餘
\zhushi{多嗜欲無足時}
是謂盜誇
\zhushi{百姓而君有餘者是由劫盜以為服飾持行誇人不知身死家破親戚並隨也}
非道哉
\zhushi{人君所行如是此非道也復言也哉者痛傷之辭}


\section{修觀第五十四}

天下
\zhushi{以修道之主觀不修道之主也}
吾何以知天下之然哉以此
\zhushi{老子言吾何知天下修道者昌背道者亡以此五事觀而知之也}


\section{玄符第五十五}

含德之厚
\zhushi{謂含懷道德之厚也}
比於赤子
\zhushi{神明保佑含德之人若父母之於赤子也}
毒蟲不螫
\zhushi{蜂蠇蛇虺不螫}
猛獸不據玃鳥不搏
\zhushi{赤子不害於物物亦不害之故太平之世人無貴賤仁心有刺之物還返其本有毒之蟲不傷於人}
骨弱筋柔而握固
\zhushi{赤子筋骨柔弱而持物堅固以其意心不移也}
未知牝牡之合而峻作精之至也
\zhushi{赤子未知男女會合而陰陽作怒者由精氣多之所致也}
終日號而不啞和之至也
\zhushi{赤子從朝至暮啼號聲不變易者和氣多之所至也}
知和日常
\zhushi{人能和氣柔弱有益於人者則為知道之常也}
知常日明
\zhushi{人能知道之常行則日以明達於玄妙也}
益生日祥
\zhushi{祥長也言益生欲自生日以長大}
心使氣日強
\zhushi{心當專一和柔而神氣實內故形柔而反使妄有所為和氣去於中故形體日以剛強也}
物壯則老
\zhushi{萬物壯極則枯老也}
謂之不道
\zhushi{枯老則不得道矣}
不道早已
\zhushi{不得道者早死}


\section{玄德第五十六}

知者不言
\zhushi{知者貴行不貴言也}
言者不知
\zhushi{駟不及舌多言多患}
塞其兌閉其門
\zhushi{塞閉之者欲絕其源}
挫其銳
\zhushi{情欲有所銳為當念道無為以挫止之}
解其紛
\zhushi{紛結恨不休也當念道恬怕以解釋之}
和其光
\zhushi{雖有獨見之明當和之使暗昧不使曜亂}
同其塵
\zhushi{不當自別殊也}
是謂玄同
\zhushi{玄天也人能行此上事是謂與天同道也}
故不可得而親
\zhushi{不以榮譽為樂獨立為哀}
亦不可得而踈
\zhushi{誌靜無欲故與人無怨}
不可得而利
\zhushi{身不欲富貴口不欲五味}
亦不可得而害
\zhushi{不與貪爭利不與勇爭氣}
不可得而貴
\zhushi{不為亂世主不處暗君位}
亦不可得而賤
\zhushi{不以乘權故驕不以失誌故屈}
故為天下貴
\zhushi{其德如此天子不得臣諸侯不得屈與世沈浮容身避害故天下貴也}


\section{淳風第五十七}

以正治國
\zhushi{以至也天使正身之人使有國也}
以奇用兵
\zhushi{奇詐也天使詐偽之人使用兵也}
以無事取天下
\zhushi{以無事無為之人使取天下為之主}
吾何以知其然哉以此
\zhushi{此今也老子言我何以知天意然哉以今日所見知}
天下多忌諱而民彌貧
\zhushi{天下謂人主也忌諱者防禁也今煩則奸生禁多則下詐相殆故貧}
民多利器國家滋昏
\zhushi{利器者權也民多權則視者眩於目聽者惑於耳上下不親故國家昏亂}
人多伎巧奇物滋起
\zhushi{人謂人君百裏諸侯也多技巧謂刻畫宮觀雕琢章服奇物滋起下則化上飾金鏤玉文繡彩色日以滋甚}
法物滋彰盜賊多有
\zhushi{法物好物也珍好之物滋生彰著則農事廢饑寒並至而盜賊多有也}
故聖人雲:
\zhushi{謂下事也}
我無為而民自化
\zhushi{聖人言:我修道承天無所改作而民自化成也}
我好靜而民自正
\zhushi{聖人言:我好靜不言不教而民自忠正也}
我無事而民自富
\zhushi{我無徭役徵召之事民安其業故皆自富也}
我無欲而民自樸
\zhushi{我常無欲去華文微服飾民則隨我為質樸也聖人言:我修道守真絕去六情民自隨我而清也}


\section{順化第五十八}

其政悶悶
\zhushi{其政教寬大悶悶昧昧似若不明也}
其民醇醇
\zhushi{政教寬大故民醇醇富厚相親睦也}
其政察察
\zhushi{其政教急疾言決於口聽決於耳也}
其民缺缺
\zhushi{政教急疾民不聊生故缺缺日以踈薄}
禍兮福所倚
\zhushi{倚因也夫福因禍而生人遭禍而能悔過責己修道行善則禍去福來}
福兮禍所伏
\zhushi{禍伏匿於福中人得福而為驕恣則福去禍來}
孰知其極
\zhushi{禍福更相生誰能知其窮極時}
其無正
\zhushi{無不也謂人君不正其身其無國也}
正復為奇
\zhushi{奇詐也人君不正下雖正復化上為詐也}
善復為訞
\zhushi{善人皆復化上為訞祥也}
人之迷其日固久
\zhushi{言人君迷惑失正以來其日已固久}
是以聖人方而不割
\zhushi{聖人行方正者欲以率下不以割截人也}
廉而不害
\zhushi{聖人廉清欲以化民不以傷害人也今則不然正己以害人也}
直而不肆
\zhushi{肆申也聖人雖直曲己從人不自申也}
光而不曜
\zhushi{聖人雖有獨見之明當如暗昧不以曜亂人也}


\section{守道第五十九}

治人
\zhushi{謂人君治理人民}
事天
\zhushi{事用也當用天道順四時}
莫若嗇
\zhushi{嗇愛惜也治國者當愛民財不為奢泰治身者當愛精氣不為放逸}
夫為嗇是謂早服
\zhushi{早先也服得也夫獨愛民財愛精氣則能先得天道也}
早服謂之重積德
\zhushi{先得天道是謂重積得於己也}
重積德則無不克
\zhushi{克勝也重積德於己則無不勝}
無不克則莫知其極
\zhushi{無不克勝則莫知有知己德之窮極也}
莫知其極可以有國
\zhushi{莫知己德者有極則可以有社稷為民致福}
有國之母可以長久
\zhushi{國身同也母道也人能保身中之道使精氣不勞五神不苦則可以長久}
是謂深根固蒂
\zhushi{人能以氣為根以精為蒂如樹根不深則拔蒂不堅則落言當深藏其氣固守其精使無漏泄}
長生久視之道
\zhushi{深根固蒂者乃長生久視之道}


\section{居位第六十}

治大國者若烹小鮮
\zhushi{鮮魚烹小魚不去腸不去鱗不敢撓恐其糜也治國煩則下亂治身煩則精散}
以道蒞天下其鬼不神
\zhushi{以道德居位治天下則鬼不敢以其精神犯人也}
非其鬼不神其神不傷人
\zhushi{其鬼非無精神也非不入正不能傷自然之人}
非其神不傷人聖人亦不傷
\zhushi{非鬼神不能傷害人以聖人在位不傷害人故鬼不敢幹之也}
夫兩不相傷
\zhushi{鬼與聖人俱兩不相傷也}
故德交歸焉
\zhushi{夫兩不相傷則人得治於陽鬼神得治於陰人得保全其性命鬼得保其精神故德交歸焉}


\section{謙德第六十一}

大國者下流
\zhushi{治大國當如居下流不逆細微}
天下之交
\zhushi{大國天下士民之所交會}
天下之牝
\zhushi{牝者陰類也柔謙和而不昌也}
牝常以靜勝牡
\zhushi{女所以能屈男陰勝陽以安靜不先求之也}
以靜為下
\zhushi{陰道以安靜為謙下}
故大國以下小國則取小國
\zhushi{能謙下之則常有之}
小國以下大國則取大國
\zhushi{此言國無大小能持謙畜人則無過失也}
故或下以取或下而取
\zhushi{下者謂大國以下小國小國以下大國更以義相取}
大國不過欲兼畜人
\zhushi{大國不失下則兼並小國而牧畜之}
小國不過欲入事人
\zhushi{使為臣仆}
夫兩者各得其所欲大者宜為下
\zhushi{大國小國各欲得其所大國又宜為謙下}


\section{為道第六十二}

道者萬物之奧
\zhushi{奧藏也道為萬物之藏無所不容也}
善人之寶
\zhushi{善人以道為身寶不敢違也}
不善人之所保
\zhushi{道者不善人之保倚也遭患逢急猶知自悔卑下}
美言可以市
\zhushi{美言者獨可於市耳夫市交易而退不相宜善言美語求者欲疾得賣者欲疾售也}
尊行可以加入
\zhushi{加別也人有尊貴之行可以別異於凡人未足以尊道}
人之不善何棄之有
\zhushi{人雖不善當以道化之蓋三皇之前無有棄民德化淳也}
故立天子置三公
\zhushi{欲使教化不善之人}
雖有拱璧以先駟馬不如坐進此道
\zhushi{雖有美璧先駟馬而至故不如坐進此道}
古之所以貴此道者何不日以求得?
\zhushi{古之所以貴此道者不日日遠行求索近得之於身}
有罪以免耶
\zhushi{有罪謂遭亂世暗君妄行形誅修道則可以解死免於眾也}
故為天下貴
\zhushi{道德洞遠無不覆濟全身治國恬然無為故可為天下貴也}


\section{恩始第六十三}

為無為
\zhushi{因成循故無所造作}
事無事
\zhushi{預有備除煩省事也}
味無味
\zhushi{深思遠慮味道意也}
大小多少
\zhushi{陳其戒令也欲大反小欲多反少自然之道也}
報怨以德
\zhushi{修道行善絕禍於未生也}
圖難於其易
\zhushi{欲圖難事當於易時未及成也}
為大於其細
\zhushi{欲為大事必作於小禍亂從小來也}
天下難事必作於易天下大事必作於細
\zhushi{從易生難從細生著}
是以聖人終不為大故能成其大
\zhushi{處謙虛天下共歸之也}
夫輕諾必寡信
\zhushi{不重言也}
多易必多難
\zhushi{不慎患也}
是以聖人猶難之
\zhushi{聖人動作舉事猶進退重難之欲塞其源也}
故終無難矣
\zhushi{聖人終生無患難之事猶避害深也}


\section{守微第六十四}

其安易持
\zhushi{治身治國安靜者易守持也}
其未兆易謀
\zhushi{情欲禍患未有形兆時易謀止也}
其脆易破
\zhushi{禍亂未動於朝情欲未見於色如脆弱易破除}
其微易散
\zhushi{其未彰著微小易散去也}
為之於未有
\zhushi{欲有所為當於未有萌芽之時塞其端也}
治之於未亂
\zhushi{治身治國於未亂之時當豫閉其門也}
合抱之木生於毫末
\zhushi{從小成大}
九層之臺起於累土
\zhushi{從卑立高}
千裏之行始於足下
\zhushi{從近至遠}
為者敗之
\zhushi{有為於事廢於自然有為於義廢於仁有為於色廢於精神也}
執者失之
\zhushi{執利遇患執道全身堅持不得推讓反還}
是以聖人無為故無敗
\zhushi{聖人不為華文不為色利不為殘賊故無敗壞}
無執故無失
\zhushi{聖人有德以教愚有財以與貧無所執藏故無所失於人也}
民之從事常於幾成而敗之
\zhushi{從為也民之為事常於功德幾成而貪位好名奢泰盈滿而自敗之也}
慎終如始則無敗事
\zhushi{終當如始不當懈怠}
是以聖人欲不欲
\zhushi{聖人欲人所不欲人欲彰顯聖人欲伏光人欲文飾聖人欲質樸人欲色聖人欲於德}
不貴難得之貨
\zhushi{聖人不眩為服不賤石而貴玉}
學不學
\zhushi{聖人學人所不能學人學智詐聖人學自然人學治世聖人學治身守道真也}
復眾人之所過
\zhushi{眾人學問反過本為末過實為華復之者使反本也}
以輔萬物之自然
\zhushi{教人反本實者欲以輔助萬物自然之性也}
而不敢為
\zhushi{聖人動作因循不敢有所造為恐遠本也}


\section{淳德第六十五}

古之善為道者非以明民將以愚之
\zhushi{說古之善以道治身及治國者不以道教民明智巧詐也將以道德教民使質樸不詐偽}
民之難治以其智多
\zhushi{民之所以難治者以其智多而為巧偽}
故以智治國國之賊
\zhushi{使智慧之人治國之政事必遠道德妄作威福為國之賊也}
不以智治國國之福
\zhushi{不使智慧之人治國之政事則民守正直不為邪飾上下相親君臣同力故為國之福也}
知此兩者亦稽式
\zhushi{兩者謂智與不智也常能智者為賊不智者為福是治身治國之法式也}
常知稽式是謂玄德
\zhushi{玄天也能知治身及治國之法式是謂與天同德也}
玄德深矣遠矣
\zhushi{玄德之人深不可測遠不可及也}
與物反矣!
\zhushi{玄德之人與萬物反異萬物欲益己玄德施與人也}
然後乃至於大順
\zhushi{玄德與萬物反異故能至大順順天理也}


\section{後己第六十六}

江海所以能為百谷王者以其善下之故能為百谷王
\zhushi{江海以卑故眾流歸之若民歸就王以卑下故能為百谷王也}
是以欲上民
\zhushi{欲在民之上也}
必以言下之
\zhushi{法江海處謙虛}
欲先民
\zhushi{欲在民之前也}
必以身後之
\zhushi{先人而後己也}
是以聖人處上而民不重
\zhushi{聖人在民上為主不以尊貴虐下故民戴而不為重}
處前而民不害
\zhushi{聖人在民前不以光明蔽後民親之若父母無有欲害之心也}
是以天下樂推而不厭
\zhushi{聖人恩深愛厚視民如赤子故天下樂推進以為主無有厭也}
以其不爭
\zhushi{天下無厭聖人時是由聖人不與人爭先後也}
故天下莫能與之爭
\zhushi{言人皆有為無有與吾爭無為}


\section{三寶第六十七}

天下皆謂我大似不肖
\zhushi{老子言:天下謂我德大我則佯愚似不肖}
夫唯大故似不肖
\zhushi{唯獨名德大者為身害故佯愚似若不肖無所分別無所割截不賤人而自責}
若肖久矣
\zhushi{肖善也謂辨惠也若大辨惠之人身高自貴行察察之政所從來久矣}
其細也夫
\zhushi{言辨惠者唯如小人非長者}
我有三寶持而保之
\zhushi{老子言:我有三寶抱持而保倚}
一曰慈
\zhushi{愛百姓若赤子}
二曰儉
\zhushi{賦斂若取之於己也}
三曰不敢為天下先
\zhushi{執謙退不為倡始也}
慈故能勇
\zhushi{以慈仁故能勇於忠孝也}
儉故能廣
\zhushi{天子身能節儉故民日用廣矣}
不敢為天下先
\zhushi{不為天下首先}
故能成器長
\zhushi{成器長謂得道人也我能為得道人之長也}
今舍慈且勇
\zhushi{今世人舍慈仁但為勇武}
舍儉且廣
\zhushi{舍其儉約但為奢泰}
舍後且先
\zhushi{舍其後己但為人先}
死矣!
\zhushi{所行如此動入死地}
夫慈以戰則勝以守則固
\zhushi{夫慈仁者百姓親附並心一意故以戰則勝敵以守衛則堅固}
天將救之以慈衛之
\zhushi{天將救助善人必與慈仁之性使能自營助也}


\section{配天第六十八}

善為士者不武
\zhushi{言貴道德不好武力也}
善戰者不怒
\zhushi{善以道戰者禁邪於胸心絕禍於未萌無所誅怒也}
善勝敵者不與
\zhushi{善以道勝敵者附近以仁來遠以德不與敵爭而敵自服也}
善用人者為之下
\zhushi{善用人自輔佐者常為人執謙下也}
是謂不爭之德
\zhushi{謂上為之下也是乃不與人爭之道德也}
是謂用人之力
\zhushi{能身為人下是謂用人臣之力也}
是謂配天古之極
\zhushi{能行此者德配天也是乃古之極要道也}


\section{玄用第六十九}

用兵有言:
\zhushi{陳用兵之道老子疾時用兵故托己設其義也}
吾不敢為主而為客
\zhushi{主先也不敢先舉兵客者和而不倡用兵當承天而後動}
不敢進寸而退尺
\zhushi{侵人境界利人財寶為進閉門守城為退}
是謂行無行
\zhushi{彼遂不止為天下賊雖行誅之不成行列也}
攘無臂
\zhushi{雖欲大怒若無臂可攘也}
扔無敵
\zhushi{雖欲仍引之若無敵可仍也}
執無兵
\zhushi{雖欲執持之若無兵刃可持用也何者?傷彼之民罹罪於天遭不道之君湣忍喪之痛也}
禍莫大於輕敵
\zhushi{夫禍亂之害莫大於欺輕敵家侵取不休輕戰貪財也}
輕敵幾喪吾寶
\zhushi{幾近也寶身也欺輕敵者近喪身也}
故抗兵相加
\zhushi{兩敵戰也}
哀者勝矣
\zhushi{哀者慈仁士卒不遠於死}


\section{知難第七十}

章吾言甚易知甚易行
\zhushi{老子言:吾所言省而易知約而易行}
天下莫能知莫能行
\zhushi{人惡柔弱好剛強也}
言有宗事有君
\zhushi{我所言有宗祖根本事有君臣上下世人不知者非我之無德心與我之反也}
夫唯無知是以不我知
\zhushi{夫唯世人之無知者是我德之暗不見於外窮微極妙故無知也}
知我者希則我者貴
\zhushi{希少也唯達道者乃能知我故為貴也}
是以聖人被褐懷玉
\zhushi{被褐者薄外懷玉者厚內匿寶藏德不以示人也}


\section{知病第七十一}

知不知上
\zhushi{知道言不知是乃德之上}
不知知病
\zhushi{不知道言知是乃德之病}
夫唯病病是以不病
\zhushi{夫唯能病苦眾人有強知之病是以不自病也}
聖人不病以其病病是以不病
\zhushi{聖人無此強知之病者以其常苦眾人有此病以此非人故不自病夫聖人懷通達之知托於不知者欲使天下質樸忠正各守純性小人不知道意而妄行強知之事以自顯著內傷精神減壽消年也}


\section{愛己第七十二}

民不畏威則大威至
\zhushi{威害也人不畏小害則大害至大害者謂死亡也畏之者當愛精神承天順地也}
無狹其所居
\zhushi{謂心居神當寬柔不當急狹也}
無厭其所生
\zhushi{人所以生者以有精神托空虛喜清靜飲食不節忽道念色邪僻滿腹為伐本厭神也}
夫唯不厭是以不厭
\zhushi{夫唯獨不厭精神之人洗心濯垢恬泊無欲則精神居之不厭也}
是以聖人自知不自見
\zhushi{自知己之得失不自顯見德美於外藏之於內}
自愛不自貴
\zhushi{自愛其身以保精氣不自貴高榮名於世}
故去彼取此
\zhushi{去彼自見自貴取此自知自愛}


\section{任為第七十三}

勇於敢則殺
\zhushi{勇敢有為則殺其身}
勇於不敢則活
\zhushi{勇於不敢有為則活其身}
此兩者
\zhushi{謂敢與不敢也}
或利或害
\zhushi{活身為利殺身為害}
天之所惡
\zhushi{惡有為也}
孰知其故?
\zhushi{誰能知天意之故而不犯?}
是以聖人猶難之
\zhushi{言聖人之明德猶難於勇敢況無聖人之德而欲行之乎?}
天之道不爭而善勝
\zhushi{天不與人爭貴賤而人自畏之}
不言而善應
\zhushi{天不言萬物自動以應時}
不召而自來
\zhushi{天不呼召萬物皆負陰而向陽}
繟然而善謀
\zhushi{繟寬也天道雖寬博善謀慮人事修善行惡各蒙其報也}
天網恢恢疏而不失
\zhushi{天所網羅恢恢甚大雖疏遠司察人善惡無有所失}


\section{制惑第七十四}

民不畏死
\zhushi{治國者刑罰酷深民不聊生故不畏死也治身者嗜欲傷神貪財殺身民不知畏之也}
奈何以死懼之?
\zhushi{人君不寬刑罰教民去情欲奈何設刑法以死懼之?}
若使民常畏死
\zhushi{當除己之所殘克教民去利欲也}
而為奇者吾得執而殺之孰敢?
\zhushi{以道教化而民不從反為奇巧乃應王法執而殺之誰敢有犯者?老子疾時王不先道德化之而先刑罰也}
常有司殺者
\zhushi{司殺者謂天居高臨下司察人過天網恢恢疏而不失也}
夫代司殺者是謂代大匠斫
\zhushi{天道至明司殺有常猶春生夏長秋收冬藏鬥杓運移以節度行之人君欲代殺之是猶拙夫代大匠斫木勞而無功也}
夫代大匠斫者希有不傷手矣
\zhushi{人君行刑罰猶拙夫代大匠斫則方圓不得其理還自傷代天殺者失紀綱不得其紀綱還受其殃也}


\section{貪損第七十五}

民之饑以其上食稅之多是以饑
\zhushi{人民所以饑寒者以其君上稅食下太多民皆化上為貪叛道違德故饑}
民之難治以其上之有為是以難治
\zhushi{民之不可治者以其君上多欲好有為也是以其民化上有為情偽難治}
民之輕死以其上求生之厚
\zhushi{人民所以侵犯死者以其求生活之道太厚貪利以自危}
是以輕死
\zhushi{以求生太厚之故輕入死地也}
夫唯無以生為者是賢於貴生
\zhushi{夫唯獨無以生為務者爵祿不幹於意財利不入於身天子不得臣諸侯不得使則賢於貴生也}


\section{戒強第七十六}

人之生也柔弱
\zhushi{人生含和氣抱精神故柔弱也}
其死也堅強
\zhushi{人死和氣竭精神亡故堅強也}
萬物草木之生也柔脆
\zhushi{和氣存也}
其死也枯槁
\zhushi{和氣去也}
故堅強者死之徒柔弱者生之徒
\zhushi{以上二事觀之知堅強者死柔弱者生也}
是以兵強則不勝
\zhushi{強大之兵輕戰樂殺毒流怨結眾弱為一強故不勝}
木強則共
\zhushi{本強大則枝葉共生其上}
強大處下柔弱處上
\zhushi{興物造功大木處下小物處上天道抑強扶弱自然之效}


\section{天道第七十七}

天之道其猶張弓乎!
\zhushi{天道暗昧舉物類以為喻也}
髙者抑之下者舉之有餘者損之不足者與之
\zhushi{言張弓和調之如是乃可用夫抑髙舉下損強益弜天之道也}
天之道損有餘而補不足
\zhushi{天道損有餘而益謙常以中和為上}
人之道則不然損不足以奉有餘
\zhushi{人道則與天道反世俗之人損貧以奉富奪弱以益強也}
孰能有餘以奉天下?唯有道者
\zhushi{言誰能居有餘之位自省爵祿以奉天下不足者乎?唯有道之君能行也}
是以聖人為而不恃
\zhushi{聖人為德施不恃其報也}
功成而不處
\zhushi{功成事就不處其位}
其不欲見賢
\zhushi{不欲使人知己之賢匿功不居榮畏天損有餘也}


\section{任信第七十八}

天下柔弱莫過於水
\zhushi{圓中則圓方中則方壅之則止決之則行}
而攻堅強者莫知能勝
\zhushi{水能懷山襄陵磨鐵消銅莫能勝水而成功也}
其無以易之
\zhushi{夫攻堅強者無以易於水}
弱之勝強
\zhushi{水能滅火陰能消陽}
柔之勝剛
\zhushi{舌柔齒剛齒先舌亡}
天下莫不知
\zhushi{知柔弱者久長剛強者折傷}
莫能行
\zhushi{恥謙卑好強梁}
故聖人雲:
\zhushi{謂下事也}
受國之垢是謂社稷主
\zhushi{人君能受國之垢濁者若江海不逆小流則能長保其社稷為一國之君主也}
受國不祥是謂天下王
\zhushi{人君能引過自與代民受不祥之殃則可以王天下}
正言若反
\zhushi{此乃正直之言世人不知以為反言}


\section{任契第七十九}

和大怨
\zhushi{殺人者死傷人者刑以相和報}
必有餘怨
\zhushi{任刑者失人情必有餘怨及於良人也}
安可以為善?
\zhushi{言一人則先天心安可以和怨為善?}
是以聖人執左契而不責於人
\zhushi{古者聖人執左契合符信也無文書法律刻契合符以為信也但刻契為信不責人以他事也}
有德司契
\zhushi{有德之君司察契信而已}
無德司徹
\zhushi{無德之君背其契信司人所失}
天道無親常與善人
\zhushi{天道無有親疏唯與善人則與司契同也}


\section{獨立第八十}

小國寡民
\zhushi{聖人雖治大國猶以為小儉約不奢泰民雖眾猶若寡少不敢勞之也}
使有什伯人之器而不用
\zhushi{使民各有部曲什伯貴賤不相犯也器謂農人之器而不用不徵召奪民良時也}
使民重死而不遠徙
\zhushi{君能為民興利除害各得其所則民重死而貪生也政令不煩則民安其業故不遠遷徙離其常處也}
雖有舟輿無所乘之
\zhushi{清靜無為不作煩華不好出入遊娛也}
雖有甲兵無所陳之
\zhushi{無怨惡於天下}
使民復結繩而用之
\zhushi{去文反質信無欺也}
甘其食
\zhushi{甘其蔬食不漁食百姓也}
美其服
\zhushi{美其惡衣不貴五色}
安其居
\zhushi{安其茅茨不好文飾之屋}
樂其俗
\zhushi{樂其質樸之俗不轉移也}
鄰國相望雞犬之聲相聞
\zhushi{相去近也}
民至老不相往來
\zhushi{其無情欲}


\section{顯質第八十一}

信言不美
\zhushi{信者如其實也不美者樸且質也}
美言不信
\zhushi{美言者滋美之華辭不信者飾偽多空虛也}
善者不辯
\zhushi{善者以道修身也不彩文也}
辯者不善
\zhushi{辯者謂巧言也不善者舌致患也山有玉掘其山水有珠濁其淵辯口多言亡其身}
知者不博
\zhushi{知者謂知道之士不博者守一元也}
博者不知
\zhushi{博者多見聞也不知者失要真也}
聖人不積
\zhushi{聖人積德不積財有德以教愚有財以與貧也}
既以為人己愈有
\zhushi{既以為人施設德化己愈有德}
既以與人己愈多
\zhushi{既以財賄布施與人而財益多如日月之光無有盡時}
天之道利而不害
\zhushi{天生萬物愛育之令長大無所傷害也}
聖人之道為而不爭
\zhushi{聖人法天所施為化成事就不與下爭功名故能全其聖功也}


\end{document}
